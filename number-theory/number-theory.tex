% $File: number-theory.tex
% $Date: Wed Mar 07 20:43:59 2012 +0800
% $Author: wyx <ppwwyyxxc@gmail.com>

\documentclass[a4paper]{article}
\usepackage{fontspec,zhspacing,amsmath,amssymb,verbatim,minted}
\usepackage[hyperfootnotes=false,colorlinks,linkcolor=blue,anchorcolor=blue,citecolor=blue]{hyperref}
\usepackage[sorting=none]{biblatex}
\usepackage[dvips]{graphicx}
\usepackage{subfigure}

\zhspacing
\setlength{\parindent}{1em}
\renewcommand{\baselinestretch}{1.4}

\renewcommand{\abstractname}{摘要}
\renewcommand{\contentsname}{目录}
\renewcommand{\figurename}{图}
\defbibheading{bibliography}{\section{参考文献}}
\bibliography{refs.bib}
% \figref{label}: reference to a figure
\newcommand{\figref}[1]{\hyperref[fig:#1]{图\ref*{fig:#1}}}
% \secref{label}: reference to a section
\newcommand{\secref}[1]{\hyperref[sec:#1]{\ref*{sec:#1}节}}

\newtheorem{theorem}{Theorem}[section]
\newtheorem{lemma}[theorem]{Lemma}
\newtheorem{proposition}[theorem]{Proposition}
\newtheorem{corollary}[theorem]{Corollary}

\let\Oldsum\sum
\renewcommand{\sum}{\displaystyle\Oldsum}
\let\Oldprod\prod
\renewcommand{\prod}{\displaystyle\Oldprod}

\newenvironment{proof}[1][Proof]{\begin{trivlist}
\item[\hskip \labelsep {\bfseries #1}]}{\end{trivlist}}
\newenvironment{definition}[1][Definition]{\begin{trivlist}
\item[\hskip \labelsep {\bfseries #1}]}{\end{trivlist}}
\newenvironment{example}[1][Example]{\begin{trivlist}
\item[\hskip \labelsep {\bfseries #1}]}{\end{trivlist}}
\newenvironment{remark}[1][Remark]{\begin{trivlist}
\item[\hskip \labelsep {\bfseries #1}]}{\end{trivlist}}

%\newcommand{\qed}{\nobreak \ifvmode \relax \else
%      \ifdim\lastskip<1.5em \hskip-\lastskip
%      \hskip1.5em plus0em minus0.5em \fi \nobreak
%      \vrule height0.75em width0.5em depth0.25em\fi}
%




%% $File: mint-defs.tex
% $Date: Sat Feb 16 22:59:11 2013 +0800
% $Author: wyx <ppwwyyxxc@gmail.com>

\usepackage{xparse}


% \inputmintedConfigured[additional minted options]{lang}{file path}{
\newcommand{\inputmintedConfigured}[3][]{\inputminted[fontsize=\footnotesize,
	label=#3,linenos,frame=lines,framesep=0.8em,tabsize=4,#1]{#2}{#3}}

% \phpsrc[additional minted options]{file path}: show highlighted php source
\newcommand{\phpsrc}[2][]{\inputmintedConfigured[#1]{php}{#2}}
% \phpsrcpart[additional minted options]{file path}{first line}{last line}: show part of highlighted php source
\newcommand{\phpsrcpart}[4][]{\phpsrc[firstline=#3,firstnumber=#3,lastline=#4,#1]{#2}}
% \phpsrceg{example id}
\newcommand{\phpeg}[1]{\inputminted[startinline,
	firstline=2,lastline=2]{php}{res/php-src-eg/#1.php}}

\newcommand{\txtsrc}[2][]{\inputmintedConfigured[#1]{text}{#2}}
\newcommand{\txtsrcpart}[4][]{\txtsrc[firstline=#3,firstnumber=#3,lastline=#4,#1]{#2}}

\newcommand{\pysrc}[2][]{\inputmintedConfigured[#1]{py}{#2}}
\newcommand{\pysrcpart}[4][]{\pysrc[firstline=#3,firstnumber=#3,lastline=#4,#1]{#2}}

\newcommand{\confsrc}[2][]{\inputmintedConfigured[#1]{squidconf}{#2}}
\newcommand{\confsrcpart}[4][]{\confsrc[firstline=#3,firstnumber=#3,lastline=#4,#1]{#2}}

\newcommand{\cppsrc}[2][]{\inputmintedConfigured[#1]{cpp}{#2}}
\newcommand{\cppsrcpart}[4][]{\cppsrc[firstline=#3,firstnumber=#3,lastline=#4,#1]{#2}}

\renewcommand{\P}[1]{\text{P}\left(#1\right)}
\renewcommand{\Pr}[1]{\text{Pr}\left\{#1\right\}}
\newcommand{\Px}[2]{\text{P}_{#1}\left(#2\right)}
\newcommand{\E}[1]{\text{E}\left[#1\right]}
\newcommand{\Ex}[1]{\text{E}#1}
\newcommand{\Var}[1]{\text{Var}\left[#1\right]}
%\newcommand{\Cov}[2]{\text{Cov}\left[#1,#2\right]}
%\newcommand{\Cov}[1]{\text{Cov}\left[#1 \right]}
\renewcommand{\T}[1]{\Theta\left(#1\right)}
\newcommand{\real}{\mathbb{R}}
\newcommand{\card}[1]{\left\|#1\right\|}
\newtheorem{lemma}{Lemma}

\NewDocumentCommand\Cov{mg}{
    \text{Cov}\left[ #1 \IfNoValueTF{#2}{}{,#2}\right]
 }

\newcommand{\qed}{\hfill \ensuremath{\Box}}



\title{Number Theory}
\author{ppwwyyxxc@gmail.com}

\begin{document}
\maketitle

\tableofcontents
%% $File: order.tex
% $Date: Mon Mar 05 15:53:40 2012 +0800
% $Author: wyx <ppwwyyxxc@gmail.com> 

\section{Order}
Definition: $  \delta_m(a)=\min \{x|a^x \equiv 1 \pmod m\} $ 

推广: $ a^d\equiv b^d \pmod p$,取倒数$ bb'\equiv1\pmod p$,则$ d=\delta_p(ab')$.性质类似
\\
若$ a^n \equiv 1 \pmod m$ ,则 $ \delta_m(a)\mid n $ .
否则设$ n=\delta_m(a)q+r,a^r \equiv a^n \equiv 1 $且$ r<\delta_m(a)$.矛盾

特别地,若 $ a^p\equiv 1 \pmod m$ , 则 $ \delta_m(a)=1 $ 或$ p$ 

Mersenne's Prime的因子特征:$ q\mid 2^p-1\Rightarrow p=\delta_q(2)\mid (q-1)\Rightarrow q\equiv 1 \pmod{2p} $ 
\\

$ (a,p)=1$,则在$ p^0,p^1,\ldots p^{a-1} \pmod a$中抽屉得 $ \exists d \le a-1: a|p^d-1\Rightarrow \delta_p(a)\le a-1$
\\

证明$ n \nmid 2^n-1$:

设$ n$最小素因子$ p$,则$ \delta_p(2) \mid (p-1,n)=(p-1,\dfrac{n}{p^{\alpha}})=1$.

或者利用递降:$ n\rightarrow \delta_n(2) ; (a,b)\rightarrow (b,(a,b))$
\\

$ n\mid 2^n+1\Rightarrow \delta_p(2)\mid (2n,p-1)=(2,p-1)\Rightarrow p=3$.

事实上有$ 3^k \mid 2^{3^k}+1$,以及$ n\mid 2^n+1 \Rightarrow m\mid 2^m+1,m=2^n+1$
\\

反证$ n \nmid m^{n-1}+1$:

设$ n-1=2^kt\Rightarrow m^{2^kt}\equiv -1 \pmod p\Rightarrow \delta_p(m)\nmid 2^kt,\delta_p(m)\mid 2^{k+1}t\Rightarrow 2^{k+1}\mid\delta_p(m)$

又$ \delta_p(m)\mid p-1,\therefore p \equiv 1 \pmod{2^{k+1}}.$
考虑到$ p$为$ n$任意素因子$ \Rightarrow n\equiv 1 \pmod{2^{k+1}}$,与$ n-1=2^kt$矛盾
\\
\\

关于$ r_k=\delta_{p^k}(a)$的求解($ p$为奇数).设$ p^{k_0}\parallel a^{r_1}-1$

i)当$ 1\le k \le k_0$时,$ a^{r_k}\equiv1 \pmod{p^k \rightarrow p}\Rightarrow r_1\mid r_k$

$ a^r\equiv1 \pmod{p^{k_0}\rightarrow p^k}\Rightarrow r_k\mid r_1. \therefore r_k=r_1$

ii)当$ k\ge k_0$时,对$k $归纳证明$ r_k=r_1 p^{k-k_0}$

引理:$ p^{k_0+i}\parallel a^{r_1p^i}-1\Leftrightarrow a^{r_1p^i}=1+p^{k_0+i}u,(u,p)=1$.

证明:归纳.
$ a^{r_1p^{i+1}}=(a^{r_1p^i})^p=(1+p^{k_0+i}u)^p=1+p^{k_0+i+1}(1+C_p^2u^2p^{k_0+i-1})$

引理中取$ i=k-k_0,$则$ a^{r_1p^{k-k_0}}\equiv 1 \pmod{p^k}\Rightarrow r_k \mid r_1p^{k-k_0}$

$ a^{r_k}\equiv 1\pmod{p^k \rightarrow p^{k-1}}\Rightarrow r_{k-1}\mid r_k \therefore r_1p^{k-k_0-1}\mid r_k \mid r_1p^{k-k_0}$

再取$ i=k-k_0-1$,由$ p^{k-1}\parallel a^{r_1p^{k-k_0-1}}-1$知$ a^{r_1p^{k-k_0-1}}\not \equiv 1 \pmod{p^k}.$


$  \therefore r_k=
\begin{cases}
r_1, & 1\le k \le k_0 \\
r_1p^{k-k_0}, & k\ge k_0 
\end{cases} 
$
\\

$ r_k=\delta_{2^k}(a)$的求解:

i)$ a=4k+1,2^{k_0}\parallel a-1,r_k=
\begin{cases}
1,& 1\le k \le k_0 \\ 
2^{k-k_0},& k\ge k_0 
\end{cases}
$ 

ii)$ a=4k+3,2^{k_0}\parallel a+1,r_k=
\begin{cases} 
1,& k=1 \\ 
2,& 2\le k\le k_0+1 \\ 
2^{k-k_0},& k\ge k_0+1 
\end{cases}$
\\

引理的推广:$ a^{mrp^i}=1+p^{k_0+i}u,(u,p)=1.$

设$ n=mrp^i$可得一命题:$ r=\delta_p(a),r\mid n,p^{\alpha}\parallel n\Rightarrow p^{\alpha}\parallel \dfrac{a^n-1}{a^r-1} $
\\

反证:对给定$ n,a$,不存在无穷个$ k,s.t.n^k\mid a^k-1$

i)$ n$含奇因子$ p,a^k\equiv 1 \pmod {p^k}\Rightarrow r_k=r_1p^{k-k_0}\mid k\Rightarrow k>r_1p^{k-k_0}\ge 3^{k-k_0}$不可能无穷个

ii)若$ k$为奇,则$ 2^k\mid a^k-1\Rightarrow 2^k\mid a-1,$只有有限个$ k$.

若$ k$为偶,$ a^{2l}\equiv 1 \pmod{2^l}.$当$ l>k_0$时,$ 2^{l-k_0}\mid l$不可能无穷个.
\\
\\
$ r_k=\delta_m(a^k)=\dfrac{r_1}{(r_1,k)}$.

证:设$ r'=\dfrac{r_1}{(r_1,k)}$.显然$ (r',\dfrac{k}{(r_1,k)})=1$

由定义,$ a^{kr_k}\equiv 1 \pmod m, a^{kr'}\equiv 1 \pmod m. \Rightarrow  r_1 \mid kr_k,r_k\mid r'$

$ \therefore r' =\dfrac{r_1}{(r_1,k)}\mid \dfrac{k}{(r_1,k)}r_k\Rightarrow r' \mid r_k. \therefore r'=r_k$

推论:有$ \varphi(r_1)$个$ k,s.t.(r_1,k)=1.$又$ a^0,a^1,\cdots,a^{r_1-1}$对模$ m$不同余

所以其中至少有$ \varphi(r_1)$个$ k,s.t.\delta_m(a^k)=r_1$.

即在模$ m$的一个缩系中至少有$ \varphi(r_1)$个$ k,s.t.r_k=r_1$
\\

若$ (m_1,m_2)=1,$则$ \delta_{m_1m_2}(a)=[\delta_{m_1}(a),\delta_{m_2}(a)]=[r_1,r_2]$

证:i)显然对$ \forall n \mid m,\delta_n(a)\mid \delta_m(a). \therefore [r_1,r_2] \mid \delta_{m_1m_2}(a)$ 

ii)$ a^{[r_1,r_2]}\equiv 1 \pmod{m_1,m_2 \rightarrow m_1m_2}\Rightarrow \delta_{m_1m_2}\mid [r_1,r_2]$

推论:$ (m_1,m_2)=1$,则对$ \forall a_1,a_2,\exists a,s.t.\delta_{m_1m_2}(a)=[\delta_{m_1}(a_1),\delta_{m_2}(a_2)]$

证:取$ a\equiv a_i \pmod{m_i},i=1,2$.则$ \delta_{m_i}(a)=\delta_{m_i}(a_i)$.由原命题即证.
\\

$ \min\{ n|2^n\equiv -1 \pmod p\}<\delta_p(2),$否则 $,2^{n-\delta_p(2)}\equiv 2^n \equiv -1$,与最小性矛盾.
\\

$ p=3k+2$时,$ x$取$ \mod p$完系,则$ x^3$亦遍历.否则$ x^3\equiv y^3\Rightarrow \delta_p(xy^{-1})\mid(3,p-1)=1$.矛盾
\\

无穷数列$  \dfrac{1}{9}(10^{k\delta_{9a}(10)}-1)(k\ge1) $ 中,每项均由1组成且均为$ a$的倍数
\\

奇素$ p,p^n|a^p-1\Rightarrow p^{n-1}|a-1$
\\

$ \exists n,s.t.p\parallel2^n-1\Rightarrow p\parallel 2^{p-1}-1$

证:假设$ p^2 \mid 2^{p-1}-1\Rightarrow \delta_{p^2}(2)\mid p-1.$

又$ 2^{pn}-1=(2^n-1)(2^{n(p-1)}+2^{n(p-2)}+\cdots+2^n+1)\equiv(2^n-1)p\equiv 0\pmod{p^2}$

$ \therefore \delta_{p^2}(2)\mid (pn,p-1)=(n,p-1)\mid n\Rightarrow 2^n\equiv 1 \pmod {p^2}$.矛盾
\\

奇素数$ p,{pn+1}$中含无穷多素数:

证:取$ x^p-1$的因子$ q,s.t.q \nmid x-1$ (why can?).则$ \delta_q(x)=p$.

设$ (q-1,p)=d$ ,则$ \exists u,v,s.t.u(q-1)+vp=d\Rightarrow x^d\equiv(x^{q-1})^n(x^p)^v\equiv 1 \pmod q\Rightarrow d=p$ 

$ \therefore p\mid q-1\Leftrightarrow q=pn+1$.又$ \dfrac{x^p-1}{x-1}$含无穷个素因子$ q$,可知$ {pn+1}$中有无穷多素数

%% $File: wilson.tex
% $Date: Tue Feb 07 22:22:39 2012 +0800
% Author:  ppwwyyxxc@gmail.com

\section{Wilson}
Wilson定理:素数$ p \Leftrightarrow (p-1)!\equiv -1 \pmod p$

可推出:$ (p-k)!(k-1)!\equiv (-1)^k \pmod p$

Lagrange定理:$ f(x)=\sum_{i=1}^{n}{a_ix^i},p \nmid a_i$,则$ n$次同余方程
$ f(x) \equiv 0  \pmod p$的解数$  \le n$

对$ n$归纳反证.假设$ n+1$个解$ c_1\cdots c_{n+1}$,则$ f(x)-f(c_1)=(x-c_1)h(x)$

于是$ c_2,\cdots c_{n+1}$均为$ n-1$次同余方程$ h(x)\equiv 0 \pmod p$的解.矛盾

推论:若$ f(x)\equiv 0$的解数$ >n$,则各项系数均被$ p$整除.
\\

$ f(x)=(x-1)(x-2)\cdots (x-p+1)=\sum_{i=0}^{p-1}{s_ix^i}\equiv x^{p-1}-1 \pmod p $(Fermat)

$ \Rightarrow f(x)-x^{p-1}+1=\sum_{i=1}^{p-2}{s_ix^i}+(p-1)!+1\equiv 0 \pmod p$ 

由Lagrange得$ p\mid s_i,1\le i \le p-2$
\\

$ f(x)=f(p-x)\Rightarrow f(-x)=f(p+x)$ 

$ \Rightarrow x^{p-1}+\sum_{i=1}^{p-2}{(-1)^is_ix^i}=(p+x)^{p-1}+\sum_{i=1}^{p-2}{s_i(p+x)^i}$

两边模$ p^2$得,$ x^{p-1}+\sum_{i=1}^{p-2}{(-1)^is_ix^i}\equiv x^{p-1}+(p-1)px^{p-2}+\sum_{i=1}^{p-2}{s_ix^i}$

$ \Rightarrow \sum_{i=1}^{p-2}{[(-1)^i-1]s_ix^i}\equiv p(p-1)x^{p-2}\pmod {p^2}$ 

$ \Rightarrow \sum_{i=1}^{p-3}{[(-1)^i-1]s_ix^i}\equiv 0 \pmod {p^2}(\because s_{p-2}=-\frac{p(p-1)}{2})$

$ \Rightarrow p^2 \mid s_1,s_3,\cdots s_{p-4}$

推论$ p^2 \mid s_1=(p-1)!(1+\frac{1}{2}+\cdots+\frac{1}{p-1}),p \mid s_{p-3}=\sum_{1\le i\le j\le p-1}{ij}$
\\

Wilson定理推广:

T1.奇素数$ p$,设$ c=\varphi(p^l),r_1,\cdots,r_c$是mod$ p^l$的缩系,则$ \prod_{i=1}^{c}{r_i}\equiv -1 \pmod{p^l}$

证:对每个$ r_i$有唯一$ r_j$使$ r_ir_j\equiv1\pmod{p^l}.$

此时$ r_i=r_j\Leftrightarrow r_i\equiv 1,-1\pmod{p^l}$配对即得证.
\\

T2:$ \because \varphi(p^l)=\varphi(2p^l),$取$ r_i'=\left\{ \begin{matrix}r_i,2 \nmid r_i\\ r_i+p^l,2\mid r_i\end{matrix} \right.$,则$ r_i'$为mod$ 2p^l$的缩系

且$ \prod_{i=1}^{c}{r_i'}\equiv -1 \pmod{p^l},2 \mid  \prod_{i=1}^{c}{r_i'}+1\Rightarrow  \prod_{i=1}^{c}{r_i'}\equiv -1 \pmod{2p^l}\\$
\\

T3:设$ c=\varphi(2^l),l\ge 3,r_1\cdots r_c$是mod$ 2^l$的缩系.则$  \prod_{i=1}^{c}{r_i}\equiv 1 \pmod{2^l}$

证:同T1,使$ r_i=r_j$的充要条件是$ \frac{r_i-1}{2}\frac{r_i+1}{2}\equiv 0 \pmod{2^{l-2}}\Leftrightarrow r_i\equiv 1,2^{l-1}\pm 1,2^l-1$

%% $File: special-numbers.tex
% $Date: Tue Feb 07 22:22:18 2012 +0800
% Author:  ppwwyyxxc@gmail.com

\section{Special Numbers}

$ 2^k-1$为素数$ \Rightarrow k$为素数 

Mersenne's Prime$ \Leftrightarrow $Perfect Number:($ \sigma(n)=2n \Leftrightarrow n=\frac{1}{2}M_{(p)}(M_{(p)}+1)$ )

i)若$ n=2^{p-1}M_{(p)},$则$ \sigma(n)=(1+2+\cdots+2^{p-1})(1+M_{(p)})=2n$

ii)若$ n$为偶完全数,易知$ n\ne 2^k,$

于是设$ n=2^{m-1}u\Rightarrow 2^mu=\sigma(n)=\sigma(2^{m-1})\sigma(u)=(2^m-1)\sigma(u)$

从而$ \sigma(u)=u+\frac{u}{2^m-1}\Rightarrow u=2^m-1,$且$ 2^m-1$为素数
\\

$ n^k+1$为素数$ \Rightarrow k$为2的幂

Fermat's Number

$ n\ge 5$时$ 2^n\equiv 2^{n-4} \pmod 10\Rightarrow F_n(n\ge 2)\equiv 7 \pmod 10$

$F_n=2^{2^n}+1,F_0F_1\cdots F_{n-1}+2=F_n\Rightarrow (F_n,F_m)=1 \Rightarrow $素数无穷多

在任意形如$ a^x-1$中设$ x=2^kq$,则可分解$ a^x-1=(a^q)^{2^k}-1=\cdots$

设$ F_n$的任一素因子$ p,2^{2^n}\equiv -1 \pmod p \therefore \delta_p(2)\mid 2^{n+1}\Rightarrow \delta_p(2)=2^k$

又$ 2^{2^k}\equiv 1 \pmod p,2^{2^n}\equiv -1 \pmod p\Rightarrow k>n\Rightarrow k=n+1$

有结论:$ \delta_p(2)=2^{n+1},2^{n+1}\mid p-1$

一般地,$ a^{2^k}\equiv -1 \pmod m\Rightarrow \delta_m(a)=2^{k+1}$
\\

伪素数递归构造

$ n \mid 2^n-2\Rightarrow 2^{2^n-1}-2=2^{nk+1}-2=2(2^{nk}-1)\equiv0\pmod{2^n-1}$
\\

孪生素数$ p,q=p+2.p+q\mid p^p+q^q$ 

证$RHS =p^p+(p+2)^p+(p+2)^{p+2}-(p+2)^p=A(p+q)+q^p(p+1)(p+3)$
\\

Sylvester's Sequence
$    a_1=2,a_n=a_{n-1}^2-a_{n-1}+1\Rightarrow \sum_{i=1}^{n}{\frac{1}{a_i}}+\frac{1}{\prod_{i=1}^{n}{a_i}}=1 $ 

$  a_{n+1}=\prod_{i=1}^{n}{a_i}+1\Rightarrow (a_n,a_m)=1,a_n\ge 2^{n-1}  $ 

最佳单位分数逼近:对$ \forall \{x_n\},\sum_{i=1}^{n}{\frac{1}{x_i}}<1\Rightarrow \sum_{i=1}^{n}{\frac{1}{x_i}}\le \sum_{i=1}^{n}{\frac{1}{a_i}}$

证:设有$ \sum_{i=1}^{j}{\frac{1}{x_i}}\le \sum_{i=1}^{j}{\frac{1}{a_i}},j=1,2,\cdots n,\sum_{i=1}^{n+1}{\frac{1}{x_i}}> \sum_{i=1}^{n+1}{a_i}$

作Abel变换:\[  n+1=\sum_{i=1}^{n+1}{\frac{x_i}{x_i}}=x_{n+1}\sum_{i=1}^{n+1}{\frac{1}{x_i}}+\sum_{j=1}^{n}{(\sum_{i=1}^{j}{\frac{1}{x_i}})(x_j-x_{j+1})} \]

\[ > x_{n+1}\sum_{i=1}^{n+1}{\frac{1}{a_i}}+\sum_{j=1}^{n}{(\sum_{i=1}^{j}{\frac{1}{a_i}})(x_j-x_{j+1})}=\sum_{i=1}^{n+1}{\frac{x_i}{a_i}}\]

\[ \ge (n+1) \sqrt[n+1]{\frac{\prod{x_i}}{\prod{a_i}}}\Rightarrow \prod_{i=1}^{n+1}{x_i}<\prod_{i=1}^{n+1}{a_i}\Rightarrow \sum_{i=1}^{n+1}{\frac{1}{x_i}}<\sum_{i=1}^{n+1}{\frac{1}{a_i}}\] 
\\

Sophie Germain素数$ p$($ 2p+1$也为素数).

若$ p\equiv 3 \pmod 4$,则$ 2p+1 \mid 2^p-1=M_{(p)}$

证:设$ k=2p+1=8t-1,2^{\frac{k-1}{2}}\equiv 1 \pmod k\Leftrightarrow (\frac{2}{k})=1=(-1)^{\frac{k^2-1}{8}}$

%% $File: function.tex
% $Date: Sat Mar 03 13:24:52 2012 +0800
% Author:  ppwwyyxxc@gmail.com

\section{Arithmetic Function}
$ d(n)$约数个数,$ \sigma (n)$约数和,$ \varphi(n)$缩系大小,均有积性

$ n=\prod{p_i^{\alpha_i}}$ 则$d(n)=\prod{(\alpha_i+1)},\sigma(n)=\prod{\frac{p_i^{\alpha_i}-1}{p_i-1}},\varphi(n)=n\prod({1-\frac{1}{p_i}})$
\\

$ d(n)$为奇$ \Leftrightarrow n=k^2$ ;$ \sigma(n)$为奇$ \Leftrightarrow n=k^2,2k^2$

$ \varphi(n)=\varphi(2n)\Leftrightarrow n$为奇.

$ \varphi(n) \mid n\Leftrightarrow n=1,2^{\alpha}3^{\beta}(\alpha \ge 1,\beta \ge 0)$

$ n$的最小正缩系元素和为$ \frac{1}{2}n\varphi(n)$.配对
\\

估界:

$ n$在$ [1,\sqrt{n}]$中约数至多$ \sqrt{n}$个,$ \therefore d(n)\le 2\sqrt{n}$

$ \sigma(n)=\frac{1}{2}\sum_{d \mid n}{d+\frac{n}{d}}\ge \frac{1}{2}d(n)2\sqrt{n}=\sqrt{n}d(n)$

$ \sigma(n)^2 \mathop \le \limits_{cauchy} d(n)\sum_{d\mid n}{d^2}=d(n)\sum_{d\mid n}{(\frac{n}{d})^2}\le n^2d(n)\sum{\frac{1}{k^2}}<2n^2d(n)$

$ \varphi(p^a)=p^a-p^{a-1}>p^{\frac{a}{2}},\varphi(2^a)>\frac{2^{\frac{a}{2}}}{2}\Rightarrow \varphi(n)>\frac{\sqrt{n}}{2}.n$为奇时有$ \varphi(n)>\sqrt{n}$

$ \varphi(n)\le n-1,d(n)+\varphi(n)\le n+1$当$ n$为合数时,$ \varphi(n)\le n-\sqrt{n}.$

\[  \sum_{d \mid n}{\varphi(d)}=\sum_{e_1=0}^{\alpha_1}{\varphi(p_1^{e_1})}\sum_{e_2=0}^{\alpha_2}{\varphi(p_2^{e_2})}\cdots =\prod_{i=1}^{r}{\sum_{j=0}^{\alpha_i}{\varphi(p_i^j)}}=\prod_{i=1}^{r}{p_i^{\alpha_i}}=n\]

\[ \frac{\varphi(mn)}{mn}=\prod_{p \mid mn}{(1-\frac{1}{p})}=\frac{\prod_{p \mid m}{(1-\frac{1}{p})}\prod_{p \mid n}{(1-\frac{1}{p})}}{\prod_{p \mid (m,n)}{(1-\frac{1}{p})}}=\frac{\frac{\varphi(m)}{m}\frac{\varphi(n)}{n}}{\frac{\varphi((m,n))}{(m,n)}} \]
$ \Rightarrow \varphi(mn)\varphi((m,n))=(m,n)\varphi(m)\varphi(n) $ 
\\

$ d(n)=\prod{(\alpha+1)}\ge 2^r,\varphi(n)\ge n \prod{(1-\frac{1}{2})}=\frac{n}{2^r}\Rightarrow d(n)\varphi(n)\ge n$

对$ \pi(n)=$小于$ n$的素数个数估界: 

设$ n=k^2l,k$有$ \sqrt{n}$种取法,$ l$为不同素数积,有$ 2^{\pi(n)}$种取法.

$ n\le \sqrt{n}2^{\pi(n)}\Rightarrow \pi(n)\ge \frac{1}{2}\log_2n$
\\

Fermat-Euler Theorem:
$ (a,m)=1\Rightarrow a^{\varphi(m)}\equiv 1 \pmod m$

证:取$ m$一组缩系$ x_1\cdots x_{\varphi(m)},$则$ ax_i$也构成一组缩系.$ \prod{ax_i}\equiv \prod{x_i}$

推广:$ a^m \equiv a^{m-\varphi(m)} \pmod m$

证:设$ m=m_1m_2:m_1$的素因子均被$ a$整除,而$ (m_2,a)=1,$则$ (m_1,m_2)=1$.

首先有$ a^{\varphi(m_2)}\equiv 1 \pmod{m_2}\Rightarrow a^{\varphi(m)\equiv 1 \pmod{m_2}}\Rightarrow a^m\equiv a^{m-\varphi(m)\pmod{m_2}}$.

于是只需$ a^m\equiv a^{m-\varphi(m)}\pmod{m_1}\Leftrightarrow m_1\mid a^{m-\varphi(m)}$ 

$\Leftrightarrow V_p(m_1)\le (m-\varphi(m))V_p(a)$

又$ V_p(m_1)=V_p(m)\le2^{V_p(m)-1}\le p^{V_p(m)-1}\le p^{V_p(m)-1}\varphi(\frac{m}{p^{V_p(m)}})$ 

$=p^{V_p(m)}\varphi(\frac{m}{p^{V_p(m)}})-\varphi(p^{V_p(m)})\varphi(\frac{m}{p^{V_p(m)}})=p^{V_p(m)}\varphi(\frac{m}{p^{V_p(m)}})-\varphi(m)$ 

$\le m-\varphi(m)\le (m-\varphi(m))V_p(a)$得证.
\\

一些等式:

$ (m,n)=1\Rightarrow m^{\varphi(n)}+n^{\varphi(m)}\equiv 1 \pmod{mn}$

$ a\varphi(a^kb^{k+1})=b\varphi(b^ka^{k+1})$

$n=4k+3 \Rightarrow \forall d \mid n ,d+\frac{n}{d}\equiv 0 \pmod 4\Rightarrow 4 \mid \sigma(n)$
\\

$ (m,n)=1,\{ a_i\}_{1}^{\varphi(m)},\{ b_i\}_1^{\varphi(n)}$为缩系,则

$ S=\{ mb_i+na_j | 1\le j\le \varphi(m),1\le i \le \varphi(n)\}$为mod $ mn$缩系.

$ 1.(S_k,mn)=1; $ 

$ 2.S_i\equiv S_j \pmod n\Rightarrow b_i\equiv b_j \pmod n\Rightarrow i=j;$ 

$ 3.|S|=\varphi(m)\varphi(n)=\varphi(mn)$;

%% $File: gauss.tex
% $Date: Mon Mar 05 15:56:05 2012 +0800
% Author: ppwwyyxxc@gmail.com
\section{Gauss Function} 
$ [x]+[y]\le [x+y];$

$ x+y\in Z\Rightarrow \{x\}+\{y\}=0,1$

$[x]+[y]+[x+y]\le [2x]+[2y] $

$ [\dfrac{m}{n}]\ge \dfrac{m-n+1}{n} $,带余除

$ [\dfrac{x}{m}]=[\dfrac{[x]}{m}]$

$ [a]+[a+\dfrac{1}{n}]+\cdots+[a+\dfrac{n-1}{n}]=[na],n\in N,a \in R$

$ [x+\dfrac{1}{2}]=[2x]-[x]\Rightarrow \sum_{k=0}^{\infty}{[\dfrac{n+2^k}{2^{k+1}}]}=n,$或用二进制证明.

$ [\sqrt{n}+\sqrt{n+1}]=[\sqrt{4n+1}]=[\sqrt{4n+2}]=[\sqrt{4n+3}]=[\sqrt{n}+\sqrt{n+2}]$
\\

$ \sum_{i=1}^{\infty}{[\dfrac{k}{p^i}]}\le \sum_{i=1}^{\infty}{\dfrac{k}{p^i}}\le k\Rightarrow p^k \nmid k!.$

特别地,$ p\ge 3\Rightarrow V_p(k!)\le \dfrac{k}{2},$可用于组合数/阶乘证明中.
\\

$ [\dfrac{x^2}{y}]+[\dfrac{y^2}{x}]=[\dfrac{x^2+y^2}{xy}]+xy\Rightarrow -1<\dfrac{x^2}{y}+\dfrac{y^2}{x}-\dfrac{x^2+y^2}{xy}-xy<2$

$ \Rightarrow \begin{cases} y^3-(1+x^2)y^2-2xy+x^3-x^2<0 \\ y^3-(1+x^2)y^2+xy+x^3-x^2>0 \end{cases}  $设$ y\ge x$

$ \textcircled{1}\Leftrightarrow y(y(y-(1+x^2))-2x)+x^3-x^2<0.$

若$ y\ge x^2+2\Rightarrow LHS\ge y(x-1)^2+x^3-x^2>0$.矛盾

$ \textcircled{2}:$若$ y\le x^2\Rightarrow (LHS)'=3y^2-(2+2x^2)y+x $令其等于$ 0$ 

$\Rightarrow \max\{LHS\}=\max\{f(x),f(x^2)\}\le 0$.矛盾

$ \therefore y=x^2+1$
\\

$ n$阶方格表,对列号为行号的倍数的格子数算两次:

$ \sum_{i=1}^{n}{[\dfrac{n}{i}]}$为第$ i$行所有($ i$的倍数)列,$ \sum_{i=1}^{n}{d(i)}$为第$ i$列所有($ i$的约数)行.两者相等.

另证由$ [\dfrac{n}{i}]-[\dfrac{n-1}{i}]= \begin{cases} 0,i\nmid n\\ 1,i \mid n \end{cases}\Rightarrow f(n)=f(n-1)+d(n)=\cdots$

类似结论:$ \sum_{i=1}^{n}{i[\dfrac{n}{i}]}=\sum_{i=1}^{n}{\sigma(i)}$
\\

$ [ax]=x,x\in N$有$ n$个解

$ \Rightarrow x=[ax]=[a]x+[\{ a\}x]$有$ n$个解$ \Leftrightarrow [a]=1$且$ \{ a\}x<1$有$ n$个解

$ \therefore \{ a\}\in [\dfrac{1}{n},\dfrac{1}{n-1})\Rightarrow a \in [1+\dfrac{1}{n},1+\dfrac{1}{n-1})$
\\



%% $File: diophantine.tex
% $Date: Tue Mar 06 20:28:54 2012 +0800
% Author: ppwwyyxxc@gmail.com
\section{Diophantine Equation} 
Pythagoras: $ a^2+b^2=c^2,(a,b,c)=1,2\mid b$的所有$ N^{+}$上的解为:

$ a=u^2-v^2,b=2uv,c=u^2+v^2,(u,v)=1,u\ge v,2 \nmid u+v$
\\

$ a^2-mab+b^2=k,k\le m$且非平方数,方程无解.
证:假设有最小解$ (a_0,b_0),a_0\ge b_0,$且$ a_0+b_0$,令$ a'=mb_0-a_0$,则$ a'\le 0$或$ a' \ge a_0$

若$ a'=0$则$ k$平方数;若$ a'<0\Rightarrow a_o\ge mb_0+1\Rightarrow a^2-ma_0b_0+b_0^2>m\ge k$

若$ a'\ge a_0\Rightarrow b_0^2-k=a_0a'\ge a_0^2\ge b_0^2\ge b_0^2-k$.每种情况均矛盾.
\\

Pell:
标准Pell方程$ x^2-dy^2=1,d\in \mathbb{N^{+}},d$非平方数必有无穷多解,$ (x_0,y_0)$称为基本解,
所有解为$ x_n+\sqrt{d}y_n=(x_0+\sqrt{d}y_0)^n$

$ x^2-dy^2=C$若有解则必有无穷多解.设最小解$ (x_1,y_1)$,则$ x_n+\sqrt{d}y_n=(x_1+\sqrt{d}y_1)(x_0+\sqrt{d}y_0)^{n-1}$为部分解

对上式中改变符号:$
\begin{cases}
x_n+\sqrt{d}y_n=(x_0+\sqrt{d}y_0)^n \\
x_n-\sqrt{d}y_n=(x_0-\sqrt{d}y_0)^n   
\end{cases}  \Rightarrow	$
两式加减即可求出通项

特征根为$ \lambda_{1,2}=x_0\pm\sqrt{d}y_0\Rightarrow $特征方程
$ \lambda^2-2x_0\lambda+1=0\Rightarrow$ 

递推关系$ \begin{cases} x_{n+1}=2x_0x_n-x_{n-1}\\ y_{n+1}=2x_0y_n-y_{n-1}\end{cases}$
\\

$ x^2-dy^2=-1,$设$ \sqrt{d}$的连分数周期为$ l$,则$ l$为偶$ \Leftrightarrow $Pell方程无解

特别地,素数$ p=4k+1$时,$ x^2-py^2=-1$	有解

$ x^2-dy^2=-1$若有解则必有无穷多解.设最小解$ (x_1,y_1)$,则所有解为$ x_n+\sqrt{d}y_n=(x_1+\sqrt{d}y_1)^{2n-1}$
\\

Lemma:

$ x^2-dy^2=4$的整数解$ x=u,y=v$是正整数解$ \Leftrightarrow \dfrac{u+\sqrt{d}v}{2}>1$

证:$ \dfrac{u+\sqrt{d}v}{2}>1\Rightarrow \dfrac{u-\sqrt{d}v}{2}\in(0,1)$,
两式相加得$ u>1$即$ u\ge 2$

又$ 1>\dfrac{u-\sqrt{d}v}{2}\ge 1-\dfrac{\sqrt{d}v}{2}\Rightarrow v>0$.得证

$ x^2-dy^2=4$若有最小解$ (x_1,y_1)$,则所有解$ \dfrac{x_n+\sqrt{d}y_n}{2}=(\dfrac{x_1+\sqrt{d}y_1}{2})^n$

证:设数列$x,y, \dfrac{x_n+\sqrt{d}y_n}{2}=(\dfrac{x_1+\sqrt{d}y_1}{2})^n$,
假设有解$ (a,b)$不在其中

不妨设$ ( \dfrac{x_1+\sqrt{d}y_1}{2})^{n+1}>\dfrac{a+\sqrt{d}b}{2}>(\dfrac{x_1+\sqrt{d}y_1}{2})^n$,则

$   \dfrac{x_1+\sqrt{d}y_1}{2}=\dfrac{x_n^2-dy_n^2}{4}\dfrac{x_1+\sqrt{d}y_1}{2}=(\dfrac{x_1+\sqrt{d}y_1}{2})^{n+1}\dfrac{x_n-\sqrt{d}y_n}{2}>\dfrac{a+\sqrt{d}b}{2}\dfrac{x_n-\sqrt{d}y_n}{2} $ 

$  \overset{\text{def}}{=} \dfrac{s+\sqrt{d}t}{2}>(\dfrac{x_1+\sqrt{d}y_1}{2})^n\dfrac{x_n-\sqrt{d}y_n}{2}=1$ 
\\

解方程$ x^2-5y^2=-4$,有恒等式$ (\dfrac{3x-5y}{2})^2-5(\dfrac{3y-x}{2})^2=x^2-5y^2$

$ \therefore (x,y)\rightarrow (\dfrac{3x-5y}{2},\dfrac{3y-x}{2}),$

可证当$ y>1$时总有$ 0<\dfrac{3x-5y}{2}<x,0<\dfrac{3y-x}{2}<y$,完成递降

递推可求得其所有解为$ \dfrac{x_n+\sqrt{5}y_n}{2}=(\dfrac{x_1+\sqrt{5}y_1}{2})^{2n+1}$

事实上也有恒等式$ (ax-Dby)^2-D(ay-bx)^2=(a^2-Db^2)(x^2-Dy^2)=x^2-Dy^2$,但用它递降不总能做到边界
\\

(Catalan)$ a^x-b^y=1$只有$ 3^2-2^3=1$一组解.
\\

Techniques:

$ x^4+y^4=z^2 ,x^4+y^2=z^4, x^{4m}+y^{4m}=z^{4m} $无非零解

解构造:$ x^n+y^n=z^{n+1}$:$ x=1+k^n,y=kx,z=x$

$ x^n+y^n=z^{n-1}$:$ x=(1+k^n)^{n-2},y=kx,z=(1+k^n)x$ 

$ x!y!=z!$:$ z=x!,y=z-1$
\\

$ x^n+1=y^{n+1},(x,n+1)=1$无解:

证:$ x^n=(y-1)(y^n+\cdots +1)$.假设$ d=(y-1,y^n+\cdots +1)>1$

则$ \exists p \mid d ,s.t.y \equiv 1 \pmod p ,x\equiv 0 \pmod p$

$ \Rightarrow \therefore y^n +\cdots + 1 \equiv n+1 \pmod p\Rightarrow p \mid n+1 $矛盾

$ \therefore d=1\Rightarrow y-1=a^n,y^n+\cdots+1 =b^n \in (y^n,(y+1)^n)$矛盾
\\

$ 3x^2-4xy+3y^2=35\Rightarrow (3x-2y)^2+5y^2=105\Rightarrow y^2\le 21$为避免负项配方估界
\\

(CMO)解$ a^m+1 \mid a^n+203, n<m$时估界,$ n=m$时易.

$ n>m$时,$ \Rightarrow a^m+1 \mid a^{n-m}-203$.若$ a^s<=203$估界.

$ a^s>203$时$ \Rightarrow a^m+1\mid a^{s-m}+203=2^{n-2m}+203$类似前结构派生解
\\

%% $File: polynomial.tex
% $Date: Tue Mar 06 20:55:25 2012 +0800
% Author: WuYuxin <ppwwyyxxc@gmail.com>
\section{多项式}
$ f(x)$次数$ \le n$,且$ f(k)(k=0\cdots n)$均为整数,

则$ f(x)$为整值多项式,且整值多项式必可表为$ \sum_{i=0}^{n}{a_iC_x^i}$

证:设$ f(x)=\sum_{i=0}^{n}{a_iC_x^i},a_i \in \mathbb{C}$,取$ x=0,1,\cdots$

$ \mathbb{Z}\ni f(0)=a_0$.又$ \mathbb{Z}\ni f(1)=a_0+a_1\Rightarrow a_1\in \mathbb{Z}\cdots a_i\in \mathbb{Z}$

推论:$ f(x)$次数$ \le n$,且对连续$ n+1$个整自变量取整值,则其为整值多项式(平移即可)
\\

整系数多项式$ P(x):u-v \mid P(u)-P(v)\Rightarrow P(1)\equiv P(k+1) \equiv \cdots \equiv P(nk+1) \pmod k$
\\

设有$ a_1,\cdots a_m$满足对$ \forall n,\exists i,a_i\mid F(n),$则$ \exists i,\forall n,a_i\mid F(n)$

反证:设$ \exists x_1, a_1 \nmid F(x_1),\cdots \exists x_m, a_m \nmid F(x_m)
\Leftrightarrow \exists d_i=p_i^{r_i},d_i \mid a_i$且$ d_i\nmid F(x_i)$

$ d_1,\cdots d_m$中同底数只保留低次幂,得$ d_1\cdots d_s.$则$ \exists N ,\forall i,N\equiv x_i \pmod{d_i}$

$ \therefore \forall i,F(N)\equiv F(x_i)\not\equiv 0 \pmod{d_i}\Rightarrow \forall i,F(N)\not\equiv 0 \pmod{a_i}$
\\

整系数多项式$ P(x)=a_nx^n+\cdots a_1x\pm1$值域的素因子无穷:假设有限-$ p_1\cdots p_k$

则$ P(i\prod{p_t})$不含素因子$ \Rightarrow P(i\prod{p_t})=\pm 1$但$ n$次多项式至多给出$ 2n$个$ \pm 1$
\\


% $File: n=ax+by.tex
% $Date: Thu Mar 08 20:51:30 2012 +0800
% Author: WuYuxin <ppwwyyxxc@gmail.com>
\section{表n为ax+by}
$(a,b)=1\Rightarrow \exists x,y\in \mathbb{N}^+,ax-by=1$
\\

$ \forall n>ab-a-b$可表为$ ax+by,x,y\in \mathbb{N}$.

{\bf 证 }:设$ n=a(x_0+bt)+b(y_0-at),$可取$ t$使得$ 0 \le y =y_0-at\le a-1$

则$ ax=n-(y_0-at)b>ab-a-b-(a-1)b=-a\Rightarrow x>-1\Rightarrow x\in \mathbb{N}$
\\

$ n=ab-a-b$不可表.反设结论不成立,则
$ ab=(x+1)a+(y+1)b\Rightarrow b\mid x+1\Rightarrow x+1\ge b$矛盾
\\

写$ n$为$ ax+by,0\le x\le b-1$,若$ n=ax+by$中$ y\ge 0$

则 $ n'=(b-1-x)a+(-1-y)b$中仍有$ 0 \le b-1-x \le b-1$,但$ -1-y <0$.

于是$ n$可表$ \Rightarrow ab-a-b-n$不可表.$ \therefore [0,ab-a-b]$
中有$ \dfrac{(a-1)(b-1)}{2}$个不可表.
\\

在矩形$\begin{matrix} 0\le x\le b\\ 0\le y\le a \end{matrix} $中有$ (a+1)(b+1)$个整点.

其中使$ 0\le ax+by<ab$的整点有$ \dfrac{(a+1)(b+1)}{2} -1$个
\\

$ n=ax+by,x,y\in \mathbb{N^+}$有至少两种表法
$ \Leftrightarrow n$可表为$ ab+a+b+ax+by,x,y\in \mathbb{N}$

i) $ ab+a+b+ax+by=a(1+b+x)+b(1+y)=b(1+a+y)+a(1+x)$

ii) $ ax_1+by_1=ax_2+by_2\Rightarrow a\mid y_2-y_1\Rightarrow y_2\ge a+1$

$ \therefore ax_2+by_2=ab+a+b+(y_2-a-1)b+(x_2-1)a$

\printbibliography

\end{document}


% 冯跃峰 初等数论;潘 初等数论; 小丛书; 奥塞经典; 刘培杰 背景研究;
