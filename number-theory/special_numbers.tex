% $File: special-numbers.tex
% $Date: Wed Mar 07 10:42:12 2012 +0800
% Author:  ppwwyyxxc@gmail.com

\section{Special Numbers}

$ 2^k-1$为素数$ \Rightarrow k$为素数 

Mersenne's Prime$ \Leftrightarrow $Perfect Number:($ \sigma(n)=2n \Leftrightarrow n=\dfrac{1}{2}M_{(p)}(M_{(p)}+1)$ )

i)若$ n=2^{p-1}M_{(p)},$则$ \sigma(n)=(1+2+\cdots+2^{p-1})(1+M_{(p)})=2n$

ii)若$ n$为偶完全数,易知$ n\ne 2^k,$

于是设$ n=2^{m-1}u\Rightarrow 2^mu=\sigma(n)=\sigma(2^{m-1})\sigma(u)=(2^m-1)\sigma(u)$

从而$ \sigma(u)=u+\dfrac{u}{2^m-1}\Rightarrow u=2^m-1,$且$ 2^m-1$为素数
\\

$ n^k+1$为素数$ \Rightarrow k$为2的幂

Fermat's Number

$ n\ge 5$时$ 2^n\equiv 2^{n-4} \pmod 10\Rightarrow F_n(n\ge 2)\equiv 7 \pmod 10$

$F_n=2^{2^n}+1,F_0F_1\cdots F_{n-1}+2=F_n\Rightarrow (F_n,F_m)=1 \Rightarrow $素数无穷多

在任意形如$ a^x-1$中设$ x=2^kq$,则可分解$ a^x-1=(a^q)^{2^k}-1=\cdots$

设$ F_n$的任一素因子$ p,2^{2^n}\equiv -1 \pmod p \therefore \delta_p(2)\mid 2^{n+1}\Rightarrow \delta_p(2)=2^k$

又$ 2^{2^k}\equiv 1 \pmod p,2^{2^n}\equiv -1 \pmod p\Rightarrow k>n\Rightarrow k=n+1$

有结论:$ \delta_p(2)=2^{n+1},2^{n+1}\mid p-1$

一般地,$ a^{2^k}\equiv -1 \pmod m\Rightarrow \delta_m(a)=2^{k+1}$
\\

伪素数递归构造

$ n \mid 2^n-2\Rightarrow 2^{2^n-1}-2=2^{nk+1}-2=2(2^{nk}-1)\equiv0\pmod{2^n-1}$
\\

孪生素数$ p,q=p+2.p+q\mid p^p+q^q$ 

{\bf 证 }:$RHS =p^p+(p+2)^p+(p+2)^{p+2}-(p+2)^p=A(p+q)+q^p(p+1)(p+3)$
\\

Sylvester's Sequence
$    a_1=2,a_n=a_{n-1}^2-a_{n-1}+1\Rightarrow \sum_{i=1}^{n}{\dfrac{1}{a_i}}+\dfrac{1}{\prod_{i=1}^{n}{a_i}}=1 $ 

$  a_{n+1}=\prod_{i=1}^{n}{a_i}+1\Rightarrow (a_n,a_m)=1,a_n\ge 2^{n-1}  $ 

最佳单位分数逼近:对$ \forall \{x_n\},\sum_{i=1}^{n}{\dfrac{1}{x_i}}<1\Rightarrow \sum_{i=1}^{n}{\dfrac{1}{x_i}}\le \sum_{i=1}^{n}{\dfrac{1}{a_i}}$

{\bf 证 }:设有$ \sum_{i=1}^{j}{\dfrac{1}{x_i}}\le \sum_{i=1}^{j}{\dfrac{1}{a_i}},j=1,2,\cdots n,\sum_{i=1}^{n+1}{\dfrac{1}{x_i}}> \sum_{i=1}^{n+1}{a_i}$

作Abel变换:\[  n+1=\sum_{i=1}^{n+1}{\dfrac{x_i}{x_i}}=x_{n+1}\sum_{i=1}^{n+1}{\dfrac{1}{x_i}}+\sum_{j=1}^{n}{(\sum_{i=1}^{j}{\dfrac{1}{x_i}})(x_j-x_{j+1})} \]

\[ > x_{n+1}\sum_{i=1}^{n+1}{\dfrac{1}{a_i}}+\sum_{j=1}^{n}{(\sum_{i=1}^{j}{\dfrac{1}{a_i}})(x_j-x_{j+1})}=\sum_{i=1}^{n+1}{\dfrac{x_i}{a_i}}\]

\[ \ge (n+1) \sqrt[n+1]{\dfrac{\prod{x_i}}{\prod{a_i}}}\Rightarrow \prod_{i=1}^{n+1}{x_i}<\prod_{i=1}^{n+1}{a_i}\Rightarrow \sum_{i=1}^{n+1}{\dfrac{1}{x_i}}<\sum_{i=1}^{n+1}{\dfrac{1}{a_i}}\] 
\\

Sophie Germain素数$ p$($ 2p+1$也为素数).

若$ p\equiv 3 \pmod 4$,则$ 2p+1 \mid 2^p-1=M_{(p)}$

{\bf 证 }:设$ k=2p+1=8t-1,2^{\frac{k-1}{2}}\equiv 1 \pmod k\Leftrightarrow (\dfrac{2}{k})=1=(-1)^{\frac{k^2-1}{8}}$
