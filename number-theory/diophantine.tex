% $File: diophantine.tex
% $Date: Tue Sep 18 21:02:55 2012 +0800
% Author: ppwwyyxxc@gmail.com
\section{Diophantine Equation} 
Pythagoras: $ a^2+b^2=c^2,(a,b,c)=1,2\mid b$的所有$ N^{+}$上的解为:

$ a=u^2-v^2,b=2uv,c=u^2+v^2,(u,v)=1,u\ge v,2 \nmid u+v$
\\

$ a^2-mab+b^2=k,k\le m$且非平方数,方程无解.
{\bf 证 }:假设有最小解$ (a_0,b_0),a_0\ge b_0,$且$ a_0+b_0$,令$ a'=mb_0-a_0$,则$ a'\le 0$或$ a' \ge a_0$

若$ a'=0$则$ k$平方数;若$ a'<0\Rightarrow a_o\ge mb_0+1\Rightarrow a^2-ma_0b_0+b_0^2>m\ge k$

若$ a'\ge a_0\Rightarrow b_0^2-k=a_0a'\ge a_0^2\ge b_0^2\ge b_0^2-k$.每种情况均矛盾.
\\

Pell:
标准Pell方程$ x^2-dy^2=1,d\in \mathbb{N^{+}},d$非平方数必有无穷多解,$ (x_0,y_0)$称为基本解,
所有解为$ x_n+\sqrt{d}y_n=(x_0+\sqrt{d}y_0)^n$

$ x^2-dy^2=C$若有解则必有无穷多解.设最小解$ (x_1,y_1)$,则$ x_n+\sqrt{d}y_n=(x_1+\sqrt{d}y_1)(x_0+\sqrt{d}y_0)^{n-1}$为部分解

对上式中改变符号:$
\begin{cases}
x_n+\sqrt{d}y_n=(x_0+\sqrt{d}y_0)^n \\
x_n-\sqrt{d}y_n=(x_0-\sqrt{d}y_0)^n   
\end{cases}  \Rightarrow	$
两式加减即可求出通项

特征根为$ \lambda_{1,2}=x_0\pm\sqrt{d}y_0\Rightarrow $特征方程
$ \lambda^2-2x_0\lambda+1=0\Rightarrow$ 

递推关系$ \begin{cases} x_{n+1}=2x_0x_n-x_{n-1}\\ y_{n+1}=2x_0y_n-y_{n-1}\end{cases}$
\\

$ x^2-dy^2=-1,$设$ \sqrt{d}$的连分数周期为$ l$,则$ l$为偶$ \Leftrightarrow $Pell方程无解

特别地,素数$ p=4k+1$时,$ x^2-py^2=-1$	有解

$ x^2-dy^2=-1$若有解则必有无穷多解.设最小解$ (x_1,y_1)$,则所有解为$ x_n+\sqrt{d}y_n=(x_1+\sqrt{d}y_1)^{2n-1}$
\\

Lemma:

$ x^2-dy^2=4$的整数解$ x=u,y=v$是正整数解$ \Leftrightarrow \dfrac{u+\sqrt{d}v}{2}>1$

{\bf 证 }:$ \dfrac{u+\sqrt{d}v}{2}>1\Rightarrow \dfrac{u-\sqrt{d}v}{2}\in(0,1)$,
两式相加得$ u>1$即$ u\ge 2$

又$ 1>\dfrac{u-\sqrt{d}v}{2}\ge 1-\dfrac{\sqrt{d}v}{2}\Rightarrow v>0$.\qed

$ x^2-dy^2=4$若有最小解$ (x_1,y_1)$,则所有解$ \dfrac{x_n+\sqrt{d}y_n}{2}=(\dfrac{x_1+\sqrt{d}y_1}{2})^n$

{\bf 证 }:设数列$x,y, \dfrac{x_n+\sqrt{d}y_n}{2}=(\dfrac{x_1+\sqrt{d}y_1}{2})^n$,
假设有解$ (a,b)$不在其中

不妨设$ ( \dfrac{x_1+\sqrt{d}y_1}{2})^{n+1}>\dfrac{a+\sqrt{d}b}{2}>(\dfrac{x_1+\sqrt{d}y_1}{2})^n$,则

$   \dfrac{x_1+\sqrt{d}y_1}{2}=\dfrac{x_n^2-dy_n^2}{4}\dfrac{x_1+\sqrt{d}y_1}{2}=(\dfrac{x_1+\sqrt{d}y_1}{2})^{n+1}\dfrac{x_n-\sqrt{d}y_n}{2}>\dfrac{a+\sqrt{d}b}{2}\dfrac{x_n-\sqrt{d}y_n}{2} $ 

$  \overset{\text{def}}{=} \dfrac{s+\sqrt{d}t}{2}>(\dfrac{x_1+\sqrt{d}y_1}{2})^n\dfrac{x_n-\sqrt{d}y_n}{2}=1$ 
\\

解方程$ x^2-5y^2=-4$,有恒等式$ (\dfrac{3x-5y}{2})^2-5(\dfrac{3y-x}{2})^2=x^2-5y^2$

$ \therefore (x,y)\rightarrow (\dfrac{3x-5y}{2},\dfrac{3y-x}{2}),$

可证当$ y>1$时总有$ 0<\dfrac{3x-5y}{2}<x,0<\dfrac{3y-x}{2}<y$,完成递降

递推可求得其所有解为$ \dfrac{x_n+\sqrt{5}y_n}{2}=(\dfrac{x_1+\sqrt{5}y_1}{2})^{2n+1}$

事实上也有恒等式$ (ax-Dby)^2-D(ay-bx)^2=(a^2-Db^2)(x^2-Dy^2)=x^2-Dy^2$,但用它递降不总能做到边界
\\

(Catalan)$ a^x-b^y=1$只有$ 3^2-2^3=1$一组解.
\\

Techniques:

$ x^4+y^4=z^2 ,x^4+y^2=z^4, x^{4m}+y^{4m}=z^{4m} $无非零解

解构造:$ x^n+y^n=z^{n+1}$:$ x=1+k^n,y=kx,z=x$

$ x^n+y^n=z^{n-1}$:$ x=(1+k^n)^{n-2},y=kx,z=(1+k^n)x$ 

$ x!y!=z!$:$ z=x!,y=z-1$
\\

$ x^n+1=y^{n+1},(x,n+1)=1$无解:

{\bf 证 }:$ x^n=(y-1)(y^n+\cdots +1)$.假设$ d=(y-1,y^n+\cdots +1)>1$

则$ \exists p \mid d ,s.t.y \equiv 1 \pmod p ,x\equiv 0 \pmod p$

$ \Rightarrow \therefore y^n +\cdots + 1 \equiv n+1 \pmod p\Rightarrow p \mid n+1 $矛盾

$ \therefore d=1\Rightarrow y-1=a^n,y^n+\cdots+1 =b^n \in (y^n,(y+1)^n)$矛盾
\\

$ 3x^2-4xy+3y^2=35\Rightarrow (3x-2y)^2+5y^2=105\Rightarrow y^2\le 21$为避免负项配方估界
\\

(CMO)解$ a^m+1 \mid a^n+203, n<m$时估界,$ n=m$时易.

$ n>m$时,$ \Rightarrow a^m+1 \mid a^{n-m}-203$.若$ a^s<=203$估界.

$ a^s>203$时$ \Rightarrow a^m+1\mid a^{s-m}+203=2^{n-2m}+203$类似前结构派生解
\\
