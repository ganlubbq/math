% File: gamma.tex
% Date: Wed Dec 19 10:13:18 2012 +0800
% Author: Yuxin Wu <ppwwyyxxc@gmail.com>

\section{Connections to Gamma Function}
Continue working on \eqnref{f-pr}, we can obtain:
 \begin{equation}
 \Pr{X=k} = \dfrac{{n\choose{k}}}{n!}(n-k)!\sum_{i=0}^{n-k}{\dfrac{(-1)^i}{i!}} = \dfrac{1}{k!}\sum_{i=0}^{n-k}\dfrac{(-1)^i}{i!}
 \label{eqn:def-p}
 \end{equation}

Introducing the \emph{incomplete gamma function} defined as:
\[ \Gamma(s,x) = \int_{x}^{\infty}t^{s-1}e^{-t}\mathrm{d}t\]
By \emph{integration by parts}, a recurrence relation can be found:
\[ \Gamma(s,x) = (s-1)\Gamma(s-1,x) + x^{s-1}e^{-x}\]
Thus,
\begin{equation}
\Gamma(n,x) = (n-1)! e^{-x}\sum_{i=0}^{n-1}\dfrac{x^i}{i!}, \forall n\in \mathbb{N}
\label{eqn:rec-gamma}
\end{equation}
Then we can rewrite the previous formula in a more elegant way:
\[ D_n = \dfrac{\Gamma(n+1, -1)}{e}\]
\[ \Pr{X=k}=\dfrac{{n\choose{k}}\Gamma(n-k+1,-1)}{en!} =\dfrac{\Gamma(n-k+1,-1)}{ek!(n-k)!}\]
\\

Another way of rewriting $ D_n$ is:
\[ D_n = \E{(Y-1)^n}\]
where $ Y$ is a random variable such that $ Y\sim Exp(1),$
here  $Exp(\lambda) $ denotes \emph{exponential distribution}.

To prove this, it is sufficient to show that
\[ \E{Y^k} = \int_{\mathbb{R}^+}x^ke^{-x}\mathrm{d}x = \Gamma(k+1) = k!\]
where $ \Gamma(x)$ is the \emph{complete gamma function}.

Thus,
\[ \E{(Y-1)^n} = \sum_{k=0}^n{n\choose{k}}\E{Y^{n-k}}(-1)^k = \sum_{k=0}^n{n\choose k}(-1)^k(n-k)! = D_n \qed\]
