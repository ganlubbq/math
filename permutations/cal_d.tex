% File: cal_d.tex
% Date: Tue Dec 18 19:16:40 2012 +0800
% Author: Yuxin Wu <ppwwyyxxc@gmail.com>

\section{Calculation of $ D_{n,k}$}
\label{sec:f-calc}
We are about to see three totally different approaches of the calculation of $ D_{n}$.

Then we can easily caculate $ D_{n,k}$ by
\[ D_{n,k} = {n\choose{k}}D_{n-k}\].

\subsection{Using the Inclusion-Exclusion Principle}

      Define $ A_j$ as:
      \[ A_j =  \{ \pi \in \Pi | j \in \mathcal{F}(\pi)\}, j = 1,2,\cdots ,n\]

      Then it is obvious to see that, for any $ t = 1,2,\cdots ,n$, we have
      \[ \card{\bigcap_{i=1}^{t}A_{r_i}} = (n-t)!\]
      where $ r_i$ is any permutation of $ (1,2,\cdots ,n).$

      Therefore, $ D_n$ can be calculated by the Inclusion-Exclusion Principle:
       \begin{align*}
       D_n = & n! - \sum_{i=1}^n\card{A_i} + \sum_{1\le i< j\le n}{\card{A_i\cap A_j}}-\cdots +
      (-1)^n\card{\bigcap_{i=1}^n{A_i}}\\
      =& n! - {n\choose{1}}(n-1)! + {n\choose{2}}(n-2)!-\cdots +(-1)^n{n\choose{n}}(n-n)!\\
      =& n!\sum_{i=0}^n{\dfrac{(-1)^i}{i!}}
      \end{align*}

      \subsection{Using Recurrence Relation}

     For any permutation $\pi$ of $ (1,2,\cdots n),$ such that $ \card{\mathcal{F}(\pi)} = 0$,
     it is obvious that $ \pi(1)$ has $ n-1$ possible values.
     Now consider the value of $ \pi(\pi(1)):$

     If $ \pi(\pi(1)) = 1, $ then $\pi'= (\pi(2),\cdots ,\pi(\pi(1)-1), \pi(\pi(1)+1), \cdots ,\pi(n))$
     is a permutation of
     $ (2,\cdots ,\pi(1)-1, \pi(1)+1, \cdots ,n)$, such that $\card{\mathcal{F}(\pi')} = 0$.
     It is easy to show that this correspondence from the given $ \pi$ to $ \pi'$ is bijective.

     If $ \pi(\pi(1)) \neq 1$, assume $ \pi(j) = 1$, then $ \pi' =(\pi(2), \cdots ,\pi(j-1), \pi(1), \pi(j+1), \cdots ,\pi(n))$
     is a permutation of $ (2,\cdots ,n),$ such that $\card{\mathcal{F}(\pi')}=0$.
     It is easy to show that this correspondence from the given $\pi$ to $ \pi'$ is also bijective.

     Therefore, we have the recurrence relation
\begin{align}
& D_n = (n-1)(D_{n-1}+D_{n-2}) \label{eqn:recurrence}\\
\Leftrightarrow & D_n - nD_{n-1} = - [D_{n-1}- (n-1)D_{n-2}] \label{eqn:recurrence-change}\\
&D_2 = 1, D_1 = 0 \nonumber
\end{align}

Continue applying \eqnref{recurrence-change}, we obtain
\begin{align*}
 &D_n - nD_{n-1} = -[D_{n-1}-(n-1)D_{n-2}]= \cdots  =(-1)^{n-2}(D_2 - 2D_1) = (-1)^n \\
 \Rightarrow & \dfrac{D_n}{n!} - \dfrac{D_{n-1}}{(n-1)!} = \dfrac{(-1)^n}{n!} \\
 \Rightarrow &D_n = n! \sum_{i=0}^n{\dfrac{(-1)^i}{i!}}
\end{align*}

\subsection{Using the Inversion Formula}
\begin{lemma}
  \textbf{(Inversion Formula)}
  \label{lemma:inversion}
  Given two sequences $ \{ a_n\},\{ b_n\} $. If the formula
  \[  b_n = \sum_{k=0}^n(-1)^k{n\choose{k}}a_k\]
  holds for all $ n = 1,2,\cdots $, then
  \[ a_n = \sum_{k=0}^n(-1)^k{n\choose{k}}b_k, n = 1,2,\cdots \]
\end{lemma}
\begin{proof}
Note that in series $ \sum_{k=0}^n(-1)^k{n\choose{k}}(1+x)^k,$
the coefficient of the term $ x^p$ is given by $
\sum_{k=p}^n(-1)^k{n\choose{k}}{k\choose{p}}.$

On the other hand, by \emph{Binomial Theorem},
\[ \sum_{k=0}^n(-1)^k{n\choose{k}}(1+x)^k = [1-(1+x)]^n = (-1)^nx^n, \]
therefore the coefficient of the term $ x^p$ is $(-1)^n\delta_{pn} $, where $ \delta_{ij}$ is the
\emph{Kronecker delta function}:
\[ \delta_{ij} = \begin{cases}
  1, i = j \\ 0, i \ne j \end{cases} \]
  Thus,
  \[ \sum_{k=p}^n(-1)^k{n\choose{k}}{k\choose{p}} = (-1)^n\delta_{pn}\]

  It follows that
\begin{align*}
    \sum_{k=0}^n(-1)^k{n\choose{k}}b_k &=
    \sum_{k=0}^n(-1)^k{n\choose{k}}\left[\sum_{p=0}^k(-1)^i{k\choose{i}}a_i\right] \\
    &=\sum_{p=0}^n(-1)^p\left[\sum_{k=p}^n(-1)^k{n\choose{k}}{k\choose{p}}a_p\right] \\
   &=\sum_{p=0}^n(-1)^p(-1)^n\delta_{pn}a_p \\
   &= a_n
\end{align*} \end{proof}

By definition, $ \sum_{k=0}^nD_{n,k} =\card{\Pi_n}= n!$,
which is equivalent to
 \begin{equation}
 n! = \sum_{k=0}^n{n\choose{k}}D_k =
  \sum_{k=0}^n\left[ (-1)^k{n\choose{k}}\right]\left[(-1)^kD_k\right]
  \label{eqn:fact}
  \end{equation}

According to \lemref{inversion}, we immediately obtain:
\begin{align*}
&(-1)^nD_n = \sum_{k=0}^n(-1)^k{n\choose{k}}k! \\
\Rightarrow & D_n = \sum_{k=0}^n(-1)^{n-k}\dfrac{n!}{(n-k)!} = n!\sum_{k=0}^n\dfrac{(-1)^k}{k!}
\end{align*}

