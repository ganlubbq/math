% $File: solve.tex
% $Date: Fri Mar 30 16:24:57 2012 +0800
% Author: WuYuxin <ppwwyyxxc@gmail.com>
\section{solve}

\subsection{elimination}
Def: the $ l$th elimination ideal $ I_l=I\cap k[x_{l+1},\cdots,x_n]$

(Elimination Theorem):$ G_l=G\cap k[x_{l+1},\cdots,x_n]$是$ I_l$的Grobner基
\\

Def: 将$ f \in I_{l-1}$写成$ f=c_q(x_{l+1},\cdots,x_n)x_l^q+\cdots+c_0(x_{l+1,\cdots,x_n})$.
其中$ x_l^q$为$ f$中$ x_l$的最高次数.称$ c_q$为$ f$的leading coefficient polynomial

(Extension Theorem):$ k$是代数闭域,$(a_{l+1}\cdots a_n)\in V(I_l)$
若$ I_{l-1}$的字典序Grobner基中存在的一个元素的leading coefficient polynomials
在$ (a_{l+1},\cdots,a_n)$处不为0,则此解可扩展,即$ \exists a_l,s.t.(a_l,\cdots,a_n)\in V(I_{l-1})$

对零维理想,可求其字典序Grobner基后找到一元多项式进行消元.
但此法若求数值解会导致之后系数误差,系数的微小误差对多项式根的数值求解影响很大,
根的个数,是否为实数都无法判断.


\subsection{finite-dimensional algebra}
对余数的算术操作,有:

$\bar{f}^G+\bar{g}^G= \overline{f+g}^G, \overline{fg}^G=\overline{\bar{f}^G\bar{g}^G}^G$
(乘积次数可能变大,要再取余)
\\

由商环$ k[x_1,\ldots,x_n]/I$中定义陪集(coset):$ [f]=f+I=\{f+h:h\in I\}$.

于是有$ \bar{f}^G=\bar{g}^G\Leftrightarrow f-g\in I\Leftrightarrow [f]=[g]$

对陪集定义对应的算术操作,则此商环有线性空间结构,称为一个代数$ A$.

定义这个代数的标准基:$ B=\{x^{\alpha}:x^{\alpha}\not \in \langle \LT(I) \rangle\}$
\\

如$ G=\{ x^2+\dfrac{3xy}{2}+\dfrac{y^2}{2}-\dfrac{3x}{2}-\dfrac{3y}{2},xy^2-x,y^3-y\}$
是一组grevlex的Grobner基. $ \langle \LT(I) \rangle=\langle x^2,xy^2,y^3 \rangle$ 

$\Rightarrow B=\{ 1,x,y,xy,y^2\}$为Remainder中可能的单项
\\

在$ A$中根据乘法操作定义映射$ m_f:m_f([g])=[f][g]=[fg]$.

可证$ m_f$为线性映射,$ m_f=m_g\Leftrightarrow f-g\in I$

考虑$ m_f$的矩阵表示,有$ m_f[i,j]=\overline{f B[j]}^G$中$ B[i]$项的系数
\\

(Finiteness Theorem)$ A=k[x_1,\ldots,x_n]/I$为有限维$ \Leftrightarrow V(I)$为有限集
$ \Leftrightarrow \forall i,\exists m_i\ge0,g\in G:x_i^{m_i}=\LT(g)$.
并称这样的理想为零维理想
\\

零维理想$ I,\forall i,I\cap k[x_i]\neq \emptyset$

{\bf 证:}重定义字典序使$ x_i$最小.在$ Grobner$基中,$ \exists g,\LT(g)=x_i^{m_i},$则$ g$只含$ x_i$

或:因为$ A$有限维,$\Rightarrow  S=\{1,[x_i],[x_i]^2,\cdots\}$线性相关.

设$ m_i=\min\{t:\{ 1,[x_i]\cdots,[x_i]^{t} \}\texttt{线性相关}\}$

$ \Rightarrow \exists c,\sum_{j=0}^{m_i}{c_j[x_i]^j=[0]}$ 即$\exists p_i(x_i)\in I$
\\

由(Fitness Theorem),有求$ B$的方法:

设$ S=\{ x_1^{a_1}\ldots x_n^{a_n}:1\le a_i\le m_i-1\},B=\{ m\in S:\overline{m}^G=m\}$
\\

(Theorem)设$ A$为由零维理想$ I$定义的商环上的代数,$ h_f$ 为$m_f$的最小多项式,则:

$ \lambda $是$ h_f(x)=0$的根$\Leftrightarrow \lambda$是$ m_f$的特征值$ \Leftrightarrow \lambda\in \{ f(x):x\in V(I)\}$

由此,分别计算$ m_{x_1},\cdots,m_{x_n}$即可解方程.

\subsection{radical}
求零维理想的根理想:

Reduce: $ p_{red}=\dfrac{p}{(p,p')}$与$ p$有相同的根但无重根(sqr free)

显然有$ \sqrt{\langle p \rangle}=\langle p_{red} \rangle,$如果$ p$ 为非零一元多项式.
\\

$ I\subset k[x_1\ldots x_n],p=\prod_{j=1}^{d}{( x_1-a_d )},a_j$两两不同.$ p_j=\dfrac{p}{x_1-a_j}$,

则$ I+\langle p \rangle = \bigcap\limits_{j}{(I+\langle x_1-a_j \rangle)}$

{\bf 证}.i)$ LHS \subset RHS$.因为属于右边交的每一个

ii) $ p_j(I+\langle x_1-a_j \rangle)\subset I+\langle p \rangle$

iii)设$ h \in RHS$,因为$ p_j$全体互素,有$ h=\sum_{j}{h_jp_jh},$由上ii),
知和式中的每一项$ \subset I+\langle p \rangle=LHS$.于是$ RHS=LHS$
\\

$ I$ 为零维理想,$ p_i \in I \cap \mathbb{C}[x_i]$,则$ \sqrt{I}=I+\langle p_{1,red},\ldots,p_{n,red} \rangle$

证:设$ RHS = J = J+\langle p_{1,red} \rangle = \bigcap\limits_{j}{(J + \langle x_1-a_{1j} \rangle)}$

$ \Rightarrow J =\bigcap\limits_{j_1,\cdots ,j_n}{(J+\langle x_1-a_{1j_1},\ldots,x_n-a_{nj_n} \rangle)}$

$ (\langle x_1-a_{1j_1},\ldots,x_n-a_{nj_n} \rangle)$为极大理想,
所以$(J+\langle x_1-a_{1j_1},\ldots,x_n-a_{nj_n} \rangle) $等于$ \mathbb{C}[x_1\ldots,x_n]$或$ \langle x_1-a_{1j_1},\ldots,x_n-a_{nj_n} \rangle$
$ \Rightarrow J$为极大理想的交,仍为极大理想$ \Rightarrow J$为根理想

由于$ p_i$的无平方部分vanish at$ V(I)$, 有$ J \subset I(V(I)) = \sqrt{I}$,又由定义, $ I\subset J\Rightarrow J\subset \sqrt{I}\subset \sqrt{J}$

于是由$ J= \sqrt{J}$可知$ J = \sqrt{I}$.得证.

