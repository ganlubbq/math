% File: gf.tex
% Date: Wed Jul 17 14:44:21 2013 +0800
% Author: Yuxin Wu <ppwwyyxxc@gmail.com>

\section{Generating Function}
\subsection{Ordinary Generating Function}
Let the \emph{ordinary generating function} of $ D_n$ be
\[ y(x) = D_0 + D_1x+D_2x^2+\cdots \]
here, we let $ D_0=1$ to be consistent with \eqnref{recurrence}.
The following formulae are obvious:
\begin{align*}
 \dfrac{\mathrm{d}y}{\mathrm{d}x} &= D_1+2D_2x+3D_3x^2+\cdots  \\
 \dfrac{\mathrm{d}(xy)}{\mathrm{d}x} &=D_0+2D_1x+3D_2x^2+\cdots \\
 \dfrac{\mathrm{d}(y+xy)}{\mathrm{d}x} &= (D_0 +D_1) + 2(D_1+D_2)x+3(D_2+D_3)x^2+\cdots \\
 &=\sum_{k=0}^{\infty}(k+1)(D_{k}+D_{k+1})x^k \\
 &=\sum_{k=2}^{\infty}D_kx^{k-2} \quad (\text{applying}\; \eqnref{recurrence}) \\
 &=\dfrac{y-D_1x-D_0}{x^2}\\
 &=\dfrac{y-1}{x^2}
\end{align*}

We obtain an ODE:
 \begin{align*}
& \dfrac{\mathrm{d}y}{\mathrm{d}x}+x\dfrac{\mathrm{d}y}{\mathrm{d}x} +y=\dfrac{y-1}{x^2}\\
\Leftrightarrow &\dfrac{\mathrm{d}y}{\mathrm{d}x}=\dfrac{1-x}{x^2}y -\dfrac{1}{x^2+x^3}
 \end{align*}
By the \emph{method of variation of parameters},
we can get its general solution of the following form:
\begin{align*}
y&=e^{\Oldint\frac{1-x}{x^2}\mathrm{d}x}\left[\int\dfrac{-1}{x^2+x^3}e^{-\Oldint\frac{1-x}{x^2}\mathrm{d}x}\mathrm{d}x+C\right] \\
&=\dfrac{1}{xe^{\frac{1}{x}}}\left[-\int\dfrac{e^{\frac{1}{x}}}{x(x+1)}\mathrm{d}x+C\right] \\
& =\dfrac{1}{xe^{\frac{1}{x}}}\left[\dfrac{1}{e}\int\dfrac{e^{1+\frac{1}{x}}}{1+\frac{1}{x}}\mathrm{d}(1+\dfrac{1}{x})+C\right]\\
&=\dfrac{1}{xe^{\frac{1}{x}+1}}\left[\int_{-\infty}^{1+\frac{1}{x}}\dfrac{e^t}{t}\mathrm{d}t+C\right]\\
&=\dfrac{-1}{xe^{\frac{1}{x}+1}}\left[\int_{-(1+\frac{1}{x})}^{\infty}\dfrac{e^{-t}}{t}\mathrm{d}t+C\right]\\
&=-\dfrac{1}{xe^{\frac{1}{x}+1}}\left[\Gamma(0,-\dfrac{x+1}{x})+C\right]
\end{align*}
$\Gamma(0,-\dfrac{x+1}{x})$ have the series form $ -e^{1+\frac{1}{x}}(x+x^3+2 x^4\cdots )$
(this can be verified by software), which indicates that
\[ \lim\limits_{x\to0}-\dfrac{1}{xe^{1+\frac{1}{x}}}\Gamma(0, -\dfrac{x+1}{x}) = 1 = D_0=y(0)\]
Therefore $ C=0$, and $ y(x) = -\dfrac{1}{xe^{1+\frac{1}{x}}}\Gamma(0, -\dfrac{x+1}{x})$

\subsection{Exponential Generating Function}
Let the \emph{exponential generating function} of $ D_n$ be
\[ y(x) = D_0 +D_1\dfrac{x}{1!}+D_2\dfrac{x^2}{2!}+\cdots \]
Note that we still take $ D_0=1$, and it follows that
\begin{align*}
  \dfrac{\mathrm{d}y}{\mathrm{d}x} &= \sum_{k=1}^{\infty}kD_k\dfrac{x^{k-1}}{k!} = \sum_{k=0}^{\infty}D_{k+1}\dfrac{x^k}{k!} \\
y+\dfrac{\mathrm{d}y}{\mathrm{d}x} &= \sum_{k=0}^{\infty}(D_k+D_{k+1})\dfrac{x^k}{k!} \\
&=\dfrac{1}{x}\sum_{k=0}^{\infty}D_{k+2}\dfrac{x^{k+1}}{(k+1)!} \quad(\text{applying} \eqnref{recurrence}) \\
&= \dfrac{1}{x}\sum_{k=1}^{\infty}D_{k+1}\dfrac{x^{k}}{k!} \\
&= \dfrac{1}{x}(\dfrac{\mathrm{d}y}{\mathrm{d}x}-D_1)\\
&= \dfrac{\mathrm{d}y}{x\mathrm{d}x}
\end{align*}
We obtain an ODE:
\[ \dfrac{\mathrm{d}y}{\mathrm{d}x}\dfrac{x-1}{x}+y = 0\Leftrightarrow \dfrac{\mathrm{d}y}{y}=\dfrac{x\mathrm{d}x}{1-x}\]
  This can be easily solved:
  \[ \ln y = -x-\ln(1-x) +C\Rightarrow y = C'\dfrac{e^{-x}}{1-x} \]
and $ C'=1$ since $ y(0) = D_0=1$. Then we can conclude
  \begin{equation}
  y(x) = \sum_{k=0}^{\infty}D_k\dfrac{x^k}{k!} = \dfrac{e^{-x}}{1-x}
  \label{eqn:egf}
  \end{equation}

\subsection{Probability Generating Function}
\label{sec:gf}
Let the \emph{probability generating function} of $ X$ be
\[ y(n,x) = \sum_{k=0}^{\infty}\Pr{X=k}x^k\]
Using \eqnref{f-pr}, we have
\[ y(n,x) = \sum_{k=0}^{\infty}\dfrac{D_{n-k}}{(n-k)!}\dfrac{x^k}{k!}\]
Therefore, for a given $x$, the sequence $ \{ y_n\}, y_n = y(n,x)$ is the \emph{convolution} of
$ \{ \dfrac{D_n}{n!}\}$ and $ \{ \dfrac{x^n}{n!} \}$.

By the \emph{convolution formula}, we have
\begin{align*}
  \sum_{n=0}^{\infty} y(n,x)t^n &= \left( \sum_{k=0}^{\infty}\dfrac{D_k}{k!}t^k\right)\left( \sum_{k=0}^{\infty}\dfrac{x^k}{k!}t^k\right) \\
  &=\dfrac{e^{-t}}{1-t}e^{xt} \quad (\text{applying} \; \eqnref{egf})\\
  &= \dfrac{e^{xt-t}}{1-t}
\end{align*}

Denote $ [x^k]g(x)$ as the coefficient of the term $ x^k$ in $ g(x)$, then we have
\[ y(n,x) = [t^n]\dfrac{e^{xt-t}}{1-t}\]

Using the probability generating function, we can calculate the moments of $ X$ again by:
\begin{align*}
  \E{X^{\underline{k}}} &= \left. \left( \dfrac{\mathrm{d}^k y(n,x)}{\mathrm{d}x^k}\right) \right|_{x=1} \\
  & = [t^n]\left. \left( \dfrac{\mathrm{d}^k(\frac{e^{xt-t}}{1-t})}{\mathrm{d}x^k}\right)\right|_{x=1}\\
  &=[t^n]\left. \left( \dfrac{t^ke^{xt-t}}{1-t}\right)\right|_{x=1}\\
  &=[t^n]\dfrac{t^k}{1-t} = [t^n]t^k(1+t+t^2+\cdots )\\
  &=\begin{cases}1 \quad ,1\le k \le n \\ 0 \quad , k > n\end{cases}
\end{align*}
Then, following the arguments in \secref{moments},
we still get \eqnref{moments}.

\subsection{Moment Generating Function \& Characteristic Function}
Let the \emph{moment generating function} of $ X$ be
\[ y(n, t) = \E{e^{tX}} = \sum_{k=0}^{n}e^{tk}\Pr{X=k} \]

It can be calculated as follows:
\begin{align*}
  \E{e^{tX}} &=\sum_{k=0}^n\dfrac{1}{k!}\sum_{i=0}^{n-k}\dfrac{(-1)^i}{i!}e^{tk} \\
  &=\sum_{s=0}^n\sum_{k=0}^s\dfrac{1}{k!}\dfrac{(-1)^{s-k}}{(s-k)!}e^{tk} \quad (\text{let}\; s = k+i)\\
  &=\sum_{s=0}^n\dfrac{1}{s!}\sum_{k=0}^{s}{s\choose k}e^{tk}(-1)^{s-k} \\
  &=\sum_{s=0}^n\dfrac{(e^t-1)^s}{s!}\\
  &=\dfrac{e^{e^t-1}}{n!}\Gamma(n+1, e^t-1) \quad (\text{applying}\; \eqnref{rec-gamma})
\end{align*}

Another way of calculating $y(n,t)$ is by using
$ \E{e^{tX}} = \sum_{k=0}^{\infty}\dfrac{t^k}{k!}\E{X^k}$ and \eqnref{moments}, we can get
\[ \E{e^{tX}} = \sum_{i=0}^n\sum_{k=0}^{\infty}\Stir{k}{i}\dfrac{t^k}{k!} \xlongequal{def}\sum_{i=0}^nf_i(t).\]
Since Stirling numbers of the second kind satisfy the following recurrence relation:
\[ i\Stir{k}{i} = \Stir{k+1}{i}-\Stir{k}{i-1}\]
we have an equation $ if_i(t) = \dfrac{\mathrm{d}f_i(t)}{\mathrm{d}t}-f_{i-1}(t)$. The solution of this equation is
$ f_i(t) = \dfrac{(e^t-1)^i}{i!}, $ which brings with the desired result.
Limited by the length of the paper, the details are left out.
\\

It follows easily that the \emph{characteristic function} of $ X$ is:
\[ \varphi_n(t)=\E{e^{itX}}  =\sum_{k=0}^n\dfrac{(e^{it}-1)^k}{k!} =\dfrac{e^{e^{it}-1}}{n!}\Gamma(n+1, e^{it}-1) \]
Note that the incomplete gamma function can be generalized to be defined on complex numbers.
