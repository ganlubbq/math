% File: characteristics.tex
% Date: Wed Jul 17 14:44:31 2013 +0800
% Author: Yuxin Wu <ppwwyyxxc@gmail.com>

\section{Characteristics of $ X$}
\subsection{Expectation and Variance}
\label{f-random}
By definition, the probability mass function of $ X$ is given by
  \begin{equation}
  \Pr{X=k} = \dfrac{D_{n,k}}{n!}=\dfrac{{n\choose{k}}D_{n-k}}{n!}, k = 0,1,\cdots ,n
  \label{eqn:f-pr}
  \end{equation}
Recall that $ \sum_{k=0}^n{n-1 \choose{n-k}}D_{n-k}=(n-1)!$ according to \eqnref{fact},
the expectation of $ X$ can be calculated as follows:
\begin{equation}
 \E{X} = \sum_{k=0}^n\dfrac{k{n\choose{k}}D_{n-k}}{n!} = \sum_{k=0}^n\dfrac{n{n-1\choose{n-k}}D_{n-k}}{n!} =
  \sum_{k=0}^n\dfrac{{n-1\choose{n-k}}D_{n-k}}{(n-1)!} = 1
  \label{eqn:exp1}
  \end{equation}

By similiar approach, the variance can also be derived:
\begin{align}
 \E{X(X-1)} &= \sum_{k=0}^n\dfrac{k(k-1){n\choose k}D_{n-k}}{n!}=\sum_{k=0}^n\dfrac{n(n-1){n-2 \choose{n-k}}D_{n-k}}{n!} = 1 \label{eqn:exp2} \\
 \Var{X} &= \E{X(X-1)} + \E{X}-\E{X}^2 = 1 \nonumber
\end{align}
\\

Another way of calculating expectation and variance is by treating $ X$ as a sum of $ n $ identical
random variables $ S_i$, where $ S_i$ is defined by:
\[ S_i = \begin{cases}1, \pi(i)=i\\0,\pi(i)\neq i\end{cases}\]

Obviously $ X = \sum_{i=1}^nS_i$ and $ S_i\sim b(1,\dfrac{1}{n})$, where $ b(n,p)$ denotes the
\emph{binomial distribution} with sample size $ n$ and success probability $ p$.

It can be easily shown that for $ i\neq j, \Pr{S_iS_j=1}=\dfrac{(n-2)!}{n!}=\dfrac{1}{n(n-1)} $, therefore $S_iS_j \sim b(1, \dfrac{1}{n(n-1)})$.
Then we can obtain:
\[ \E{X} = \sum_{i=1}^n\E{S_i} = 1\]
\[ \Cov[S_i,S_j]=\E{S_iS_j}-\E{S_i}\E{S_j}=\dfrac{1}{n^2(n-1)}\]
\[ \Var{X} = \sum_{i=1}^n\Var{S_i}+\sum_{i\neq j}\Cov[S_i,S_j] = 1\]

\subsection{Moments}
\label{sec:moments}
Next, we calculate the moments of $ X$.

Introducing the \emph{Stirling number of the second kind}\cite{wiki_stir}  $ \Stir{m}{k} $ , it is well-known that:
\[ \sum_{k=0}^m\Stir{m}{k}x^{\underline{k}} = x^m\]
where $ x^{\underline{k}}$ is the $ k$th \emph{falling factorial} of $ x$ defined by:
\[ x^{\underline{k}} = x(x-1)\cdots (x-k+1). \quad x^{\underline{0}} = 1, \text{in particular}\]

Therefore, we have
\[ \sum_{k=0}^m\Stir{m}{k}\E{X^{\underline{k}}} = \E{X^m}\]

To calculate the moments, we only have to calculate $ \E{X^{\underline{k}}}$.
Continue the derivation in \eqnref{exp1} and \eqnref{exp2}, it is easy to show that
\[ \E{X^{\underline{k}}} =\begin{cases} 1, 0\le k\le n\\
0,  k>n\end{cases} \]
(Another proof of this
will be shown in \secref{gf}).

As a result,
 \begin{equation*}
 \E{X^m} =\begin{cases} \sum_{k=0}^m\Stir{m}{k} = B_m, m = 1,2,\cdots n \\
   \\
  \sum_{k=0}^n\Stir{m}{k}, m  = n+1, n+2,\cdots \end{cases}
  \end{equation*}
where $ B_m$ is the $ m$th \emph{Bell number}\cite{wiki_bell}.

Note that $ \Stir{m}{k} = 0$ for $ m < k$. It is reasonable to rewrite the above formula as follows:
\begin{equation}
  \E{X^m} = \sum_{k=0}^n\Stir{m}{k}, m=1,2,\cdots
  \label{eqn:moments}
\end{equation}

