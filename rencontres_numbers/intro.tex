% File: intro.tex
% Date: Sun Dec 23 15:31:06 2012 +0800
% Author: Yuxin Wu <ppwwyyxxc@gmail.com>

\section{Introduction}
Let $ \pi= (\pi(1), \pi(2),\cdots ,\pi(n))$ be a permutation of $ (1,2,\cdots ,n)$.
The \emph{Fixed Points} of the permutation $ \pi$ ,
denoted as $ \mathcal{F}(\pi)$, is defined as followed:
\[ \mathcal{F}(\pi) = \{ k \in [1,n]\cap \mathbb{N} \mid \pi(k) = k \}\]

Let $ \Pi_n$ be the set of all possible permutations of $ (1,2,\cdots ,n)$.
The \emph{Rencontres Numbers}\cite{wiki_rn}
$ D_{n,k}$ is defined as followed:
\[ D_{n,k} = \card{ \{ \pi \in \Pi_n \mid \card{\mathcal{F}(\pi)} = k \}}, k = 0,1,\cdots n\]
where $ \card{A}$ is the cardinality of a set $ A$.

In particular, we denote $ D_{n,0}$ as $  D_{n} $ for short.

Let $ \pi $ be a random permutation of $ (1,2,\cdots ,n),$ where
every possible permutation has the same possibility $ \dfrac{1}{n!}$.
The random variable $ X = \card {\mathcal{F}(\pi)} $ is what we will focus on in this paper.

