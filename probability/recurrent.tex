% File: recurrent.tex
% Date: Fri Nov 29 19:42:11 2013 +0800
% Author: Yuxin Wu <ppwwyyxxc@gmail.com>
\section{Recurrent Event}
\subsection{Definition}

我们考虑一个重复实验,每次的结果可能是$E_j (j = 1, 2, \cdots )$.
$\mathcal{E}$是一个有限结果序列的可判定属性.
我们姑且称$\mathcal{E}$是一个事件, 其在序列$E_1, \cdots $的第$n$位"发生",
意思为$E_1, \cdots ,E_n$具有属性$\mathcal{E}$

若属性满足以下两个条件,则称其定义了一个Recurrent Event:
\begin{enumerate}
    \item
      $ \mathcal{E}$在$ E_{1},\cdots,E_{n+m} $的第$ n$和第$ n+m$个位置出现,等价于$ \mathcal{E}$在$ E_{1},\cdots ,E_{n}$和$
      E_{n+1},\cdots ,E_{n+m}$的最后出现

      \item
        若$ \mathcal{E}$在序列第$ n$位出现,则恒有$ P(E_{1},\cdots ,E_{n+m}) = P(E_{1},\cdots ,E_{n})P(E_{n+1},\cdots ,E_{n+m})$
\end{enumerate}

\textbf{例子:}

Random Walk 里的Return, 从负方向Return, 更新最大值;

排队论里的达到空闲.

考虑连续Bernoulli实验中的连续成功次数,若规定不计算重叠,则连续$ r$次成功也是Recurrent Event.
\\

仿照\secref{walk}的记号,设$ \mathcal{E}$在$ n$处出现的概率为$ u_n$,在$ n$处首次出现的概率为$ f_n$.设$ u_0 = 1, f_0=0$.并设其PGF分别为$ U(t),F(t)$.
需要注意,$ u_k$不是一个概率分布,$ f_k$仅当事件一定会发生($ f = \sum_{n=0}^\infty f_n = 1$)时是概率分布.

设随机变量$ T$表示等待时间,$ P(T=n) = f_n,P(T=\infty) = 1-f$.

设第$ r$次出现位置的分布是$ f^{(r)}_k$则它应是$ f^{(r-1)}_k与f_k$的卷积.于是其PGF满足 $ F^{(r)}(t) = (F(t))^r$

至少出现$ r$次的概率为$ (F(1))^r = f^r$. 当$ f=1$时,称$ \mathcal{E}$ \textbf{常返}(persistent),否则称\textbf{暂留}(transient).

若使得$ u_n\neq 0$的所有$ n$有公因子$ \lambda >1$,则称$ \mathcal{E}$是\textbf{周期}的.

由于 $ u_n = f_1u_{n-1}+\cdots f_nu_0 (n\ge 1)\Rightarrow U(s)-u_0 = F(s)U(s)\Rightarrow U(s)=\dfrac{1}{1-F(s)}$

$ \mathcal{E}$暂留等价于$ \dfrac{1}{1-f} = u=\sum_{n=0}^\infty u_n$收敛.

\subsection{Examples}
Bernoulli试验中的Return, 是Recurrent Event.

在$2n$处发生一次的概率为: $ u_{2n} = {{2n}\choose{n}} p^nq^n = {{-\frac{1}{2}}\choose{n}} (-4pq)^n \Rightarrow  U(s) = \dfrac{1}{\sqrt{1-4pqs^2}}$

$ \Rightarrow F(s) = 1 - \sqrt{1 - 4pqs^2} \Rightarrow  f = F(1) = 1 - |p-q|. $

$p=q=\dfrac{1}{2}$时, 为常返, 等待时间$F'(1) = \infty $ ;  否则为暂留.

\vspace{0.5cm}

一次扔$r$个硬币, 全部同面是Recurrent Event. 发生概率为:$ u_n = 2^{-rn}\sum_{k=0}^n{n\choose{k}}^r$

有正态逼近, 有 $ 2^{-n}{n\choose{k}} \to \dfrac{1}{\sqrt{\pi n / 2}} e^{-\frac{(k-n/2)^2}{n/2}}$

可得$u_n \sim \left( \dfrac{2}{\pi n} \right) ^{\frac{r}{2}}\sum_{j=-n/2}^{n/2}e^{\frac{-2rj^2}{n}}
\sim \left( \dfrac{2}{\pi n} \right) ^{\frac{r}{2}} \dfrac{1}{2}\sqrt{\dfrac{n}{r}}\int_{-\infty}^{+\infty}e^{-\frac{x^2}{2}} \mathrm{d}x
\sim \dfrac{1}{\sqrt{r}}\left( \dfrac{2}{\pi n}\right) ^{\frac{r-1}{2}}$

$ r \le 3$时, 常返, 等待时间为$ \infty, r \ge 4$时为暂留.


\subsection{A Theorem of Particular Importance}
  设$ \mathcal{E}$常返,非周期.$ \mu = \E{T} = F'(1),$则$ u_n\to \mu^{-1}$.特别地,$ \mu=\infty$时,$ u_n\to 0$.

\begin{proof}:
  \textbf{(P. Erdos \& W. Feller, 1949)}

  设$\lambda=\sup u_n,u_{n_v}$是趋于$ \lambda$的子列.

  首先证明,$ \forall j, $ s.t.$ f_j>0,u_{n_v-j}\to \lambda$.

  \proof:反设$ u_{n_v-j}\to \lambda'<\lambda$,对足够大的$ v$:
\begin{align*}
  \lambda-\varepsilon < u_{n_v} &=\sum_{i=0}^{n_v}f_iu_{n_v-i}\\
     &<(\sum_{\substack{i=0\\i\neq j}}^{n_v}f_i)(\lambda + \varepsilon) + f_j(\lambda' + \varepsilon)\\
     &<(1-f_j)(\lambda+\varepsilon) + f_j(\lambda' + \varepsilon)
     =\lambda-f_j(\lambda-\lambda') +\varepsilon \Rightarrow \lambda' = \lambda \qed
\end{align*}
重复此操作,可得命题:$u_{n_v}\to \lambda, f_j>0\Rightarrow  u_{n_v-kj}\to \lambda$

由$ \mathcal{E}$的非周期性,所有使得$ f_j>0$的$ j$无公因子.

$\Rightarrow  \exists 足够大的N 和 a_1,\cdots a_l$ ,s.t. $f_{a_i}>0,且 \forall k >N, k 可表为 \sum_{i=1}^l{x_ia_i} $

设$ N^\star = n_v-N,$则$ \forall M, u_{N^\star - M} = u_{n_v - (N+M)}\to \lambda$

再令$ r_n = \sum_{k=n+1}^\infty f_k, R(t) = \sum_{k=0}^{\infty}r_kt^k$ , 由和式变换可知$ R(t)U(t) = (1-t)^{-1}$

即$ \sum_{k=0}^{N^\star}{r_ku_{N^\star - k}} = 1.\Rightarrow  \forall M, \sum_{k=0}^M{r_ku_{N^\star - k}}\le 1$

令式中$v\to \infty , $得$\forall M, \sum_{k=0}^M{r_k} \le \dfrac{1}{\lambda} $

由期望的第二种求法有$ \mu = F'(1) = R(1). $于是 $\lambda \le \dfrac{1}{\mu}$.

重考虑$ \lambda=\inf u_n$的情形,类似以上推理可证$ \lambda \ge \dfrac{1}{\mu}$. (需注意第一步放缩时利用常返性).

于是$ u_n\to \mu$. \qed
\end{proof}

推广:$ \mathcal{E}$常返,具有周期$ \lambda$时,有$ u_{n\lambda}\to \dfrac{\lambda}{\mu}$

\subsection{Number of Occurences}
设$ N_n$为前$ n$次中$ \mathcal{E}$出现的次数. $ N_n = \sum_{k=1}^n{Y_k}, P(Y_n=1) = u_n\Rightarrow \E{Y_n} = u_n$

$ \therefore \E{N_n} = \sum_{k=1}^nu_k. 当等待时间的\mu,\sigma 有限时 \E{N_n}\to \dfrac{n}{\mu}, \Var{N_n}\to \dfrac{n\sigma^2}{\mu^3}$

当$ \mu=\infty$时,$ \E{N_n}$可能不与$ n$同阶(反直觉).

如公平赌博中的Return,$ u_{2n}\to \sqrt{\dfrac{1}{\pi n}}\Rightarrow \E{N_{2n}}\to 2\sqrt{\dfrac{n}{\pi}}$

\subsection{Runs of Specific Patterns}
\textbf{Success Runs:}

设$ \mathcal{E}$为Bernoulli实验中长为$ r$的连续成功(不计算重叠),则$ \mathcal{E}$是Recurrent Event.

依照定义,任意连续的$ r$次成功的实验$ n-r+1,\cdots ,n$中必定有一次出现了$ \mathcal{E}$

$ \Rightarrow  p^r = u_n + u_{n-1}p + \cdots +u_{n-r+1}p^{r-1}, n\ge r$.

以及$ u_1=\cdots =u_{r-1}=0,u_0=1$.

将其写为GF形式: $ p^r(t^r +t^{r+1}\cdots ) = (U(t)-1)(1+pt+p^2t^2+\cdots +p^{r-1}t^{r-1})$

$ \Leftrightarrow U(t) = \dfrac{1-t+qp^rt^{r+1}}{(1-t)(1-p^rt^r)} \Rightarrow  F(t) = \dfrac{p^rt^r(1-pt)}{1-t+qp^rt^{r+1}}$

平均等待时间$ \mu = F'(1) = \dfrac{1-p^r}{qp^r}$
\\

Pattern: SSFFSS.

类似上面的递推,这里有$ p^4q^2 = u_n + p^2q^2u_{n-4} + p^3q^2u_{n-5}$

若欲求期望等待时间,令$ n\to \infty$即可.
\\

\textbf{Runs of Either Kind:}设$\mathcal{E}$为出现一个长为$ a$的连续成功或长为$ b$的连续失败.

设$ \mathcal{E}_1,\mathcal{E}_2$为连续$a$次成功,连续$b$次失败这两个循环事件.

除0时刻外其余的发生概率都应直接相加,所以$ U(t) = U_1(t)+U_2(t)-1$
\\

连续$ a$次成功发生在连续$ b$次失败之前的概率$ x$:

解法1: 设$ u,v$分别为``第一次成功/失败''条件下此事件的概率,列出二元线性方程组解之,$ x=pu+qv$.

解法2:设$ \mathcal{E}_1,\mathcal{E}_2$ 对应的等待时间PGF为$ F(t), G(t),  x_n$为$ \mathcal{E}_1$初次发生在
$ n$且$ \mathcal{E}_2$未发生的概率,其PGF为$ X(n)$,类似定义$ y_n$,则有

$ \begin{cases}
 x_n =  f_n-(\sum_{i=1}^{n-1}y_if_{n-i}) \Rightarrow  X(t) = F(t)-Y(t)F(t) \\
 y_n =  g_n-(\sum_{i=1}^{n-1}x_ig_{n-i}) \Rightarrow  Y(t) = G(t)-X(t)G(t)
\end{cases}\Rightarrow X(t) = \dfrac{F(t)[1-G(t)]}{1-F(t)G(t)}$

利用L' Hospital, $x=\lim\limits_{t\to 1^{-}}X(t)=\dfrac{p^{a-1}(1-q^b)}{p^{a-1}+q^{b-1}-p^{a-1}q^{b-1}}$
\\
