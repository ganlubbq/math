% $File: probability.tex
% $Date: Thu Dec 13 19:45:06 2012 +0800
% $Author: wyx <ppwwyyxxc@gmail.com>

\documentclass[a4paper]{article}
\usepackage{fontspec,amssymb,zhspacing,verbatim,minted, zhmath}
\usepackage[fleqn]{amsmath}
\usepackage[hyperfootnotes=false,colorlinks,linkcolor=blue,anchorcolor=blue,citecolor=blue]{hyperref}
\usepackage[sorting=none]{biblatex}
\usepackage{subfigure}
\usepackage{indentfirst}
\usepackage{extarrows}
\zhspacing


\renewcommand{\abstractname}{摘要}
\renewcommand{\contentsname}{目录}
\renewcommand{\figurename}{图}
\defbibheading{bibliography}{\section{参考文献}}

% \figref{label}: reference to a figure
\newcommand{\figref}[1]{\hyperref[fig:#1]{图\ref*{fig:#1}}}
% \secref{label}: reference to a section
\newcommand{\secref}[1]{\hyperref[sec:#1]{\ref*{sec:#1}节}}
\DeclareMathOperator{\LM}{\tiny{LM}}
\DeclareMathOperator{\LT}{\tiny{LT}}
\DeclareMathOperator{\rank}{\tiny{rank}}
\DeclareMathOperator{\sgn}{sgn}
\DeclareMathOperator{\Cov}{Cov}

\let\Oldsum\sum
\renewcommand{\sum}{\displaystyle\Oldsum}
\let\Oldprod\prod
\renewcommand{\prod}{\displaystyle\Oldprod}
\let\Oldcap\bigcap
\renewcommand{\bigcap}{\displaystyle\Oldcap}
\let\Oldcup\bigcup
\renewcommand{\bigcup}{\displaystyle\Oldcup}
\let\Oldint\int
\renewcommand{\int}{\displaystyle\Oldint}
\let\Oldiint\iint
\renewcommand{\iint}{\displaystyle\Oldiint}

% $File: mint-defs.tex
% $Date: Sat Feb 16 22:59:11 2013 +0800
% $Author: wyx <ppwwyyxxc@gmail.com>

\usepackage{xparse}


% \inputmintedConfigured[additional minted options]{lang}{file path}{
\newcommand{\inputmintedConfigured}[3][]{\inputminted[fontsize=\footnotesize,
	label=#3,linenos,frame=lines,framesep=0.8em,tabsize=4,#1]{#2}{#3}}

% \phpsrc[additional minted options]{file path}: show highlighted php source
\newcommand{\phpsrc}[2][]{\inputmintedConfigured[#1]{php}{#2}}
% \phpsrcpart[additional minted options]{file path}{first line}{last line}: show part of highlighted php source
\newcommand{\phpsrcpart}[4][]{\phpsrc[firstline=#3,firstnumber=#3,lastline=#4,#1]{#2}}
% \phpsrceg{example id}
\newcommand{\phpeg}[1]{\inputminted[startinline,
	firstline=2,lastline=2]{php}{res/php-src-eg/#1.php}}

\newcommand{\txtsrc}[2][]{\inputmintedConfigured[#1]{text}{#2}}
\newcommand{\txtsrcpart}[4][]{\txtsrc[firstline=#3,firstnumber=#3,lastline=#4,#1]{#2}}

\newcommand{\pysrc}[2][]{\inputmintedConfigured[#1]{py}{#2}}
\newcommand{\pysrcpart}[4][]{\pysrc[firstline=#3,firstnumber=#3,lastline=#4,#1]{#2}}

\newcommand{\confsrc}[2][]{\inputmintedConfigured[#1]{squidconf}{#2}}
\newcommand{\confsrcpart}[4][]{\confsrc[firstline=#3,firstnumber=#3,lastline=#4,#1]{#2}}

\newcommand{\cppsrc}[2][]{\inputmintedConfigured[#1]{cpp}{#2}}
\newcommand{\cppsrcpart}[4][]{\cppsrc[firstline=#3,firstnumber=#3,lastline=#4,#1]{#2}}

\renewcommand{\P}[1]{\text{P}\left(#1\right)}
\renewcommand{\Pr}[1]{\text{Pr}\left\{#1\right\}}
\newcommand{\Px}[2]{\text{P}_{#1}\left(#2\right)}
\newcommand{\E}[1]{\text{E}\left[#1\right]}
\newcommand{\Ex}[1]{\text{E}#1}
\newcommand{\Var}[1]{\text{Var}\left[#1\right]}
%\newcommand{\Cov}[2]{\text{Cov}\left[#1,#2\right]}
%\newcommand{\Cov}[1]{\text{Cov}\left[#1 \right]}
\renewcommand{\T}[1]{\Theta\left(#1\right)}
\newcommand{\real}{\mathbb{R}}
\newcommand{\card}[1]{\left\|#1\right\|}
\newtheorem{lemma}{Lemma}

\NewDocumentCommand\Cov{mg}{
    \text{Cov}\left[ #1 \IfNoValueTF{#2}{}{,#2}\right]
 }

\newcommand{\qed}{\hfill \ensuremath{\Box}}



\title{Probability}
\author{\\(ppwwyyxxc@gmail.com)}

\begin{document}
\maketitle
\tableofcontents
% File: basic.tex
% Date: Thu Dec 27 23:09:06 2012 +0800
% Author: Yuxin Wu <ppwwyyxxc@gmail.com>

\section{Basic Concepts}
\subsection{Events}
一次随机试验中每一种可能的结果称为一个{\bf 基本事件}或{\bf 样本点$ \omega$},所有基本事
件的全体为该试验的样本空间$ \Omega$

同一试验的样本空间可能不唯一,因为观察结果的角度不同.对扔两次色子,$ \Omega_1 =
\{++,+-,--,-+\}, \Omega_2 = \{\texttt{两正,两负,一正一负}\}$

至多可数的样本空间称为离散样本空间,不可数称为连续样本空间.

$ \Omega$的可测子集$ A$称为事件.对结果$ \omega \in A$,则称事件A发生了.

$ A \subset B \Rightarrow $A发生了B必发生.

Morgan律:$ (\cup A_i)^c = \cap A_i^c$

\subsection{Probability Space}

概率空间$ (\Omega, \mathcal{F}, P)$:

$\Omega$ 是全体可能结果组成的集合.$ \mathcal{F}$是全体可观测事件组成的事件族.$ P:\mathcal{F} \rightarrow [0,1]$是求事件的概率的运算.

当$ \mathcal{F}$满足以下条件时,称其为{\bf $ \sigma -$代数}:
\begin{enumerate}
	\item $ \Omega \in \mathcal{F}$
	\item $ A\in \mathcal{F} \Rightarrow A^c \in \mathcal{F}$
	\item 可数并:$ A_1 \cdots \in \mathcal{F} \Rightarrow  \cup_{i = 1}^{\infty}{A_i}
		\in \mathcal{F}$
\end{enumerate}
事实上,由可数并,可推出有限并,可数交,有限交$ \in \mathcal{F}$.

当$ \Omega$为至多可数集时,总可取$ \Omega$的所有子集族作为$ \mathcal{F}$.
当$ \Omega$不可数时,取这样的$ \mathcal{F}$会造成数学上的困难,因此只取感兴趣的,
可以知道概率的事件的最小$ \sigma$-代数.
\\

{\bf 概率的定义:}对每个事件$ A\in \mathcal{F},$定义实数$ P(A)$,满足以下条件:
\begin{enumerate}
	\item 非负性:$ P(A) \ge 0$
	\item 规范性: $ P(\Omega) = 1$
	\item 可数可加性:

		对两两互不相容的事件$ A_1 \cdots \in \mathcal{F}, P(\cup{A_n}) =
		\sum{P(A_n)}$
\end{enumerate}

试验的样本空间,事件域($ \sigma$代数)及定义在其上的概率构成的三元组$ (\Omega,
\mathcal{F}, P)$称为描述一个随机试验的{\bf 概率空间}.
\subsection{Properties of Probability}

事件序列的极限定义:
$ \overline {\lim \limits_{n \to \infty}}A_n  = \bigcap_{n
=1}^{\infty}\bigcup_{k=n}^{\infty}A_k $(当且仅当有无穷个$ A_n$发生)

$ \mathop{\underline \lim} \limits_{n \to \infty} A_n = \bigcup_{n
=1}^{\infty}\bigcap_{k=n}^{\infty}A_k$(当且仅当至多有有限个$ A_n$不发生)

当上下极限相等时(如对于单调事件序列),称为序列$ A_n$的极限.
\\

利用可数可加,可得到如下结论:
\begin{enumerate}
	\item $ P(\empty) = 0:
		P(\Omega) = P(\Omega \cup \empty \cup \empty \cdots) = P(\Omega) +
		P(\empty) + \cdots$
	\item 有限可加
	\item 求逆:$ P(A) + P(A^c) = 1$
	\item Jordan公式(容斥),归纳证明
	\item $ P(A - B) = P(A) - P(A\cap B)$,特别地,$ B \subset A \Rightarrow
		P(B) \le P(A)$
	\item 下连续性:设$ A_i$单调增($ A_1 \subset A_2 \subset \cdots$),则$
		P(\lim \limits_{n \to \infty }{A_n}) = \lim \limits_{n \to
		\infty}{P(A_n)}$.

		$ P(\cup A_n) = P(A_1) + \sum_{i = 1}^{\infty}{P(A_{i+1} - A_i)} =
		P(A_1) = \lim \limits_{n \to \infty}\sum_{i = 1}^{n-1}{[P(A_{i+1} -
		P(A_i))]} = \lim \limits_{n \to \infty}P(A_n)$

	\item 上连续性:设$ A_i$单调减,则$ P(\lim \limits_{n \to \infty} A_n ) =
		\lim \limits_{n \to \infty} P(A_n)$

		$ 1 - \lim \limits_{n \to \infty}P(A_n) = \lim \limits_{n \to
		\infty}P(A_n^c)=P(\bigcup_{n=1}^{\infty}A_n^c) =
		P((\bigcap_{n=1}^{\infty}A_n)^c) = 1-P(\bigcap_{n=1}^{\infty}A_n)$

		概率的上下连续性等价,统称为连续性.

	\item	有限可加+下连续$ \Leftrightarrow $可数可加.

		由下连续性,\[  P(\bigcup_{n=1}^{\infty}A_n) =
		P(\bigcup_{n=1}^{\infty}F_n) \mathop{=}\limits_{\texttt{下连续}}\lim\limits_{n \to \infty}P(F_n)=
		\lim\limits_{n\to\infty}P(\bigcup_{i=1}^nA_i)
		\mathop{=}\limits_{\texttt{有限可加}}
		\lim\limits_{n\to\infty}\sum_{i=1}^n{P(A_i)}\mathop{=} \limits_{\texttt{收敛}}
	\sum_{i=1}^{\infty}P(A_i)\]

\item 推广可数可加:
	$ A_1,A_2\cdots $满足$ P(A_iA_j)=0$(弱于互斥),则$ P(A) = \sum{P(A_n)}$
\end{enumerate}

\subsection{Classical Definitions of Probability}

{\bf 古典概型}: 基本事件只有有限个且概率相同.

掷硬币$ n$次,取每种排列为基本事件,即为古典概型:
\[  P(\texttt{首次正面出现在k次}=\dfrac{1}{2^k})\]

掷硬币直到出现正面为止,基本事件$ \omega_k$为``首次正面出现在第k次'',则有无穷个基本事件,且
概率不同.
利用可数可加性,$ \omega_{\infty} = 0$,但不是不可能事件.

一般地,对于至多可数集合$ \Omega$,每个基本事件的概率都可求出时,$ \forall A
\subset \Omega, P(A) = \sum_{\omega \in A}P(\omega)$

若无限抛掷硬币,将排列作为基本事件,则有不可数个基本事件,此时若考虑等可能分析,则
每个基本事件概率为0.无法求出某个事件的概率(因为不可数个实数的和没有意义).

{\bf 几何概型}:随机现象的 样本空间充满某个可测区域,且任一点落在度量相同的子区域
内是等可能的.

{\bf Buffon投针的分析做法}:设针中点与最近平行线距$ x\in [0, \dfrac{d}{2}]$,与
直线成角$ \varphi\in[0,\pi]$,在上区域中求$ x \le \dfrac{l}{2}\sin{\varphi}$
部分的概率. $ P(A) =
\dfrac{\int_{0}^{\pi}{\dfrac{l}{2}\sin{\varphi}\mathrm{d}{\varphi}}}{\dfrac{d}{2}\pi}
= \dfrac{2l}{d\pi}$

古典/几何概型的另一个问题:Bertrand悖论--圆内一弦长度超过正三角形边长的概率由三
种解释.

原因:当可能结果有无穷个时,难以规定``等可能''这一概念,因此概率空间被模糊定义了.

\subsection{Conditional Probability}
概率空间$( \Omega, \mathcal{F}, P)$中$ P(B) > 0$.定义$ P_B(A) = P(A | B) = \dfrac{P(A\cap B)}{P(B)}$,
则可证$ (\Omega, \mathcal{F}, P_B)$也是概率空间.

  {\bf 乘法公式:}$ P(\bigcap_{i=1}^{n-1}A_i)>0$ (使得条件概率有意义)时,由定义归纳可得

\[ \Rightarrow P(\bigcap_{i=1}^nA_i) = \prod_{i=1}^n{P(A_i | \bigcap_{j=1}^{i-1}A_j)}\]

{\bf 可靠性函数与风险率:}设前$ t$时刻正常,$[t, t+\Delta t]$时段失效的概率为$ \lambda(t)\Delta t + o(\Delta t)$,求
设备在$ (0,t)$上无故障的概率.

设$ A_t$表示设备在$ (0,t)$内正常,$ P(\overline{A_{t+\Delta t}} | A_t) = \lambda(t)\Delta t + o(\Delta t)$.
\[ p(t + \Delta t) = P(A_t)P(A_{t+\Delta t}|A_t) = p(t)[1 - \lambda(t)\Delta t + o(\Delta t)]\Rightarrow \dfrac{dp(t)}{dt}=-\lambda(t)p(t)\]

注意到$ p(0)=1,$有$ p(t) = e^{-\Oldint_0^t{\lambda(s)ds}}$
\\

{\bf 全概率公式:}设$ B_1,B_2\cdots$为样本空间$ \Omega$的一个正划分,则
\[ P(A) =\sum_{i=1}^{\infty}{P(AB_i)}= \sum_{i=1}^{\infty}{P(B_i)P(A|B_i)}\]

{\bf 赌徒输光:}两人各有赌资$ i, n-i$,每次赌博胜者拿走对方1元,胜率分别为$ p,1-p$.

设$ A_i$表示甲有$ i$元,最终破产.$ B$表示某次甲胜,$ P(B) = p$,则有$ P(A_i|B)=P(A_{i+1}),P(A_i|\overline{B})=P(A_{i-1})$

于是$ P(A_i)=pP(A_{i+1}) + (1-p)P(A_{i-1})$,边界$ P(A_0)=1,P(A_n)=0$.

\[  P(A_i)=\left\{ \begin{array}{lc} 1-\dfrac{1-r^i}{1-r^n}  & p \ne \dfrac{1}{2} \\ 1-\dfrac{i}{n}& p = \dfrac{1}{2} \end{array}  \right.r = \dfrac{1-p}{p} \]

对赌场$ (n \to \infty)$,甲最终会输光的概率为$ P(A_i) = \min\{1,r^i\}$
\\

{\bf Polya 模型}:从黑球,红球中任取若干次.取出的红球与黑球个数确定的情形下,概率是否与次序相关.

若放回抽样,结果不影响下次,故概率相等.

若不放回抽样,前次结果影响后次,但概率仍与次序无关.

若放回若干同色球(传染病模型),每次取出会增加下次取出同色球的概率. 但结果与次序无关.

若放回若干异色球(安全模型),结果才与次序有关.
\\

{\bf 敏感问题问卷调查:}在问卷上要求每个人准备一枚硬币,对于指定的隐私题目,请填写人投掷一次硬币:如果正
面朝上,则如实填写个人的真实情况;如果反面朝上,那么就再投掷一次硬币,正面就填"是",反面就填"否".当然,若
第一次投掷硬币为正的话,填写人完全可以假装再投一次硬币来掩人耳目.

假设回收后有效问卷有$M$份,其中该问题答"是"的有$N$个人.如实填写了该问题的人平均有$\dfrac{M}{2}$个;在另
外$ \dfrac{M}{2}$人中,平均有$ \dfrac{M}{4}$人答的"是".因此,我们所需要的最终结果应该为$ \dfrac{(N-M)/4}{M/2} $
\\

{\bf Bayes:}$ P(B_i|A) = \dfrac{P(B_i)P(A|B_i)}{P(A)} = \dfrac{P(B_i)P(A|B_i)}{\sum_{t=1}^n{P(B_t)P(A|B_t)}}$

\subsection{Independence}
{\bf 定义:}$ P(AB)=P(A)P(B)$. 实际中以经验判断.

$ A,B$独立$ \Rightarrow A与\mathcal{F}_B$中任一事件独立.

多个事件相互独立:直观想法--$ A与\mathcal{F}_{B,C\cdots}$中任一事件独立.
\\

{\bf 定义}:其中任意$ k$个事件的交的概率等于概率的乘积.

无穷个事件相互独立:任意有限个事件相互独立.

$ A_1\cdots A_n$ 相互独立,则任意对其分组,各组事件分别产生的事件域相互独立.

相关系数:$ r(A,B)=\dfrac{P(AB)- P(A)P(B)}{\sqrt{P(A)[1-P(A)]P(B)[1-P(B)]}}$

$ \left \{\begin{matrix}-P(A)P(B)\\  -[1-P(A)][1-P(B)]\end{matrix} \right . \le P(AB)-P(A)P(B) \le \left \{ \begin{matrix} P(A)[1-P(B)]\\ P(B)[1-P(A)]\end{matrix}\right .
			\Rightarrow  |r(A,B)| \le 1$

$ r(A,B)=1\Leftrightarrow P(A)=P(AB)=P(B)$

$ r(A,B)>0\Leftrightarrow P(A|B)>P(A)\Leftrightarrow P(B|A)>P(B)$

% File: rv.tex
% Date: Fri Dec 14 23:47:19 2012 +0800
% Author: Yuxin Wu <ppwwyyxxc@gmail.com>
\section{}
\subsection{Random Variables}
样本空间$ \Omega \rightarrow \mathbb{R}$的函数$ X = X(\omega)$称为\textbf{随机变量}.值域有限或可列称为离散随机变量,值域充满数
轴上的某个区间,称为连续随机变量.记$ F(x) = P(X \le x)$为$ X$的\textbf{分布函数}.

显然,$ F$是$ (-\infty, \infty)$的单调不减函数,有界,于是各点有左右极限,且无穷处有极限.

\begin{equation*}
	\begin{split}
		F(b) - F(c + 0) & = F(b) - \lim \limits_{a \to c^{+}} F(a) =\lim \limits_{a \to c^{+}}P(a < X \le b) \\
		                & = P(\bigcup_{a \to c^{+}}\{a < X \le b\}) = P(c < X \le b)                         \\
		F(d - 0) - F(a) & = P(a < X < d)
	\end{split}
\end{equation*}

即$ F(c + 0) = F(c), F(d - 0) = P(X<d), 且可得F(-\infty) = 0, F(\infty) = 1$.

满足以上性质的函数$ F$必定为某随机变量的分布函数.

任意Borel集$ B\subset \mathbb{R}, P(X \in B) 可由F$ 计算得到.
特别的,$ P(X=x) = F(x) - F(x - 0),若F在x连续,则P(X=x) = 0$
\\

\textbf{$p$-分位数:}满足$ P(X \le x) \ge p, P(X <x) \le p 的x. $

对连续随机变量,等价定义$F(x) = p$的点为$p$-(下侧)分位数.

$ p =\dfrac{1}{2}$时称为\textbf{中位数}.$\Leftrightarrow \dfrac{1}{2} \le F(x) \le \dfrac{1}{2} + P(X=x) $

中位数的统计意义:使得$E|X-a|$最小的$a$
\\

存在非负可积函数,使得$ F(x) = \int_{-\infty}^{x}{p(t)dt}$,则称$ p(x)$为$ X$的\textbf{概率密度函数}.

在$ F(x)$导数存在的点有$ p(x) = F'(x)$,其余点处$ p(x)$可任意取值.

\subsection{Expectation \& Variance}
\begin{equation*} \begin{split}
	E(X) & = \int_{\mathbb{R}}{x\mathrm{d}F(x)},其中积分为Lebsegue积分. \\
			& = \left\{  \begin{matrix}
	\sum{x_ip(x_i)} = \sum_{n=0}^{\infty}{P(X>n) - P(X<-n)}, 离散,\sum{|x_i|}P(X = x_i)< \infty \\
	\int_{\mathbb{R}}{xp(x)\mathrm{d}x} = \int_{\mathbb{R^+}}{P(X>x)\mathrm{d}x} - \int_{\mathbb{R^-}}{P(X<x)\mathrm{d}x},连续,\int_{\mathbb{R}}{|x|p(x)\mathrm{d}x} < \infty
\end{matrix}\right.
\end{split} \end{equation*}

$\bullet$ 绝对收敛保证了和的存在且与顺序无关.
\\

当期望存在时: $nP(X > n) = n\int_n^{\infty}\mathrm{d}F(x)\le \int_n^{\infty}x\mathrm{d}F(x)$

上式取极限$n\to \infty, 得 nP(X>n) \to 0 ,即 \lim \limits_{x\to \infty}x(1-F(x)) = 0$

同理有$\lim \limits_{x \to -\infty}xF(x) = \lim \limits_{x\to \infty}x(1-F(x)) = 0$

由此极限可推出\textbf{期望的几何意义:} $\int_{-\infty}^{EX}{F(x)\mathrm{d}x} = \int_{EX}^{\infty}{(1-F(x))\mathrm{d}x}$

即:$y=F(x), x=EX, 将0\le y \le 1$分成面积相等的两部分.

\textbf{证:}对两边进行分部积分即可.
\\

\textbf{Cauchy-Schwartz:}

$E(X^2),E(Y^2) < \infty, 则(E(XY))^2 \le E(X^2)E(Y^2)$

\textbf{证:}考虑$ f(u) = E(Xu+Y)^2 = (EX^2)u^2 + 2E(XY)u+EY^2$的判别式即可.
\\

\textbf{期望的统计意义:}

$E(X-a)^2 =E(X-EX)^2 + 2 E(X-EX)(EX-a) + (EX-a)^2= E(X-EX)^2 + (EX-a)^2 \ge E(X-EX)^2. $
\\

\textbf{方差:}

若$ E(X^2)$存在,则定义$ Var(X) = E(X - E(X))^2 = \left \{ \begin{matrix}\sum(x_i - E(X))^2p(x_i) \\ \int_{\mathbb{R}}{(x-E(X))^2p(x)\mathrm{d}x} \end{matrix}\right.$

$ Var(X) = E(X^2 - 2XE(X) + (E(X))^2) = E(X^2) - (E(X))^2$
\\

\textbf{其他统计量:}

变异系数$ C_v(X) = \dfrac{\sigma(X)}{EX} = \dfrac{\sqrt{Var(X)}}{EX}$.消去了量纲的影响.

偏度系数$ \beta_{s} = \dfrac{E(X-EX)^3}{(Var(X))^{\frac{3}{2}}}$.描述偏离对称性的程度.

峰度系数$ \beta_{k} = \dfrac{E(X-EX)^4}{(Var(X))^2} - 3$.描述相比于正态分布的尖峭程度(尾部粗细).正态分布峰度为0.

\begin{flalign*}
	\textbf{(Chebyshev)} & \forall \varepsilon > 0, P(|X-EX| \ge \varepsilon)          \\
	=                    & \int_{|x- EX| \ge \varepsilon}{p(x)\mathrm{d}x} \le \int_{|x-EX| \ge \varepsilon}{\dfrac{(x-EX)^2}{\varepsilon^2}p(x)\mathrm{d}x} \\
	\le                  & \dfrac{\int_{\mathbb{R}}{(x-EX)^2}p(x)\mathrm{d}x}{\varepsilon^2} = \dfrac{Var(X)}{\varepsilon^2}\\
	统计意义:&与均值的距离远近对概率的限定.
\end{flalign*}

\subsection{Characteristic Function}
$ \varphi_X(t) = E(e^{itX})$

$ \varphi_{\vec{X}}(\vec{t}) = Ee^{i\vec{t}^T\vec{X}}$

\begin{description}

  \item[界:]$ |\varphi_X(t)| = |Ee^{itX}| \le E|e^{itX}| = 1$

  \item[对称性:] $ \varphi_X(t)是实偶函数 \Leftrightarrow \varphi_X(t) = \overline{\varphi_X(t)} = E\overline{e^{itX}}=Ee^{-itX} = \varphi_X(-t) = \varphi_{-X}(t)
      \Leftrightarrow F_X(x) = F_{-X}(x)为对称分布\Leftrightarrow F_X(x)关于(0,\dfrac{1}{2})对称$

        \item[线性变换:] $ \varphi_{a+bX}(t) = e^{iat}\varphi_X(bt)$

        \item[卷积变乘积:] $ X,Y$独立,$ \varphi_{X+Y}(t) = Ee^{it(X+Y)} = \varphi_X(t)\varphi_Y(t)$

          \item[矩的计算:] $ \varphi^{(k)}(t) = \int_{\mathbb{R}}i^kx^ke^{itx}p(x)\mathrm{d}x = i^kE[X^ke^{itX}]\Rightarrow \varphi^{(k)}(0) = i^kEX^k$

          \item[独立性判定:] $ X_1,\cdots ,X_n独立\Leftrightarrow \varphi_{\vec{X}}(\vec{t})=\prod_{k=1}^n{\varphi_{X_k}(t_k)}$

          \item[分析性质:]\hfill\\
   $ \varphi(x)在\mathbb{R}上一致连续$

对连续型随机变量$ X, \varphi_X(t) = \int_{\mathbb{R}}e^{itx}p(x)\mathrm{d}x$为$ p(x)$的Fourier变换.

于是有逆变换: $F(x) = \lim \limits_{y \to - \infty} \lim \limits_{T\to \infty}\dfrac{1}{2\pi}\int_{-T}^T{\dfrac{e^{-ity}-e^{-itx}}{it}\varphi(t)\mathrm{d}t} $

$\hspace{2.3cm} p(x) = \dfrac{1}{2\pi}\int_{\mathbb{R}}e^{-itx}\varphi_X(t)\mathrm{d}t$

  $即 F(x)与\varphi(t)有一一对应的关系. 且连续性定理:$
  \[ \{ F_n(x)\}弱收敛到F(x)\Leftrightarrow \{ \varphi_n(x)\}收敛到\varphi(x)\]
表明这种对应关系也是连续的
      \end{description}


\subsection{Common Discrete Distributions}
\begin{enumerate}
  \item \textbf{Bernoulli分布(二项分布)}$ b(n,p)$

    $ P(X = k) = C_n^kp^k(1-p)^{n-k}, k = 0,\cdots,n$

    $ EX = np, Var(X) = E[X(X-1)]+EX-(EX)^2 = np(1-p)$

    $	b(n,p)  = n * b(1,p)\Rightarrow \varphi(t) = (1-p+pe^{it})^n$

  二项分布的最大值发生在$ k = \left \{ \begin{matrix}
      (n+1)p , (n+1)p - 1 &, (n+1)p \in \mathbb{N} \\
      \lfloor(n+1)p\rfloor&,  (n+1)p \not \in \mathbb{N}
    \end{matrix}\right.$

\item \textbf{Poisson 分布} $ P(\lambda),(\lambda > 0)$

  $ P(X = k) = \dfrac{\lambda^k}{k!}e^{-\lambda}$

  最大值发生在$ k = \lfloor \lambda \rfloor$

  $ EX = Var(X) = \lambda, \varphi(t) = \sum_{k=0}^{\infty}e^{itk}\dfrac{\lambda^k}{k!}e^{-\lambda} = e^{\lambda(e^{it} - 1)}$

  \textbf{Poisson 定理},对二项分布$ b(n,p_n),\lim \limits_{n\to \infty} np_n = \lambda > 0$,
  \begin{flalign*}
    P(X = k) & = C_n^kp_n^k(1-p_n)^{n-k}                                                                                \\
    & =\dfrac{n(n-1)\cdots(n-k+1)}{k!n^k}(np_n)^k(1-\dfrac{np_n}{n})^{n-k}                                     \\
    & =(1-\dfrac{1}{n})\cdots(1-\dfrac{k-1}{n})\dfrac{[\lambda + o(1)]^k}{k!}[1-\dfrac{\lambda+o(1)}{n}]^{n-k} \\
    & \rightarrow \dfrac{\lambda^k}{k!}e^{-\lambda}, n\to \infty
\end{flalign*}

    \item \textbf{几何分布}$ G(p)$

      $ P(X=k) = p(1-p)^{k-1}, k \in \mathbb{N^+}$

      $ EX = \sum_{n=1}^{\infty}\sum_{k=1}^{n}{p(1-p)^{n-1}}=
      \sum_{k=1}^{\infty}\sum_{n=k}^{\infty}{p(1-p)^{n-1}}= \dfrac{1}{p}$

      类似方法使用两次求出$ E(X(X-1)),Var(X) = \dfrac{1-p}{p^2}$

      $ \varphi(t) = \dfrac{pe^{it}}{1-(1-p)e^{it}}$

      尾概率$ P(X>m) = (1-p)^m$

      无记忆性 $\Leftrightarrow P(X>m+n) = P(X>m)P(X>n)\Leftrightarrow X\sim G(P(X\le 1)) $

          (即解Cauchy方程)

        \item \textbf{超几何分布} $ h(n,N,M),(n, M\le N)$

          意义:$ N$件物品含有$ M$件次品,不放回抽取$ n$次得到的次品数.

          $ P(X=k) = \dfrac{C_M^kC_{N-M}^{n-k}}{C_N^n}, k \in [\max\{0, n-N + M\}, \min\{M,n\}]$

          $ EX = \sum_{k}{k\dfrac{C_M^kC_{N-M}^{n-k}}{C_N^n}}  = \dfrac{Mn}{N}\sum_{k}{\dfrac{C_{M-1}^{k-1}C_{N-M}^{n-k}}{C_{N-1}^{n-1}}} = \dfrac{Mn}{N}$

          类似地使用Vandermonde Convolution,有$ 	E[X(X-1)] = \dfrac{M(M-1)n(n-1)}{N(N-1)}$

          $\Rightarrow Var(x)  = \dfrac{nM(N-M)(N-n)}{N^2(N-1)} $

        二项逼近:当$N\to\infty, \frac{M}{N}\to p $时,$ h(n,N,M)\to b(N,p)$

      \item \textbf{Pascal 分布(负二项分布)} $ Nb(r,p)$

        意义:事件发生第$ r$次时的实验次数.$ Nb(r, p) = r * G(p) $

        $ P(X = k) = C_{k-1}^{r-1}p^r(1-p)^{k-r}, k = r,r+1,\cdots$

        $ EX = \dfrac{r}{p}, Var(x) = \dfrac{r(1-p)}{p^2}$

        $ \varphi(t) = (\dfrac{pe^{it}}{1-(1-p)e^{it}})^r$

      \item \textbf{错排问题} $ X $为匹配到自己的人数.

        $ P(X = k) = \dfrac{C_n^kD_{n-k}}{n!}, 其中D_k = k!\sum_{i=0}^{k}{\dfrac{(-1)^i}{i!}}为错排数$

        $ EX = \sum_{k=0}^n{\dfrac{nC_{n-1}^{n-k}D_{n-k}}{n!}} = 1. $

        或:每个人匹配到自己的期望为$ \dfrac{1}{n},$所以总期望为$ 1$

        $ E[X(X-1)] = 1, Var(X) = 1$
    \end{enumerate}
    \subsection{Common Continuous Distributions}
    \begin{enumerate}
      \item \textbf{均匀分布} $U[a,b]$

        $p(x) = \dfrac{1}{b-a}I_{[a,b]}(x) = \left \{ \begin{matrix}\dfrac{1}{b-a}, & a\le x \le b \\0, &  \text{else}\end{matrix}\right.$

        $EX = \dfrac{a+b}{2}, Var(X) = \dfrac{(b-a)^2}{12}$

        $ U[-1,1]$的特征函数$\varphi_{X'}(t)=\int_{0}^1{\cos(tx)\mathrm{d}x} = \dfrac{\sin t}{t}\Rightarrow \varphi(t) =
      e^{\frac{(a+b)it}{2}}\dfrac{\sin(\frac{(b-a)t}{2})}{\frac{(b-a)t}{2}}$

      或:$ \varphi(t) = \int_{a}^b{\dfrac{e^{itx}}{b-a}\mathrm{d}x} = \dfrac{e^{ibt}-e^{iat}}{it(b-a)}$

    \item \textbf{指数分布}$Exp(\lambda),(\lambda > 0)$

      $p(x) = \lambda e^{-\lambda x}I_{\mathbb{R^+}}(x)$

      $F(x) = 1-e^{-\lambda t}$

      设备在时刻$t$的失效率$\lambda(t) = \lambda$为常数,则寿命$X\sim Exp(\lambda)$

      $EX = \int_{\mathbb{R^+}}{P(X>x)\mathrm{d}x} = \int_{\mathbb{R^+}}{e^{-\lambda x}\mathrm{d}x} = \dfrac{1}{\lambda}$

      $ EX^2 = \int_{\mathbb{R^+}}{P(X^2 > x)\mathrm{d}x} \xlongequal{x = u^2} \dfrac{2}{\lambda}\int_{\mathbb{R^+}}{u\lambda e^{-\lambda u}\mathrm{d}u}
      = \dfrac{2EX}{\lambda} = \dfrac{2}{\lambda^2}$

      $Var(X) = EX^2 - (EX)^2 = \dfrac{1}{\lambda^2}$

      $ \varphi(t) = \lambda\int_{\mathbb{R}^+}e^{itx-\lambda x}\mathrm{d}x=\dfrac{\lambda}{\lambda-it}$

      无记忆性 $\Leftrightarrow P(X>m+n) = P(X>m)P(X>n)\Leftrightarrow X\sim Exp(\lambda)$

          $(0,t]$发生的次数$\sim P(\lambda t),$则第一次发生的时间$\sim Exp(\lambda)$

          $ X\sim U(0,1)\Rightarrow  \dfrac{-\ln X}{\lambda}\sim Exp(\lambda)$


      \item \textbf{一维Gauss分布(正态分布)}$N(\mu, \sigma^2)$

        $p(x) = \dfrac{1}{\sqrt{2\pi \sigma^2}}e^{-\frac{(x-\mu)^2}{2\sigma^2}}$

        对$Y\sim N(\mu, \sigma^2), $标准化:$X = \dfrac{Y - \mu}{\sigma},则 X\sim N(0,1)$

        $ P(X \le x) = \int_{-\infty}^{x}{\frac{1}{\sqrt{2\pi}}e^{-\frac{u^2}{2}}\mathrm{d}u} \xlongequal{def}\Phi(x)$

        利用Poisson积分,可验证$\Phi(\infty) = 1$

        $ P(|Y-\mu| < k\sigma) = \Phi(k) - \Phi(-k) = 2\Phi(k)-1$
        \\

        $EX = \int_{\mathbb{R}}{x\dfrac{1}{\sqrt{2\pi}}e^{-\frac{x^2}{2}}\mathrm{d}x} = 0$

        $ Var(X) = EX^2 = 1. 分部积分$

        $ EX^{2k+1} =0, EX^{2k} = (2k-1)!!$

        $\varphi_X(t) = 2\int_{\mathbb{R}^+}\cos(tx)\dfrac{1}{\sqrt{2\pi}}e^{-\frac{x^2}{2}}\mathrm{d}x$,

        求导有$ \dfrac{\mathrm{d}\varphi_X(t)}{\mathrm{d}t} = 2\int_{\mathbb{R}^+}\sin(tx)\mathrm{d}(\dfrac{1}{\sqrt{2\pi}}e^{-\frac{x^2}{2}})
        = -t\varphi_X(t)\Rightarrow \varphi_X(t) = e^{-\frac{t^2}{2}}$.

      于是,$ \varphi_Y(t) = e^{i\mu t-\frac{\sigma^2t^2}{2}}$
      \\

      误差函数:$ Erf(x) = 2\Phi(x\sqrt{2})-1$

    \item \textbf{Gamma分布} $\Gamma(\alpha, \lambda),(\alpha > 0为形状参数,\lambda>0为尺度参数)$

      $ \Gamma 函数: \Gamma(x) = \int_{0}^{\infty}t^{x-1}e^{-t}\mathrm{d}t$

      $ p(x) = \dfrac{\lambda^{\alpha}}{\Gamma(\alpha)}x^{\alpha - 1}e^{-\lambda x}I_{[0,\infty)}(x)$

      $ EX = \dfrac{\lambda^{\alpha}}{\Gamma(\alpha)}\Gamma(\alpha+1) = \dfrac{\alpha}{\lambda}$

      $ EX^2 = \dfrac{\alpha(\alpha+1)}{\lambda^2},Var(X) = \dfrac{\alpha}{\lambda^2}$

      $ \Gamma(1,\lambda )= Exp(\lambda), \Gamma(\dfrac{n}{2},\dfrac{1}{2}) = \chi ^2(n)$

      $ \varphi(t) = (\dfrac{\lambda-it}{\lambda})^{\alpha}$
    \item \textbf{Beta分布} $ Be(a,b),(a,b>0)$

      $ Beta函数:B(a,b) = \int_{0}^{1}{x^{a-1}(1-x)^{b-1}\mathrm{d}x} = \dfrac{\Gamma(a)\Gamma(b)}{\Gamma(a+b)}$

      $ p(x) = \dfrac{1}{B(a,b)}x^{a-1}(1-x)^{b-1}I_{(0,1)}(x)$

      $ EX = \dfrac{\Gamma(a+b)}{\Gamma(a)\Gamma(b)}\dfrac{\Gamma(a+1)\Gamma(b)}{\Gamma(a+b+1)} = \dfrac{a}{a+b}$

      $ EX^2 = \dfrac{\Gamma(a+b)}{\Gamma(a)\Gamma(b)}\dfrac{\Gamma(a+2)\Gamma(b)}{\Gamma(a+b+2)} = \dfrac{a(a+1)}{(a+b)(a+b+1)}$

      $ Var(X) = \dfrac{ab}{(a+b)^2(a+b+1)}$

      $ p(x)=ax^{a-1}\Leftrightarrow X\sim Be(a,1)$

      \item \textbf{卡方分布} $ \chi^2(k).(自由度k>0)$

        $ Y_1 \xlongequal{i.i.d} \cdots \xlongequal{i.i.d} Y_k \sim N(0,1)
      \Rightarrow X = \sum{Y^2}\sim \chi^2(k)$

      $ p(x) = \dfrac{x^{\frac{k}{2}-1}e^{-\frac{x}{2}}}{2^{k/2}\Gamma(\frac{k}{2})} I_{[0,+\infty)(x)}$

      $ EX = k. Var(X) = 2k$

      $ \varphi(t) = (1-2it)^{-\frac{n}{2}}$

    \item \textbf{对数正态分布}$ LN(\mu, \sigma^2)$

      $ X\sim N(\mu,\sigma^2)\Rightarrow Y = e^{X}\sim LN(\mu,\sigma^2)$

    $ p(x) = \dfrac{1}{\sqrt{2\pi}x\sigma}e^{-\frac{(\ln x- \mu)^2}{2\sigma^2}}I_{\mathbb{R}^{+}}(x)$

    $ EX = e^{\mu + \frac{\sigma^2}{2}}, Var(X) = e^{2\mu+\sigma^2}(e^{\sigma^2}-1). 中位数e^\mu$

  \item \textbf{Cauchy分布}$ Cauchy(\mu, \lambda).$

    $ Y \xlongequal{i.i.d} Z \sim N(0,1) \Rightarrow \dfrac{Y}{Z}\sim Cauchy(0,1)$

  $ p(x) = \dfrac{\lambda}{\pi(\lambda^2 +(x-\mu)^2)}$

  期望与方差不存在.

  $ \varphi(t) = e^{i\mu t - \lambda |t|}$

\item \textbf{Weibull分布}$ W(\lambda,k)$

  设备失效率=$ \lambda t^{k-1}\Rightarrow  寿命服从W(\lambda,k). $

$ W(\lambda,1) = Exp(\lambda)$

$ p(x) = \dfrac{k}{\lambda}(\dfrac{x}{\lambda})^{k-1}e^{-(\frac{x}{\lambda})^k}I_{[0,+\infty)}(x)$

$ F(x) = 1-e^{-(\frac{x}{\lambda})^k}$

$ EX = \lambda \Gamma(1+\dfrac{1}{k}), Var(X) = \lambda^2\Gamma(1+\dfrac{2}{k})-(EX)^2$

    \item \textbf{Laplace分布} $ L(\mu,b)$

      $ p(x)=\dfrac{1}{2b}e^{-\frac{|x-\mu|}{b}}, F(x)=\dfrac{1+\sgn(x-\mu)}{2}(1-e^{-\frac{|x-\mu|}{b}})$

      $ X,Y\sim Exp(\frac{1}{\lambda})\Rightarrow X-Y\sim L(0,\lambda), |X-Y|\sim Exp(\dfrac{1}{\lambda})$

    $ X,Y\sim \chi^2(1)\Rightarrow X-Y\sim L(0,1)$

\end{enumerate}

\textbf{密度公式:}
函数$g(x)单调连续,h(x) = g^{-1}(x)连续可微,则Y = g(X)$的密度函数为$ p_Y(y) = p_X(h(y))|h'(y)|$

\textbf{证:}$F_Y(y) = P(g(X)\le y)=\int_{h(-\infty,y]}{p_X(x)\mathrm{d}x}
\xlongequal{x=h(z)}\int_{-\infty}^{y}RHS$

应用: $X\sim \Gamma(\alpha, \lambda) \Rightarrow  kX \sim \Gamma(\alpha, \dfrac{\lambda}{k}), 2\lambda X \sim \chi^2(2\alpha)$
\\

\textbf{对角线:}$ F_X(x)$若为严格增的连续函数,$F_X(X) \sim U(0,1)$

\textbf{证}:$ F_{F_X(X)}(x) = P(F_X(X)\le x) = P(X \le F_X^{-1}(x)) = F_X(F_X^{-1}(x)) =x$

% File: multivariate.tex
% Date: Fri Nov 29 11:47:49 2013 +0800
% Author: Yuxin Wu <ppwwyyxxc@gmail.com>
\section{Multivariate Distributions}
\subsection{Joint Distribution}
\textbf{联合分布函数:}$ F(x_1,x_2,\cdots, x_n) = P(X_1\le x_1,X_2\le x_2,\cdots,X_n\le x_n)$
\\

二维联合分布函数$ F(x,y)\Leftrightarrow $
\begin{enumerate}
  \item 对每个变元单调非减,右连续;

  \item $ F(-\infty,y) = F(x,-\infty)= 0\le F(x,y)\le 1= F(\infty,\infty)$;

  \item $ P(a<X\le b, c<Y\le d) = F(b,d)-F(a,d)-F(b,c)+F(a,c)\ge 0$.

\end{enumerate}

\textbf{边际分布函数:}$ F_X(x) = F(x,\infty), F_Y(y) = F(\infty,y)$

\textbf{联合密度函数:} 存在非负函数$ p(x,y), F(x,y) = \int_{-\infty}^x{\int_{-\infty}^y{p(u,v)\mathrm{d}u\mathrm{d}v}}$

\textbf{边际密度函数:}当联合密度函数存在时,每个边际分布函数对应的密度

$ p_X(x) = \int_{\mathbb{R}}p(x,y)\mathrm{d}y,p_Y(y) = \int_{\mathbb{R}}{p(x,y)\mathrm{d}x}$

在$ F(x,y)$偏导数存在处有$ p(x,y) = \dfrac{\partial^2}{\partial x\partial y}F(x,y)$

\textbf{独立}:$ F(x_1,\cdots x_n)=\prod{F_i(x_i)}\Leftrightarrow \begin{cases} P(X_1=x_1,\cdots X_n=x_n)=\prod{P(X_i=x_i)}\\ p(x_1,\cdots x_n)=\prod{p_i(x_i)}\end{cases}$

\subsection{Common Multivariate Distributions}
\begin{enumerate}
  \item \textbf{多项分布}

    $ n$次实验,每次有$ r$种可能结果,概率分别为$ p_1\cdots p_r$.以各种结果出现的次数作为随机变量,有
    $ P(X_1=n_1, \cdots X_r = n_r) = \dfrac{n!}{\Oldprod{n_i!}}\prod{p_i^{n_i}}$

    \setlength{\mathindent}{-3cm}
    概率是多项式$ (\sum p_i x_i)^n$展开式中的系数.
    \begin{align*}
    P(X_1=n_1)&=\dfrac{n!}{n_1!(n-n_1!)}p_1^{n_1}\sum_{n_2+\cdots +n_r=n-n_1}{\dfrac{(n-n_1)!}{\Oldprod_{i=2}^r{n_i!}}{\Oldprod_{i=2}^r{p_i^{n_i}}}} \\
              &=\dfrac{n!}{n_1!(n-n_1!)}p_1^{n_1}(\sum_{i=2}^r{p_i})^{n-n_1}\\
              &=\dfrac{n!}{n_1!(n-n_1!)}p_1^{n_1}(1-p_1)^{n-n_1}. 边缘分布为二项分布.
    \end{align*}

  \item \textbf{多维超几何分布}

    $ N $个球中, $i$号球有$ N_i$个,任取$ n$个,其中各号球的个数作为随机变量,有
    $ P(X_1=n_1,\cdots X_r=n_r)=\dfrac{\prod{{{N_i}\choose{n_i}} }}{{N\choose{n}} }$

  \item \textbf{多维均匀分布} $ U(D)$

    $ D $为$\mathbb{R}^n$的有界可测子集,测度为$ S_D, p(x_1\cdots x_n)=\dfrac{1}{S_D}I_D(x_1\cdots x_n)$

  \item \textbf{二维指数分布}

    $ F(x,y) = (1-e^{-x}-e^{-y}+e^{-x-y-\lambda xy})I_{\mathbb{R^+}\times \mathbb{R^+}}(x,y)$

    边际分布为$ Exp(1)$

  \item \textbf{二维正态分布} $ N(\mu_1, \mu_2,\sigma_1,\sigma_2,\rho), (|\rho| < 1)$

    $ p(x,y) = \dfrac{1}{2\pi\sigma_1\sigma_2\sqrt{1-\rho^2}}e^{-\frac{Q}{2(1-\rho^2)}} $

    其中$Q = \dfrac{(x-\mu_1)^2}{\sigma_1^2}-2\rho\dfrac{(x-\mu_1)(y-\mu_2)}{\sigma_1\sigma_2}+\dfrac{(y-\mu_2)^2}{\sigma_2^2}$

    由$ |\rho|< 1$可知$ Q$为正定二次型.
    \setlength{\mathindent}{-3cm}
    \begin{align*}
    \textbf{边际分布:} &p_X(x) =  \dfrac{1}{2\pi\sigma_1\sigma_2\sqrt{1-\rho^2}}\int_{\mathbb{R}}{e^{-\frac{Q}{2(1-\rho^2)}}\mathrm{d}y} \\
      = & \dfrac{1}{2\pi\sigma_1\sigma_2\sqrt{1-\rho^2}}e^{-\frac{(x-\mu_1)^2}{2\sigma_1^2}}\int_{\mathbb{R}}e^{-\frac{T^2}{2}}\mathrm{d}y\quad (T = \rho\dfrac{x-\mu_1}{\sigma_1\sqrt{1-\rho^2}}-\dfrac{y-\mu_2}{\sigma_2\sqrt{1-\rho^2}})\\
      = & \dfrac{1}{2\pi\sigma_1\sigma_2\sqrt{1-\rho^2}}e^{-\frac{(x-\mu_1)^2}{2\sigma_1^2}}\sigma_2\sqrt{1-\rho^2}\int_{\mathbb{R}}{e^{-\frac{t^2}{2}}\mathrm{d}t} \\
      = & \dfrac{1}{\sqrt{2\pi}\sigma_1}e^{-\frac{(x-\mu_1)^2}{2\sigma_1^2}} 为一维正态分布.
    \end{align*}

    设$ (X,Y)\sim N(0,0,1,1,\rho)$

    $则p(x,y) = \dfrac{1}{2\pi\sqrt{1-\rho^2}}e^{-\frac{x^2-2\rho xy+y^2}{2(1-\rho^2)}} = \dfrac{1}{\sqrt{2\pi}}e^{-\frac{y^2}{2}}g(x,y)$

    其中$ z(x) = g(x,y)$为$Z\sim N(\rho y,1-\rho^2)$的密度函数.

    $ \Rightarrow $\textbf{条件正态分布:} (直观解释:考虑$ \rho=0,1$时的情形)
\begin{align*}
    对\quad  (X,Y)\sim N(0,0,1,1,\rho)&\Rightarrow X|Y=y \sim N(\rho y, 1-\rho^2)\\
   & \Rightarrow  \E{X|Y} = \rho Y, \E{X|Y=y} = \rho y
\end{align*}
 \begin{align*}
  &对 \quad (X,Y) \sim N(\mu_1, \mu_2, \sigma_1^2, \sigma_2^2, \rho) \\
  &\Rightarrow  X|Y=y \sim N(\mu_1+\rho\dfrac{\sigma_1}{\sigma_2}(y-\mu_2),
\sigma_1^2(1-\rho^2))&
\end{align*}

\textbf{相关系数:}
    \begin{align*}
     r(X,Y) =& \Cov{X}{Y} = \E{XY}-\Ex{X}\Ex{Y}=\iint_{\mathbb{R}^2}xyp(x,y)\mathrm{d}x\mathrm{d}y\\
    =& \int_{\mathbb{R}}\dfrac{y}{\sqrt{2\pi}}e^{-\frac{y^2}{2}}\int_{\mathbb{R}}xg(x,y)\mathrm{d}x\mathrm{d}y\\
    =& \int_{\mathbb{R}}\Ex{Z}\dfrac{y}{\sqrt{2\pi}}e^{-\frac{y^2}{2}}\mathrm{d}y\\
    =& \rho \Ex{Y^2} = \rho \Var{Y} = \rho
    \end{align*}

    做线性变换可知,任意二维正态分布的相关系数为$ \rho$.

    $ \rho = 0$时易证$ p(x,y) = p_X(x)p_Y(y),$于是,对于联合正态分布,不相关与独立等价.

    最大值期望:$ \E{\max\{ X,Y\}} = \E{\dfrac{X+Y+|X-Y|}{2}} = \sqrt{\dfrac{1-\rho}{\pi}} $

    \textbf{椭圆域内概率:}$ D=\{(x,y): Q \le t \}, 则P((X,Y)\in D) = 1-e^{-\frac{t}{2(1-\rho^2)}}$

    \textbf{证:}先做仿射变换 $ u = \dfrac{x-\mu_1}{\sigma_1}-\rho\dfrac{y-\mu_2}{\sigma_2}, v = \dfrac{y-\mu_2}{\sigma_2}\sqrt{1-\rho^2}, u^2+v^2=Q$

    $ J= \begin{vmatrix} \dfrac{1}{\sigma_1} & 0 \\ -\dfrac{\rho}{\sigma_2} & \dfrac{\sqrt{1-\rho^2}}{\sigma_2} \end{vmatrix} = \dfrac{\sqrt{1-\rho^2}}{\sigma_1\sigma_2}$

    于是,LHS$ =\dfrac{1}{2\pi\sigma_1\sigma_2\sqrt{1-\rho^2}}\iint\limits_{D}e^{-\frac{Q}{2(1-\rho^2)}}\mathrm{d}x\mathrm{d}y = \dfrac{1}{2\pi(1-\rho^2)}\iint\limits_{u^2+v^2\le t}e^{-\frac{u^2+v^2}{2(1-\rho^2)}}\mathrm{d}u\mathrm{d}v$

    再做变换$ u = r\sin\theta, v=r\cos\theta, |J^{-1}| = r,有:$

    LHS = $\dfrac{1}{2\pi(1-\rho^2)}\int_{0}^{2\pi}{\mathrm{d}\theta}\int_{0}^{\sqrt{t}}{re^{-\frac{r^2}{2(1-\rho^2)}}\mathrm{d}r}  = 1-e^{-\frac{t}{2(1-\rho^2)}}$

\end{enumerate}

\subsection{Convolution}
\textbf{和的分布(卷积):}

$P(X+Y=k) = \sum_{i\in \mathbb{Z}}{P(X=i)P(y=k-i)}$

$(X,Y)$的联合密度函数为$ p(x,y)$,有:

$F_{X+Y}(z) = \iint\limits_{x+y\le z}p(x,y)\mathrm{d}x\mathrm{d}y \Rightarrow p_{X+Y}(z)=\int_{\mathbb{R}}p(z-t,t)\mathrm{d}t$

$X,Y$独立时,有: $F_{X+Y}(z) = \iint\limits_{x+y\le z}p_X(x)p_Y(y)\mathrm{d}x\mathrm{d}y $

$\Rightarrow p_{X+Y}(z)= \int_{\mathbb{R}}p_X(z-y)p_Y(y)\mathrm{d}y=\int_{\mathbb{R}}{p_X(x)p_Y(z-x)\mathrm{d}x}$

\begin{enumerate}
  \setlength{\mathindent}{-3cm}
  \item \textbf{Poisson分布} $ P(\lambda_1)*P(\lambda_2) = P(\lambda_1+\lambda_2)$

    $\Leftarrow  \sum_{i=0}^k{\dfrac{\lambda_1^ie^{-\lambda_1}}{i!}\dfrac{\lambda_2^{k-i}e^{-\lambda_2}}{(k-i)!}}= $
      $\dfrac{e^{-\lambda_1-\lambda_2}}{k!}\sum_{i=0}^k{{k\choose{i}} \lambda_1^i\lambda_2^{k-i}}=\dfrac{e^{-\lambda_1-\lambda_2}}{k!}(\lambda_1+\lambda_2)^k$

      (Raikov)独立变量的和服从Poisson分布,则每个都服从Poisson分布.

  \item \textbf{二项分布} $ b(n,p)*b(m,p) = b(m+n,p)$
    \begin{align*}
      \Leftarrow \sum_{i=0}^k{P(X=i)P(Y=k-i)}&=\sum_{i=\max\{ 0,k-m\}}^{\min\{ n,k\}}{{n\choose{i}} p^i(1-p)^{n-i}{m\choose{k-i}} p^{k-i}(1-p)^{m-(k-i)}}\\
        &= p^k(1-p)^{m+n-k}\sum{{n\choose{i}} {m\choose{k-i}} }\\
        &= {{m+n}\choose{k}} p^k(1-p)^{m+n-k}
    \end{align*}
  \item \textbf{Gamma分布} $ \Gamma(\alpha_1,\lambda)*\Gamma(\alpha_2,\lambda) = \Gamma(\alpha_1+\alpha_2, \lambda)$
    \begin{align*}
      \Leftarrow p_{X+Y}(z)&=  \dfrac{\lambda^{\alpha_1+\alpha_2}e^{-\lambda(z-y)}e^{-\lambda y}}{\Gamma(\alpha_1)\Gamma(\alpha_2)}\int_{0}^{z}{(z-y)^{\alpha_1-1}y^{\alpha_2-1}\mathrm{d}y} \\
      &\xlongequal{y=zt}       \dfrac{\lambda^{\alpha_1+\alpha_2}e^{-\lambda z}}{\Gamma(\alpha_1)\Gamma(\alpha_2)}z^{\alpha_1+\alpha_2-1}\int_{0}^{1}{(1-t)^{\alpha_1-1}t^{\alpha_2-1}\mathrm{d}t}\\
      &=\dfrac{\lambda^{\alpha_1+\alpha_2}}{\Gamma(\alpha_1+\alpha_2)}z^{\alpha_2+\alpha_2-1}e^{\lambda z}
    \end{align*}

    $ \Rightarrow $ \textbf{卡方分布} $ \chi^2(m)*\chi^2(n) = \chi^2(m+n)$
\item \textbf{正态分布} $ N(\mu_1,\sigma_1^2)*N(\mu_2,\sigma_2^2) = N(\mu_1+\mu_2,\sigma_1^2+\sigma_2^2)$
  \begin{align*}
    \Leftarrow p_{X+Y}(z) &= \dfrac{1}{2\pi\sigma_1\sigma_2}\int_{\mathbb{R}}{e^{-\frac{Q}{2}}\mathrm{d}y}\quad(Q = \dfrac{(z-y-\mu_1)^2}{\sigma_1^2} + \dfrac{(y-\mu_2)^2}{\sigma_2^2} )\\
      & \xlongequal{对Q配方} \dfrac{1}{2\pi\sigma_1\sigma_2}e^{-\frac{(z-\mu_1-\mu_2)^2}{2(\sigma_1^2+\sigma_2^2)}}\int_{\mathbb{R}}{e^{-\frac{A}{2}(y-T)^2}\mathrm{d}y}, (A = \dfrac{1}{\sigma_1^2} + \dfrac{1}{\sigma_2^2})\\
      & =\dfrac{1}{2\pi\sigma_1\sigma_2}e^{-\frac{(z-\mu_1-\mu_2)^2}{2(\sigma_1^2+\sigma_2^2)}}\sqrt{\dfrac{2\pi}{A}}\\
      &= \dfrac{1}{\sqrt{2\pi(\sigma_1^2+\sigma_2^2)}}e^{-\frac{(z-\mu_1-\mu_2)^2}{2(\sigma_1^2+\sigma_2^2)}}
  \end{align*}
  (Cram\'er)独立变量的和服从正态分布,则每个都服从正态分布.

\end{enumerate}

\subsection{Other Functions on Random Variables}
\begin{description}
  \item[期望:]
    $ \E{g(X,Y)}=\begin{cases}\sum_{i,j}g(x_i,y_j)P(X=x_i,Y=y_j)\\\iint_{\mathbb{R}^2}g(x,y)p(x,y)\mathrm{d}x\mathrm{d}y \end{cases}$

    期望的\textbf{可加性}对任意随机变量成立:

    $
    \E{X+Y}=\int_{\mathbb{R}}x(\int_{\mathbb{R}}p(x,y)dy)dx+\int_{\mathbb{R}}y(\int_{\mathbb{R}}p(x,y)dx)dy
    = \Ex{X}+\Ex{Y}$

  \item[最值:]
    $ Y=\max\{ X_1,\cdots, X_n\}, F_Y(x) = \prod{F_i(x)}$

    $ Z = \min\{ X_1,\cdots ,X_n\}, F_Z(x) = 1-\prod{(1-F_i(x))}$

    当$ X_1,\cdots ,X_n$i.i.d.时,$ p_Y(x)=nF(x)^{n-1}p(x), p_Z(x)=n(1-F(x))^{n-1}p(x)$
  \item[第$ k$小值(次序统计量):]设$ X_1,\cdots ,X_n$i.i.d.,设``第$ k$小值''这个随机变量为$ M_k$

      $ p_{M_k}(x)=n!\dfrac{F(x)^{k-1}}{(k-1)!}\dfrac{[1-F(x)]^{n-k}}{(n-k)!}p(x)$

      $i<j, p_{M_i,M_j}(x,y)=n!\dfrac{F(x)^{i-1}}{(i-1)!}\dfrac{[F(y)-F(x)]^{j-i-1}}{(j-i-1)!}\dfrac{[1-F(y)]^{n-j}}{(n-j)!}p(x)p(y)$

      $ p_{M_1,\cdots ,M_n}(x_1,\cdots x_n)=\begin{cases}n!\prod{p(x_i)},if x_1<\cdots <x_n\\0,else \end{cases}$

      设极差$ R_n = M_n - M_1, $由$ p_{M_1,M_n}$做Jacobi可得$ p_{R_n,M_1}$

      $\Rightarrow F_{R_n}(x) = \int_{\mathbb{R}}np(t)(F(x+t)-F(t))^{n-1}\mathrm{d}t $
  \item[双射:]
   $ \begin{cases}U=g(X,Y)\\V=h(X,Y)\end{cases},p_{U,V}(u,v) = p_{X,Y}(x(u,v),y(u,v))|\dfrac{\partial(x,y)}{\partial(u,v)}|$

  \item[独立积:]$ Z=XY$\hfill

   设$T = Y$,利用二维双射,有 $p_{Z,T}(z,t) = \dfrac{p_X(\frac{z}{t})p_Y(t)}{|t|}$.对$ t$积分即得$ p_Z(z)$

 \item[独立商:]$ Z=\frac{X}{Y}$\hfill

   设$ T=Y,p_{Z,T}(z,t)=p_X(zt)p_Y(t)|t|.$对$ t$积分.

 \item[线性变换:]$ \vec Y = \mathbf{A} \vec X + \vec B$

   $ p_{\vec Y}(\vec x)=p_{\vec X}(\mathbf{A}^{-1}\vec x-\mathbf{A}^{-1}\vec B)|\det \mathbf{A}^{-1}|$

   正态分布的标准化:设$(X,Y)\sim N(\mu_1,\mu_2,\sigma_1^2,\sigma_2^2,\rho) $

   做变换$\begin{pmatrix} X'\\Y'\end{pmatrix}=\begin{pmatrix}\frac{1}{\sigma_1}&0\\0&\frac{1}{\sigma_2}
   \end{pmatrix}\begin{pmatrix}X\\Y\end{pmatrix}-\begin{pmatrix}\frac{\mu_1}{\sigma_1}\\\frac{\mu_2}{\sigma_2}\end{pmatrix}$

   则$(X',Y')\sim N(0,0,1,1,\rho) $

\end{description}

\subsection{Correlation}

对独立的$X_1,\cdots X_n$,显然有$\E{\prod{X_i}}=\prod{\Ex{X_i}}
,\Var{\sum{a_iX_i}}=\sum{a_i^2\Var{X_i}}$

对任意的$ X,Y,有\Var{X\pm Y}=\Var{X}+\Var{Y}\pm2\E{(X-\Ex{X})(Y-\Ex{Y})}$.

记协方差$ \Cov{X}{Y} = \E{(X-\Ex{X})(Y-\Ex{Y})} = \E{XY}-\Ex{X}\Ex{Y}(若\E{XY}存在)$

记相关系数$ r(X,Y)=\dfrac{\Cov{X}{Y}}{\sigma(X)\sigma(Y)}$

独立$ \Rightarrow \E{XY} = \Ex{X}\Ex{Y}\Leftrightarrow r(X,Y) = 0\Leftrightarrow 不相关$

$ r(X,Y) = \pm 1\Leftrightarrow X,Y几乎处处线性相关\Leftrightarrow P(Y=aX+b)=1$
\\

$ \E{XY}$是对角线上非负的对称双线性函数,可作为随机变量的内积的定义.

因此,由Cauchy's Inequality:

$\Cov{X}{Y}^2=|\langle X-\Ex{X}, Y-\Ex{Y}\rangle |^2 \le
|X-\Ex{X}||Y-\Ex{Y}|=\Var{X}\Var{Y} $

$ \Leftrightarrow |r(X,Y)|\le 1. $

$  r$可看做$X-\Ex{X}, Y-\Ex{Y}$的夹角余弦,$\Ex{X}$看做$ X$在常数子空间上的投影.

将对角线上非负的对称双线性函数$ \Cov{X}{Y}$看做内积,同样可得$ |r(X,Y)|\le 1$.
\\

\textbf{线性回归}:求使$ \E{X-(aY+b)}^2$最小的$(a,b)$:

考虑内积空间中垂直最小, $\E{(X-(\hat{a}Y+\hat{b})(aY+b))} = 0 $

$\Rightarrow  \E{X-(\hat{a}Y+\hat{b})}=0,\E{Y(X-(\hat{a}Y+\hat{b}))}=0. $

解得$\hat{a}=\dfrac{\Cov{X}{Y}}{\Var{Y}},\hat{b}=\Ex{X}-\hat{a}\Ex{Y}.\quad
\hat{X} = \dfrac{\Cov{X}{Y}}{\Var{Y}}(Y-\Ex{Y})+\Ex{X}$
\begin{align*}
  此时误差 \E{X-\hat{X}}^2 &=
  \E{(X-\Ex{X})-\dfrac{\Cov{X}{Y}}{\Var{Y}}(Y-\Ex{Y})}^2 \\
  &= \Var{X}-\dfrac{\Cov{X}{Y}^2}{\Var{Y}}\\
  &=\Var{X}[1-r(X,Y)^2]
\end{align*}
\label{sec:linear}

\textbf{协方差矩阵} $ \vec{X} = (X_1,\cdots ,X_n)',每个分量期望都存在,则随机向量\vec{X}有期望$.

定义协方差矩阵为$\Cov{\vec{X}}= (\Cov{X_i}{X_j})_{n\times
  n}=\E{(\vec{X}-\Ex{\vec{X}})(\vec{X}-\Ex{\vec{X}})'}$

$\Var{\sum{X_i}}=\sum{\Var{X_i}} +\sum_{1\le<i<j\le n}{\Cov{X_i}{X_j}}$为矩阵中所有元素和.

$ \Cov{\vec{X}}$为\textbf{半正定对称阵}:任取向量$ \vec{\alpha}$,
\begin{align*}
\vec{\alpha'}\Cov{\vec{X}}\vec{\alpha} =& \sum_{i=1}^{n}\sum_{j=1}^{n}{\alpha_i\alpha_j\Cov{X_i}{X_j}}\\
=&\sum_{i=1}^n\sum_{j=1}^n{\E{\alpha_i(X_i-\Ex{X_i})\alpha_j(X_j-\Ex{X_j})}}\\
=&\E{\sum_{i=1}^n\sum_{j=1}^n{\alpha_i(X_i-\Ex{X_i})\alpha_j(X_j-\Ex{X_j})}}\\
=&\E{\sum_{i=1}^n{\alpha_i(X_i-\Ex{X_i})}\sum_{j=1}^n{\alpha_j(X_j-\Ex{X_j})}}\\
=&\E{\sum_{i=1}^n{\alpha_i(X_i-\Ex{X_i})}}^2\ge 0
\end{align*}

\subsection{Gauss Distribution}
$ \vec{Z} = (Z_1,\cdots Z_n)^{T}, Z_i\sim N(0,1)相互独立,A=(a_{ij})_{m\times n}, \vec{\mu} = (\mu_1,\cdots ,\mu_m)^{T},则\vec{X} = A\vec{Z}+\vec{\mu}
服从m$维\textbf{Gauss分布},记$\vec{X}\sim N(\Ex{\vec{X}},\Cov{\vec{X}}) = N(\vec{\mu}, \Sigma)$

$ \Ex{\vec{X}} = \vec{\mu}, \Var{X_i} = \sum_{k}a_{ik}^2$
\begin{align*}
 \Cov{Z_k}{Z_l} = \delta_{kl}&\Rightarrow
 \Cov{X_i}{X_j}=\sum_{k,l}a_{ik}a_{jl}\Cov{Z_k}{Z_l} = \sum_{k}a_{ik}a_{jk}\\
 &\Rightarrow \Sigma = \Cov{\vec{X}}=AA^T
\end{align*}

  $ \vec{X}服从m维$\textbf{正态分布}$\Leftrightarrow r(A)=r(AA^T)=r(\Cov{\vec{X}})=m\Leftrightarrow \Cov{\vec{X}}正定$

$ \varphi(\vec{t})=e^{i\vec{t}^T\vec{\mu}-\frac{1}{2}\vec{t}^T\Sigma\vec{t}},$且对于每个非负实对称阵$ \Sigma$,此函数都是Gauss分布.

$ \bullet $由特征函数容易证明,Gauss分布做线性变换后仍为Gauss分布.

$ p(\vec{x})=\dfrac{1}{(2\pi)^{\frac{n}{2}}\sqrt{|\Sigma|}}e^{-\dfrac{(\vec{x}-\vec{\mu})^T\Sigma^{-1}(\vec{x}-\vec{\mu})}{2}}$
\vspace{1cm}

\textbf{Matrix Tricks}:

设$n维随机向量 \vec{X}\sim N(0,I), $
取\textbf{正交阵}$A,使其首行为(\dfrac{1}{\sqrt{n}},\cdots ,\dfrac{1}{\sqrt{n}}),$
则显然$ \vec{Y} = A\vec{X}\sim N(0,I)$
\begin{flalign*}
  \bar{X} = &\dfrac{\Oldsum{X_i}}{n}=\dfrac{Y_1}{\sqrt{n}}&\\
  (n-1)S^2 =& \sum(X_i-\bar{X})^2=\sum{X_i^2}-n\bar{X}^2=X^TX-Y_1^2&\\
  =& Y^TAA^{-1}Y-Y_1^2=Y_2^2+\cdots +Y_n^2\sim \chi^2(n-1) &\\
  \Rightarrow & \bar{X}与(n-1)S^2独立&\\
\end{flalign*}

设$ n$个独立同分布正态样本$ X_i \sim N(\mu, \sigma^2),$同上设$ A,$
并特别地,构造$ A$ 的第$ k\ge 2$行为$ (\dfrac{-1}{\sqrt{(k-1)k}},
\dfrac{-1}{\sqrt{(k-1)k}},\cdots ,\dfrac{k}{\sqrt{(k-1)k}},0,\cdots ,0)$,使得行和
为0.

则有$ \Ex{Y_1} = \sqrt{n}\mu, \Ex{Y_k} = 0(k\ge 2), \Var{Y_k} = \sigma^2(k\ge 1)$

\subsection{Conditional Distribution}
设$ A$为事件,$ F_{X|A}(x)=P(X\le x|A) = \int_{-\infty}^{x}p_{X|A}(u)\mathrm{d}u$

当$ Y$为随机变量时,根据新的下标重载函数$ F$:

$ F_{X|Y}(x|y) = P(X\le x|Y=y)=\int_{-\infty}^x{\dfrac{p_{X,Y}(u,y)}{p_Y(y)}\mathrm{d}u}=\int_{-\infty}^x{ p_{X|Y}(u|y)}\mathrm{d}u$

\begin{description}
  \item [逆乘法公式:]\hfill\\
    若$p_{X,Y}(x,y) = g(y)h(x,y),且\forall y, \int_{R}h(x,y)\mathrm{d}x=1$

    $则g(y) = p_Y(y), h(x,y) = p_{X|Y}(x|y)$

  \item [连续全概率公式:] $ p_X(x) = \int_{\mathbb{R}}{p_Y(y)p_{X|Y}(x|y)\mathrm{d}y}$

  \item[连续Bayes公式:]$ p_{X|Y}(x|y) = \dfrac{p_X(x)p_{Y|X}(y|x)}{\int_{\mathbb{R}}{p_X(x)p_{Y|X}(y|x)}\mathrm{d}x}$

  \item[条件期望:]\hfill\\
    $ \E{X|Y=y} = \int_{\mathbb{R}}{xp_{X|Y}(x|y)\mathrm{d}x}$, 其值是关于$ Y$的
    随机变量,记做$\E{X|Y} $

    $\quad \E{g(X)h(Y)|Y=y} = \E{g(X)h(y)|Y=y} = h(y) \E{g(X)|Y=y}$

    $\Rightarrow \E{g(X)h(Y)|Y} = h(Y)\E{g(X)|Y}$

\item[重期望公式:]\hfill
  \begin{flalign*}
    \Ex{X} &= \int_{\mathbb{R}}\int_{\mathbb{R}}{xp(x,y)\mathrm{d}x\mathrm{d}y} =\int_{\mathbb{R}}\int_{\mathbb{R}}xp(x|y)p_Y(y)\mathrm{d}x\mathrm{d}y &\\
    &=\int_{\mathbb{R}}{\E{X|Y=y}p_Y(y)\mathrm{d}y} = \E{\E{X|Y}}&\\
    \Ex{X} &= \sum_{j}\E{X|Y=y_j}P(Y=y_j) 用于很多趣题.&\\
    推论:&随机个随机变量和.设 X_1\cdots \text{i.i.d},则有
    \E{\sum_{i=1}^N{X_i}}=\Ex{X}\Ex{N}&
  \end{flalign*}
\item [条件期望是最佳平方逼近:] $ \Ex{X^2}<\infty 时,\E{X-\E{X|Y}}^2 =\min \E{X-\varphi(Y)}^2$ \hfill\\
  \proof:由$ \E{\E{X|Y}}^2\le \E{\E{X^2|Y}} =\Ex{X^2}$知存在性.

  考虑期望在线性空间中的点积意义,需证$ \E{(X-\E{X|Y})\varphi(Y)}=0$.

  LHS = $ \E{\E{(X-\E{X|Y})\varphi(Y)|Y}} = \E{\varphi(Y)\E{X-\E{X|Y}|Y}} =
  \E{\varphi(Y)(\E{X|Y} - \E{\E{X|Y}|Y})} = 0$

\end{description}



% File: limit.tex
% Date: Sat Dec 15 15:17:10 2012 +0800
% Author: Yuxin Wu <ppwwyyxxc@gmail.com>
\section{}
\subsection{Convergence}
依概率收敛:$\forall \varepsilon>0, P(|X_n-X|\ge \varepsilon)\to 0 (n\to \infty),\Leftrightarrow  X_n\overset{P}{\rightarrow}X$

弱收敛:$ \forall F的连续点x,\lim\limits_{n\to \infty}F_n(x)=F(x)\Leftrightarrow F_n(x)\overset{W}{\rightarrow}F(x)$

依分布收敛:$ X_n\overset{L}{\rightarrow}X$

\subsection{Law of Large Numbers}
\begin{enumerate}
    \item 弱大数定律:$\dfrac{\Oldsum{X_i}}{n}\overset{P}{\rightarrow}\mu$

      意义:用平均值作为期望是合理的,即使不知道其分布

      \textbf{Chebyshev}: $X_n $两两不相关,方差一致有界

      \textbf{Markov:}$ X_n$两两不相关,$ \lim\limits_{n\to\infty}\dfrac{1}{n^2} Var(\sum{X_i})= 0(Markov条件)$

      \textbf{Bernstein}:只需$ X_n$渐进不相关($ \lim\limits_{|k-l|\to \infty}\Cov(X_k,X_l)=0$),方差一致有界

      \textbf{Khintchine}:只需$ X_n$i.i.d.,期望存在

    推论:\textbf{J.Bernoulli}:
      $ n$  次试验中$ A$发生的次数$ S_n,\dfrac{S_n}{n}\overset{P}{\rightarrow}p$

    意义:概率是频率的极限

    \item 强大数定律:$ P(\lim\limits_{n\to\infty}\dfrac{\Oldsum{X_i}}{n}=\dfrac{\Oldsum{EX_i}}{n}) = 1$

      \textbf{Borel:}$ X_n$i.i.d.,$ EX^4<+\infty$

      \textbf{Kolmogorov:}$ X_n$独立,$ \sum{\dfrac{Var(X_i)}{i^2}}<+\infty$

\end{enumerate}

\subsection{Central Limit Theorem}
\begin{enumerate}
    \item \textbf{Lindeberg-Levy:}$ X_n$i.i.d.,$ Y_n=\dfrac{\Oldsum{X_i}-n\mu}{\sigma\sqrt{n}}\overset{L}{\rightarrow}N(0,1)$

    \item \textbf{De Moivre-Laplace:}$ n次实验中A发生了S_n次,Y_n=\dfrac{S_n-np}{\sqrt{np(1-p)}}\overset{L}{\rightarrow}N(0,1)$
    \item \textbf{Lindeberg/Lyapunov:}$ X_n独立,设B_n=\sqrt{\Oldsum{\sigma_i^2}},若满足$Lindeberg条件:
      \[ \forall t>0,\lim\limits_{n\to\infty}\dfrac{1}{B_n^2}\sum_{i=1}^n{\int_{|x-\mu_i|>tB_n}{(x-\mu_i)^2p_i(x)\mathrm{d}x}=0} \]
      或Lyapunov条件(弱于Lindeberg):
      \[ \exists t>0,\lim\limits_{n\to\infty}[\sum_{i=1}^n{Var(X_i)}]^{-t}\sum_{i=1}^nE(|X_i-\mu_i|^{2+t})=0 \]
      则$ \dfrac{1}{B_n}\sum{(X_i-\mu_i)}\overset{L}{\rightarrow}N(0,1)$.

\end{enumerate}



\end{document}

