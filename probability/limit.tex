% File: limit.tex
% Date: Tue Jan 08 19:47:04 2013 +0800
% Author: Yuxin Wu <ppwwyyxxc@gmail.com>
\section{Analytic Topics}
\subsection{Estimation}
\begin{flalign*}
	\textbf{(Chebyshev)} & \forall \varepsilon > 0, P(|X-\Ex{X}| \ge \varepsilon)          \\
	=                    & \int_{|x- \Ex{X}| \ge \varepsilon}{p(x)\mathrm{d}x}
    \le \int_{|x-\Ex{X}| \ge \varepsilon}{\dfrac{(x-\Ex{X})^2}{\varepsilon^2}p(x)\mathrm{d}x} \\
	\le                  &
    \dfrac{\int_{\mathbb{R}}{(x-\mu)^2}p(x)\mathrm{d}x}{\varepsilon^2} =
    \dfrac{\sigma^2}{\varepsilon^2}\\
	统计意义:&与均值的距离远近对概率的限定.
\end{flalign*}

\subsection{Convergence}
依概率收敛:$\forall \varepsilon>0, P(|X_n-X|\ge \varepsilon)\to 0 (n\to \infty),\Leftrightarrow  X_n\overset{P}{\rightarrow}X$

弱收敛:$ \forall F的连续点x,\lim\limits_{n\to \infty}F_n(x)=F(x)\Leftrightarrow F_n(x)\overset{W}{\rightarrow}F(x)$

依分布收敛:$ X_n\overset{L}{\rightarrow}X$

\subsection{Law of Large Numbers}
\begin{enumerate}
    \item 弱大数定律:$\dfrac{\Oldsum{X_i}}{n}\overset{P}{\rightarrow}\mu$

      意义:用平均值作为期望是合理的,即使不知道其分布

      \textbf{Chebyshev}: $X_n $两两不相关,方差一致有界

      \textbf{Markov:}$ X_n$两两不相关,$ \lim\limits_{n\to\infty}\dfrac{1}{n^2}
      \Var{\sum{X_i}}= 0(Markov条件)$

      \textbf{Bernstein}:只需$ X_n$渐进不相关($ \lim\limits_{|k-l|\to \infty}\Cov(X_k,X_l)=0$),方差一致有界

      \textbf{Khintchine}:只需$ X_n$i.i.d.,期望存在

    推论:\textbf{J.Bernoulli}:
      $ n$  次试验中$ A$发生的次数$ S_n,\dfrac{S_n}{n}\overset{P}{\rightarrow}p$

    意义:概率是频率的极限

    \item 强大数定律:$
      P(\lim\limits_{n\to\infty}\dfrac{\Oldsum{X_i}}{n}=\dfrac{\Oldsum{\Ex{X_i}}}{n}) = 1$

      \textbf{Borel:}$ X_n$i.i.d.,$ \Ex{X^4}<+\infty$

      \textbf{Kolmogorov:}$ X_n$独立,$ \sum{\dfrac{\Var{X_i}}{i^2}}<+\infty$

\end{enumerate}

\subsection{Central Limit Theorem}
\begin{enumerate}
    \item \textbf{Lindeberg-Levy:}$ X_n$i.i.d.,$ Y_n=\dfrac{\Oldsum{X_i}-n\mu}{\sigma\sqrt{n}}\overset{L}{\rightarrow}N(0,1)$

    \item \textbf{De Moivre-Laplace:}$ n次实验中A发生了S_n次,Y_n=\dfrac{S_n-np}{\sqrt{np(1-p)}}\overset{L}{\rightarrow}N(0,1)$
    \item \textbf{Lindeberg/Lyapunov:}$ X_n独立,设B_n=\sqrt{\Oldsum{\sigma_i^2}},若满足$Lindeberg条件:
      \[ \forall t>0,\lim\limits_{n\to\infty}\dfrac{1}{B_n^2}\sum_{i=1}^n{\int_{|x-\mu_i|>tB_n}{(x-\mu_i)^2p_i(x)\mathrm{d}x}=0} \]
      或Lyapunov条件(弱于Lindeberg):
      \[ \exists
        t>0,\lim\limits_{n\to\infty}(\sum_{i=1}^n{\Var{X_i}})^{-t}\sum_{i=1}^n\E{|X_i-\mu_i|^{2+t}}=0 \]
      则$ \dfrac{1}{B_n}\sum{(X_i-\mu_i)}\overset{L}{\rightarrow}N(0,1)$.

\end{enumerate}

