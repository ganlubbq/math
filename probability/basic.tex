% File: basic.tex
% Date: Sun Nov 04 15:19:39 2012 +0800
% Author: Yuxin Wu <ppwwyyxxc@gmail.com>

\section{}
\subsection{Events}
一次随机试验中每一种可能的结果称为一个{\bf 基本事件}或{\bf 样本点$ \omega$},所有基本事
件的全体为该试验的样本空间$ \Omega$

同一试验的样本空间可能不唯一,因为观察结果的角度不同.对扔两次色子,$ \Omega_1 =
\{++,+-,--,-+\}, \Omega_2 = \{\texttt{两正,两负,一正一负}\}$

至多可数的样本空间称为离散样本空间,不可数称为连续样本空间.

$ \Omega$的可测子集$ A$称为事件.对结果$ \omega \in A$,则称事件A发生了.

$ A \subset B \Rightarrow $A发生了B必发生.

Morgan律:$ (\cup A_i)^c = \cap A_i^c$

\subsection{Probability Space}

概率空间$ (\Omega, \mathcal{F}, P)$:

$\Omega$ 是全体可能结果组成的集合.$ \mathcal{F}$是全体可观测事件组成的事件族.$ P:\mathcal{F} \rightarrow [0,1]$是求事件的概率的运算.

当$ \mathcal{F}$满足以下条件时,称其为{\bf $ \sigma -$代数}:
\begin{enumerate}
	\item $ \Omega \in \mathcal{F}$
	\item $ A\in \mathcal{F} \Rightarrow A^c \in \mathcal{F}$
	\item 可数并:$ A_1 \cdots \in \mathcal{F} \Rightarrow  \cup_{i = 1}^{\infty}{A_i}
		\in \mathcal{F}$
\end{enumerate}
事实上,由可数并,可推出有限并,可数交,有限交$ \in \mathcal{F}$.

当$ \Omega$为至多可数集时,总可取$ \Omega$的所有子集族作为$ \mathcal{F}$.
当$ \Omega$不可数时,取这样的$ \mathcal{F}$会造成数学上的困难,因此只取感兴趣的,
可以知道概率的事件的最小$ \sigma$-代数.
\\

{\bf 概率的定义:}对每个事件$ A\in \mathcal{F},$定义实数$ P(A)$,满足以下条件:
\begin{enumerate}
	\item 非负性:$ P(A) \ge 0$
	\item 规范性: $ P(\Omega) = 1$
	\item 可数可加性:

		对两两互不相容的事件$ A_1 \cdots \in \mathcal{F}, P(\cup{A_n}) =
		\sum{P(A_n)}$
\end{enumerate}

试验的样本空间,事件域($ \sigma$代数)及定义在其上的概率构成的三元组$ (\Omega,
\mathcal{F}, P)$称为描述一个随机试验的{\bf 概率空间}.
\subsection{Properties of Probability}

事件序列的极限定义:
$ \overline {\lim \limits_{n \to \infty}}A_n  = \bigcap_{n
=1}^{\infty}\bigcup_{k=n}^{\infty}A_k $(当且仅当有无穷个$ A_n$发生)

$ \mathop{\underline \lim} \limits_{n \to \infty} A_n = \bigcup_{n
=1}^{\infty}\bigcap_{k=n}^{\infty}A_k$(当且仅当至多有有限个$ A_n$不发生)

当上下极限相等时(如对于单调事件序列),称为序列$ A_n$的极限.
\\

利用可数可加,可得到如下结论:
\begin{enumerate}
	\item $ P(\empty) = 0:
		P(\Omega) = P(\Omega \cup \empty \cup \empty \cdots) = P(\Omega) +
		P(\empty) + \cdots$
	\item 有限可加
	\item 求逆:$ P(A) + P(A^c) = 1$
	\item Jordan公式(容斥),归纳证明
	\item $ P(A - B) = P(A) - P(A\cap B)$,特别地,$ B \subset A \Rightarrow
		P(B) \le P(A)$
	\item 下连续性:设$ A_i$单调增($ A_1 \subset A_2 \subset \cdots$),则$
		P(\lim \limits_{n \to \infty }{A_n}) = \lim \limits_{n \to
		\infty}{P(A_n)}$.

		$ P(\cup A_n) = P(A_1) + \sum_{i = 1}^{\infty}{P(A_{i+1} - A_i)} =
		P(A_1) = \lim \limits_{n \to \infty}\sum_{i = 1}^{n-1}{[P(A_{i+1} -
		P(A_i))]} = \lim \limits_{n \to \infty}P(A_n)$

	\item 上连续性:设$ A_i$单调减,则$ P(\lim \limits_{n \to \infty} A_n ) =
		\lim \limits_{n \to \infty} P(A_n)$

		$ 1 - \lim \limits_{n \to \infty}P(A_n) = \lim \limits_{n \to
		\infty}P(A_n^c)=P(\bigcup_{n=1}^{\infty}A_n^c) =
		P((\bigcap_{n=1}^{\infty}A_n)^c) = 1-P(\bigcap_{n=1}^{\infty}A_n)$

		概率的上下连续性等价,统称为连续性.

	\item	有限可加+下连续$ \Leftrightarrow $可数可加.

		由下连续性,\[  P(\bigcup_{n=1}^{\infty}A_n) =
		P(\bigcup_{n=1}^{\infty}F_n) \mathop{=}\limits_{\texttt{下连续}}\lim\limits_{n \to \infty}P(F_n)=
		\lim\limits_{n\to\infty}P(\bigcup_{i=1}^nA_i)
		\mathop{=}\limits_{\texttt{有限可加}}
		\lim\limits_{n\to\infty}\sum_{i=1}^n{P(A_i)}\mathop{=} \limits_{\texttt{收敛}}
	\sum_{i=1}^{\infty}P(A_i)\]

\item 推广可数可加:
	$ A_1,A_2\cdots $满足$ P(A_iA_j)=0$(弱于互斥),则$ P(A) = \sum{P(A_n)}$
\end{enumerate}

\subsection{Classical Definitions of Probability}

{\bf 古典概型}: 基本事件只有有限个且概率相同.

掷硬币$ n$次,取每种排列为基本事件,即为古典概型:
\[  P(\texttt{首次正面出现在k次}=\dfrac{1}{2^k})\]

掷硬币直到出现正面为止,基本事件$ \omega_k$为``首次正面出现在第k次'',则有无穷个基本事件,且
概率不同.
利用可数可加性,$ \omega_{\infty} = 0$,但不是不可能事件.

一般地,对于至多可数集合$ \Omega$,每个基本事件的概率都可求出时,$ \forall A
\subset \Omega, P(A) = \sum_{\omega \in A}P(\omega)$

若无限抛掷硬币,将排列作为基本事件,则有不可数个基本事件,此时若考虑等可能分析,则
每个基本事件概率为0.无法求出某个事件的概率(因为不可数个实数的和没有意义).

{\bf 几何概型}:随机现象的 样本空间充满某个可测区域,且任一点落在度量相同的子区域
内是等可能的.

{\bf Buffon投针的分析做法}:设针中点与最近平行线距$ x\in [0, \dfrac{d}{2}]$,与
直线成角$ \varphi\in[0,\pi]$,在上区域中求$ x \le \dfrac{l}{2}\sin{\varphi}$
部分的概率. $ P(A) =
\dfrac{\int_{0}^{\pi}{\dfrac{l}{2}\sin{\varphi}\mathrm{d}{\varphi}}}{\dfrac{d}{2}\pi}
= \dfrac{2l}{d\pi}$

古典/几何概型的另一个问题:Bertrand悖论--圆内一弦长度超过正三角形边长的概率由三
种解释.

原因:当可能结果有无穷个时,难以规定``等可能''这一概念,因此概率空间被模糊定义了.

\subsection{Conditional Probability}
概率空间$( \Omega, \mathcal{F}, P)$中$ P(B) > 0$.定义$ P_B(A) = P(A | B) = \dfrac{P(A\cap B)}{P(B)}$,
则可证$ (\Omega, \mathcal{F}, P_B)$也是概率空间.

  {\bf 乘法公式:}$ P(\bigcap_{i=1}^{n-1}A_i)>0$ (使得条件概率有意义)时,由定义归纳可得

\[ \Rightarrow P(\bigcap_{i=1}^nA_i) = \prod_{i=1}^n{P(A_i | \bigcap_{j=1}^{i-1}A_j)}\]

{\bf 可靠性函数与风险率:}设前$ t$时刻正常,$[t, t+\Delta t]$时段失效的概率为$ \lambda(t)\Delta t + o(\Delta t)$,求
设备在$ (0,t)$上无故障的概率.

设$ A_t$表示设备在$ (0,t)$内正常,$ P(\overline{A_{t+\Delta t}} | A_t) = \lambda(t)\Delta t + o(\Delta t)$.
\[ p(t + \Delta t) = P(A_t)P(A_{t+\Delta t}|A_t) = p(t)[1 - \lambda(t)\Delta t + o(\Delta t)]\Rightarrow \dfrac{dp(t)}{dt}=-\lambda(t)p(t)\]

注意到$ p(0)=1,$有$ p(t) = e^{-\Oldint_0^t{\lambda(s)ds}}$
\\

{\bf 全概率公式:}设$ B_1,B_2\cdots$为样本空间$ \Omega$的一个正划分,则
\[ P(A) =\sum_{i=1}^{\infty}{P(AB_i)}= \sum_{i=1}^{\infty}{P(B_i)P(A|B_i)}\]

{\bf 赌徒输光:}两人各有赌资$ i, n-i$,每次赌博胜者拿走对方1元,胜率分别为$ p,1-p$.

设$ A_i$表示甲有$ i$元,最终破产.$ B$表示某次甲胜,$ P(B) = p$,则有$ P(A_i|B)=P(A_{i+1}),P(A_i|\overline{B})=P(A_{i-1})$

于是$ P(A_i)=pP(A_{i+1}) + (1-p)P(A_{i-1})$,边界$ P(A_0)=1,P(A_n)=0$.

\[  P(A_i)=\left\{ \begin{array}{lc} 1-\dfrac{1-r^i}{1-r^n}  & p \ne \dfrac{1}{2} \\ 1-\dfrac{i}{n}& p = \dfrac{1}{2} \end{array}  \right.r = \dfrac{1-p}{p} \]

对赌场$ (n \to \infty)$,甲最终会输光的概率为$ P(A_i) = \min\{1,r^i\}$
\\

{\bf Polya 模型}:从黑球,红球中任取若干次.取出的红球与黑球个数确定的情形下,概率是否与次序相关.

若放回抽样,结果不影响下次,故概率相等.

若不放回抽样,前次结果影响后次,但概率仍与次序无关.

若放回若干同色球(传染病模型),每次取出会增加下次取出同色球的概率. 但结果与次序无关.

若放回若干异色球(安全模型),结果才与次序有关.
\\

{\bf 敏感问题问卷调查:}在问卷上要求每个人准备一枚硬币,对于指定的隐私题目,请填写人投掷一次硬币:如果正
面朝上,则如实填写个人的真实情况;如果反面朝上,那么就再投掷一次硬币,正面就填"是",反面就填"否".当然,若
第一次投掷硬币为正的话,填写人完全可以假装再投一次硬币来掩人耳目.

假设回收后有效问卷有$M$份,其中该问题答"是"的有$N$个人.如实填写了该问题的人平均有$\dfrac{M}{2}$个;在另
外$ \dfrac{M}{2}$人中,平均有$ \dfrac{M}{4}$人答的"是".因此,我们所需要的最终结果应该为$ \dfrac{(N-M)/4}{M/2} $
\\

{\bf Bayes:}$ P(B_i|A) = \dfrac{P(B_i)P(A|B_i)}{P(A)} = \dfrac{P(B_i)P(A|B_i)}{\sum_{t=1}^n{P(B_t)P(A|B_t)}}$

\subsection{Independence}
{\bf 定义:}$ P(AB)=P(A)P(B)$. 实际中以经验判断.

$ A,B$独立$ \Rightarrow A与\mathcal{F}_B$中任一事件独立.

多个事件相互独立:直观想法--$ A与\mathcal{F}_{B,C\cdots}$中任一事件独立.
\\

{\bf 定义}:其中任意$ k$个事件的交的概率等于概率的乘积.

无穷个事件相互独立:任意有限个事件相互独立.

$ A_1\cdots A_n$ 相互独立,则任意对其分组,各组事件分别产生的事件域相互独立.

相关系数:$ r(A,B)=\dfrac{P(AB)- P(A)P(B)}{\sqrt{P(A)[1-P(A)]P(B)[1-P(B)]}}$

$ \left \{\begin{matrix}-P(A)P(B)\\  -[1-P(A)][1-P(B)]\end{matrix} \right . \le P(AB)-P(A)P(B) \le \left \{ \begin{matrix} P(A)[1-P(B)]\\ P(B)[1-P(A)]\end{matrix}\right .
			\Rightarrow  |r(A,B)| \le 1$

$ r(A,B)=1\Leftrightarrow P(A)=P(AB)=P(B)$

$ r(A,B)>0\Leftrightarrow P(A|B)>P(A)\Leftrightarrow P(B|A)>P(B)$
