% File: rv.tex
% Date: Sun Nov 11 19:46:22 2012 +0800
% Author: Yuxin Wu <ppwwyyxxc@gmail.com>
\section{}
\subsection{Random Variables}
样本空间$ \Omega \rightarrow \mathbb{R}$的函数$ X = X(\omega)$称为\textbf{随机变量}.值域有限或可列称为离散随机变量,值域充满数
轴上的某个区间,称为连续随机变量.记$ F(x) = P(X \le x)$为$ X$的\textbf{分布函数}.

显然,$ F$是$ (-\infty, \infty)$的单调不减函数,有界,于是各点有左右极限,且无穷处有极限.

\begin{equation*}
	\begin{split}
		F(b) - F(c + 0) & = F(b) - \lim \limits_{a \to c^{+}} F(a) =\lim \limits_{a \to c^{+}}P(a < X \le b) \\
		                & = P(\bigcup_{a \to c^{+}}\{a < X \le b\}) = P(c < X \le b)                         \\
		F(d - 0) - F(a) & = P(a < X < d)
	\end{split}
\end{equation*}

即$ F(c + 0) = F(c), F(d - 0) = P(X<d), 且可得F(-\infty) = 0, F(\infty) = 1$.

满足以上性质的函数$ F$必定为某随机变量的分布函数.

任意Borel集$ B\subset \mathbb{R}, P(X \in B) 可由F$ 计算得到.
特别的,$ P(X=x) = F(x) - F(x - 0),若F在x连续,则P(X=x) = 0$
\\

\textbf{$p$-分位数:}满足$ P(X \le x) \ge p, P(X <x) \le p 的x. $

对连续随机变量,等价定义$F(x) = p$的点为$p$-(下侧)分位数.

$ p =\dfrac{1}{2}$时称为\textbf{中位数}.$\Leftrightarrow \dfrac{1}{2} \le F(x) \le \dfrac{1}{2} + P(X=x) $

中位数的统计意义:使得$E|X-a|$最小的$a$
\\

存在非负可积函数,使得$ F(x) = \int_{-\infty}^{x}{p(t)dt}$,则称$ p(x)$为$ X$的\textbf{概率密度函数}.

在$ F(x)$导数存在的点有$ p(x) = F'(x)$,其余点处$ p(x)$可任意取值.

\subsection{Expectation \& Variance}
\begin{equation*} \begin{split}
	E(X) & = \int_{\mathbb{R}}{x\mathrm{d}F(x)},其中积分为Lebsegue积分. \\
			& = \left\{  \begin{matrix}
	\sum{x_ip(x_i)} = \sum_{n=0}^{\infty}{P(X>n) - P(X<-n)}, 离散,\sum{|x_i|}P(X = x_i)< \infty \\
	\int_{\mathbb{R}}{xp(x)\mathrm{d}x} = \int_{\mathbb{R^+}}{P(X>x)\mathrm{d}x} - \int_{\mathbb{R^-}}{P(X<x)\mathrm{d}x},连续,\int_{\mathbb{R}}{|x|p(x)\mathrm{d}x} < \infty
\end{matrix}\right.
\end{split} \end{equation*}

$\bullet$ 绝对收敛保证了和的存在且与顺序无关.
\\

当期望存在时: $nP(X > n) = n\int_n^{\infty}\mathrm{d}F(x)\le \int_n^{\infty}x\mathrm{d}F(x)$

上式取极限$n\to \infty, 得 nP(X>n) \to 0 ,即 \lim \limits_{x\to \infty}x(1-F(x)) = 0$

同理有$\lim \limits_{x \to -\infty}xF(x) = \lim \limits_{x\to \infty}x(1-F(x)) = 0$

由此极限可推出\textbf{期望的几何意义:} $\int_{-\infty}^{EX}{F(x)\mathrm{d}x} = \int_{EX}^{\infty}{(1-F(x))\mathrm{d}x}$

即:$y=F(x), x=EX, 将0\le y \le 1$分成面积相等的两部分.

\textbf{证:}对两边进行分部积分即可.
\\

\textbf{Cauchy-Schwartz:}

$E(X^2),E(Y^2) < \infty, 则(E(XY))^2 \le E(X^2)E(Y^2)$

\textbf{证:}考虑$ f(u) = E(Xu+Y)^2 = (EX^2)u^2 + 2E(XY)u+EY^2$的判别式即可.
\\

\textbf{期望的统计意义:}

$E(X-a)^2 =E(X-EX)^2 + 2 E(X-EX)(EX-a) + (EX-a)^2= E(X-EX)^2 + (EX-a)^2 \ge E(X-EX)^2. $

若$ E(X^2)$存在,则定义$ Var(X) = E(X - E(X))^2 = \left \{ \begin{matrix}\sum(x_i - E(X))^2p(x_i) \\ \int_{\mathbb{R}}{(x-E(X))^2p(x)\mathrm{d}x} \end{matrix}\right.$

$ Var(X) = E(X^2 - 2XE(X) + (E(X))^2) = E(X^2) - (E(X))^2$

变异系数$ C_v(X) = \dfrac{\sigma(X)}{EX} = \dfrac{\sqrt{Var(X)}}{EX}$.消去了量纲的影响.

偏度系数$ \beta_{s} = \dfrac{E(X-EX)^3}{(Var(X))^{\frac{3}{2}}}$.描述偏离对称性的程度.

峰度系数$ \beta_{k} = \dfrac{E(X-EX)^4}{(Var(X))^2} - 3$.描述相比于正态分布的尖峭程度(尾部粗细).正态分布峰度为0.

\begin{flalign*}
	\textbf{(Chebyshev)} & \forall \varepsilon > 0, P(|X-EX| \ge \varepsilon)          \\
	=                    & \int_{|x- EX| \ge \varepsilon}{p(x)\mathrm{d}x} \le \int_{|x-EX| \ge \varepsilon}{\dfrac{(x-EX)^2}{\varepsilon^2}p(x)\mathrm{d}x} \\
	\le                  & \dfrac{\int_{\mathbb{R}}{(x-EX)^2}p(x)\mathrm{d}x}{\varepsilon^2} = \dfrac{Var(X)}{\varepsilon^2}\\
	统计意义:&与均值的距离远近对概率的限定.
\end{flalign*}

\subsection{Common Discrete Distributions}
\begin{enumerate}
	\item \textbf{二项分布}$ b(n,p)$

		$	b(n,p)  = n * b(1,p)$
		$ P(X = k) = C_n^kp^k(1-p)^{n-k}, k = 0,\cdots,n$

		$ EX = np, Var(X) = E[X(X-1)]+EX-(EX)^2 = np(1-p)$

	二项分布的最大值发生在$ k = \left \{ \begin{matrix}
			(n+1)p 或 (n+1)p - 1 &, (n+1)p \in \mathbb{N} \\
			\lfloor(n+1)p\rfloor&,  (n+1)p \not \in \mathbb{N}
		\end{matrix}\right.$

	\item \textbf{Poisson 分布} $ P(\lambda),(\lambda > 0)$

		$ P(X = k) = \dfrac{\lambda^k}{k!}e^{-\lambda}$

		$ EX = Var(X) = \lambda$

		\textbf{Poisson 定理},对二项分布$ b(n,p_n),\lim \limits_{n\to \infty} np_n = \lambda > 0$,
		\begin{flalign*}
			P(X = k) & = C_n^kp_n^k(1-p_n)^{n-k}                                                                                \\
						   & =\dfrac{n(n-1)\cdots(n-k+1)}{k!n^k}(np_n)^k(1-\dfrac{np_n}{n})^{n-k}                                     \\
						& =(1-\dfrac{1}{n})\cdots(1-\dfrac{k-1}{n})\dfrac{[\lambda + o(1)]^k}{k!}[1-\dfrac{\lambda+o(1)}{n}]^{n-k} \\
					 & \rightarrow \dfrac{\lambda^k}{k!}e^{-\lambda}, n\to \infty
		\end{flalign*}

	\item \textbf{几何分布}$ G(p)$

		$ P(X=k) = p(1-p)^{k-1}, k \in \mathbb{N^+}$

		$ EX = \sum_{n=1}^{\infty}\sum_{k=1}^{n}{p(1-p)^{n-1}}=
		\sum_{k=1}^{\infty}\sum_{n=k}^{\infty}{p(1-p)^{n-1}}= \dfrac{1}{p}$

		类似方法使用两次求出$ E(X(X-1)),Var(X) = \dfrac{1-p}{p^2}$

		尾概率$ P(X>m) = (1-p)^m$

		无记忆性 $\Leftrightarrow P(X>m+n) = P(X>m)P(X>n)\Leftrightarrow X\sim G(P(X\le 1)) $

		(即解Cauchy方程)

	\item \textbf{超几何分布} $ h(n,N,M),(n, M\le N)$

		意义:$ N$件物品含有$ M$件次品,不放回抽取$ n$次得到的次品数.

		$ P(X=k) = \dfrac{C_M^kC_{N-M}^{n-k}}{C_N^n}, k \in [\max\{0, n-N + M\}, \min\{M,n\}]$

		$ EX = \sum_{k}{k\dfrac{C_M^kC_{N-M}^{n-k}}{C_N^n}}  = \dfrac{Mn}{N}\sum_{k}{\dfrac{C_{M-1}^{k-1}C_{N-M}^{n-k}}{C_{N-1}^{n-1}}} = \dfrac{Mn}{N}$

		类似地使用Vandermonde Convolution,有$ 	E[X(X-1)] = \dfrac{M(M-1)n(n-1)}{N(N-1)}$

		$\Rightarrow Var(x)  = \dfrac{nM(N-M)(N-n)}{N^2(N-1)} $

	\item \textbf{Pascal 分布(负二项分布)} $ Nb(r,p)$

		意义:事件发生第$ r$次时的实验次数.$ Nb(r, p) = r * G(p) $

		$ P(X = k) = C_{k-1}^{r-1}p^r(1-p)^{k-r}, k = r,r+1,\cdots$

		$ EX = \dfrac{r}{p}, Var(x) = \dfrac{r(1-p)}{p^2}$

	\item \textbf{错排问题} $ X $为匹配到自己的人数.

		$ P(X = k) = \dfrac{C_n^kD_{n-k}}{n!}, 其中D_k = k!\sum_{i=0}^{k}{\dfrac{(-1)^i}{i!}}为错排数$

		$ EX = \sum_{k=0}^n{\dfrac{nC_{n-1}^{n-k}D_{n-k}}{n!}} = 1. $

		或:每个人匹配到自己的期望为$ \dfrac{1}{n},$所以总期望为$ 1$

		$ E[X(X-1)] = 1, Var(X) = 1$
\end{enumerate}
\subsection{Common Continuous Distributions}
\begin{enumerate}
	\item \textbf{均匀分布} $U[a,b]$

	$p(x) = \dfrac{1}{b-a}I_{[a,b]}(x) = \left \{ \begin{matrix}\dfrac{1}{b-a}, & a\le x \le b \\0, &  \text{else}\end{matrix}\right.$

		$EX = \dfrac{a+b}{2}, Var(X) = \dfrac{(b-a)^2}{12}$

	\item \textbf{指数分布}$Exp(\lambda),(\lambda > 0)$

		$p(x) = \lambda e^{-\lambda x}I_{\mathbb{R^+}}(x)$

		$F(x) = 1-e^{-\lambda t}$

		设备在时刻$t$的失效率$\lambda(t) = \lambda$为常数,则寿命$X\sim Exp(\lambda)$

		$EX = \int_{\mathbb{R^+}}{P(X>x)\mathrm{d}x} = \int_{\mathbb{R^+}}{e^{-\lambda x}\mathrm{d}x} = \dfrac{1}{\lambda}$

		$ EX^2 = \int_{\mathbb{R^+}}{P(X^2 > x)\mathrm{d}x} \xlongequal{x = u^2} \dfrac{2}{\lambda}\int_{\mathbb{R^+}}{u\lambda e^{-\lambda u}\mathrm{d}u}
		= \dfrac{2EX}{\lambda} = \dfrac{2}{\lambda^2}$

		$Var(X) = EX^2 - (EX)^2 = \dfrac{1}{\lambda^2}$

		无记忆性 $\Leftrightarrow P(X>m+n) = P(X>m)P(X>n)\Leftrightarrow X\sim Exp(\lambda)$

	$(0,t]$发生的次数$\sim P(\lambda t),$则第一次发生的时间$\sim Exp(\lambda)$

    $ X\sim U(0,1)\Rightarrow  \dfrac{-\ln X}{\lambda}\sim Exp(\lambda)$


\item \textbf{一维Gauss分布(正态分布)}$N(\mu, \sigma^2)$

	$p(x) = \dfrac{1}{\sqrt{2\pi \sigma^2}}e^{-\frac{(x-\mu)^2}{2\sigma^2}}$

	对$Y\sim N(\mu, \sigma^2), $标准化:$X = \dfrac{Y - \mu}{\sigma},则 X\sim N(0,1)$

    $ P(X \le x) = \int_{-\infty}^{x}{\frac{1}{\sqrt{2\pi}}e^{-\frac{u^2}{2}}\mathrm{d}u} \xlongequal{def}\int_{-\infty}^{x}{\varphi(u)\mathrm{d}u}\xlongequal{def}\Phi(x)$

	利用Poisson积分,可验证$\Phi(\infty) = 1$

  $EX = \int_{\mathbb{R}}{x\dfrac{1}{\sqrt{2\pi}}e^{-\frac{x^2}{2}}\mathrm{d}x} = 0$

	$ Var(X) = EX^2 = 1. 分部积分$

	做线性逆变换可得$EY = \mu, Var(Y) = \sigma^2$

	$ P(|Y-\mu| < k\sigma) = \Phi(k) - \Phi(-k) = 2\Phi(k)-1$

\item \textbf{Gamma分布} $\Gamma(\alpha, \lambda),(\alpha > 0为形状参数,\lambda>0为尺度参数)$

    $ \Gamma 函数: \Gamma(x) = \int_{0}^{\infty}t^{x-1}e^{-t}\mathrm{d}t$

	$ p(x) = \dfrac{\lambda^{\alpha}}{\Gamma(\alpha)}x^{\alpha - 1}e^{-\lambda x}I_{[0,\infty)}(x)$

	$ EX = \dfrac{\lambda^{\alpha}}{\Gamma(\alpha)}\Gamma(\alpha+1) = \dfrac{\alpha}{\lambda}$

	$ EX^2 = \dfrac{\alpha(\alpha+1)}{\lambda^2},Var(X) = \dfrac{\alpha}{\lambda^2}$

	$ \Gamma(1,\lambda )= Exp(\lambda), \Gamma(\dfrac{n}{2},\dfrac{1}{2}) = \chi ^2(n)$
  \item \textbf{Beta分布} $ Be(a,b),(a,b>0)$

    $ Beta函数:B(a,b) = \int_{0}^{1}{x^{a-1}(1-x)^{b-1}\mathrm{d}x} = \dfrac{\Gamma(a)\Gamma(b)}{\Gamma(a+b)}$

  $ p(x) = \dfrac{1}{B(a,b)}x^{a-1}(1-x)^{b-1}I_{(0,1)}(x)$

    $ EX = \dfrac{\Gamma(a+b)}{\Gamma(a)\Gamma(b)}\dfrac{\Gamma(a+1)\Gamma(b)}{\Gamma(a+b+1)} = \dfrac{a}{a+b}$

    $ EX^2 = \dfrac{\Gamma(a+b)}{\Gamma(a)\Gamma(b)}\dfrac{\Gamma(a+2)\Gamma(b)}{\Gamma(a+b+2)} = \dfrac{a(a+1)}{(a+b)(a+b+1)}$

    $ Var(X) = \dfrac{ab}{(a+b)^2(a+b+1)}$

    $ p(x)=ax^{a-1}\Leftrightarrow X\sim Be(a,1)$

  \item \textbf{卡方分布} $ \chi^2(k).(自由度k>0)$

    $ Y_1 \xlongequal{i.i.d} \cdots \xlongequal{i.i.d} Y_k \sim N(0,1)
    \Rightarrow X = \sum{Y^2}\sim \chi^2(k)$

  $ p(x) = \dfrac{x^{\frac{k}{2}-1}e^{-\frac{x}{2}}}{2^{k/2}\Gamma(\frac{k}{2})} I_{[0,+\infty)(x)}$

    $ EX = k. Var(X) = 2k$

  \item \textbf{对数正态分布}$ LN(\mu, \sigma^2)$

    $ X\sim N(\mu,\sigma^2)\Rightarrow Y = e^{X}\sim LN(\mu,\sigma^2)$

    $ p(x) = \dfrac{1}{\sqrt{2\pi}x\sigma}e^{-\frac{(\ln x- \mu)^2}{2\sigma^2}}I_{\mathbb{R}^{+}}(x)$

    $ EX = e^{\mu + \frac{\sigma^2}{2}}, Var(X) = e^{2\mu+\sigma^2}(e^{\sigma^2}-1). 中位数e^\mu$

  \item \textbf{Cauchy分布}$ Cauchy(\mu, \lambda).$

    $ Y \xlongequal{i.i.d} Z \sim N(0,1) \Rightarrow \dfrac{Y}{Z}\sim Cauchy(0,1)$

    $ p(x) = \dfrac{\lambda}{\pi(\lambda^2 +(x-\mu)^2)}$

    期望与方差不存在.

  \item \textbf{Weibull分布}$ W(\lambda,k)$

    设备失效率=$ \lambda t^{k-1}\Rightarrow  寿命服从W(\lambda,k). $

    $ W(\lambda,1) = Exp(\lambda)$

    $ p(x) = \dfrac{k}{\lambda}(\dfrac{x}{\lambda})^{k-1}e^{-(\frac{x}{\lambda})^k}I_{[0,+\infty)}(x)$

      $ F(x) = 1-e^{-(\frac{x}{\lambda})^k}$

      $ EX = \lambda \Gamma(1+\dfrac{1}{k}), Var(X) = \lambda^2\Gamma(1+\dfrac{2}{k})-(EX)^2$

    \item \textbf{Laplace分布} $ L(\mu,b)$

      $ p(x)=\dfrac{1}{2b}e^{-\frac{|x-\mu|}{b}}, F(x)=\dfrac{1+\sgn(x-\mu)}{2}(1-e^{-\frac{|x-\mu|}{b}})$

      $ X,Y\sim Exp(\frac{1}{\lambda})\Rightarrow X-Y\sim L(0,\lambda), |X-Y|\sim Exp(\dfrac{1}{\lambda})$

      $ X,Y\sim \chi^2(1)\Rightarrow X-Y\sim L(0,1)$

\end{enumerate}

\textbf{密度公式:}
函数$g(x)单调连续,h(x) = g^{-1}(x)连续可微,则Y = g(X)$的密度函数为$ p_Y(y) = p_X(h(y))|h'(y)|$

\textbf{证:}$F_Y(y) = P(g(X)\le y)=\int_{h(-\infty,y]}{p_X(x)\mathrm{d}x} \xlongequal{x=h(z)}RHS$

应用: $X\sim \Gamma(\alpha, \lambda) \Rightarrow  kX \sim \Gamma(\alpha, \dfrac{\lambda}{k}), 2\lambda X \sim \chi^2(2\alpha)$

\textbf{对角线:}$ F_X(x)$若为严格增的连续函数,$F_X(X) \sim U(0,1)$

\textbf{证}:$ F_{F_X(X)}(x) = P(F_X(X)\le x) = P(X \le F_X^{-1}(x)) = F_X(F_X^{-1}(x)) =x$
