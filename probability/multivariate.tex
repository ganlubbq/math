% File: multivariate.tex
% Date: Thu Dec 27 23:31:33 2012 +0800
% Author: Yuxin Wu <ppwwyyxxc@gmail.com>
\section{Multivariate Distributions}
\subsection{Joint Distribution}
\textbf{联合分布函数:}$ F(x_1,x_2,\cdots, x_n) = P(X_1\le x_1,X_2\le x_2,\cdots,X_n\le x_n)$
\\

二维联合分布函数$ F(x,y)\Leftrightarrow $
\begin{enumerate}
  \item 对每个变元单调非减,右连续;

  \item $ F(-\infty,y) = F(x,-\infty)= 0\le F(x,y)\le 1= F(\infty,\infty)$;

  \item $ P(a<X\le b, c<Y\le d) = F(b,d)-F(a,d)-F(b,c)+F(a,c)\ge 0$.

\end{enumerate}

\textbf{边际分布函数:}$ F_X(x) = F(x,\infty), F_Y(y) = F(\infty,y)$

\textbf{联合密度函数:} 存在非负函数$ p(x,y), F(x,y) = \int_{-\infty}^x{\int_{-\infty}^y{p(u,v)\mathrm{d}u\mathrm{d}v}}$

\textbf{边际密度函数:}当联合密度函数存在时,每个边际分布函数对应的密度

$ p_X(x) = \int_{\mathbb{R}}p(x,y)\mathrm{d}y,p_Y(y) = \int_{\mathbb{R}}{p(x,y)\mathrm{d}x}$

在$ F(x,y)$偏导数存在处有$ p(x,y) = \dfrac{\partial^2}{\partial x\partial y}F(x,y)$

\textbf{独立}:$ F(x_1,\cdots x_n)=\prod{F_i(x_i)}\Leftrightarrow \begin{cases} P(X_1=x_1,\cdots X_n=x_n)=\prod{P(X_i=x_i)}\\ p(x_1,\cdots x_n)=\prod{p_i(x_i)}\end{cases}$

\subsection{Common Multivariate Distributions}
\begin{enumerate}
  \item \textbf{多项分布}

    $ n$次实验,每次有$ r$种可能结果,概率分别为$ p_1\cdots p_r$.以各种结果出现的次数作为随机变量,有
    $ P(X_1=n_1, \cdots X_r = n_r) = \dfrac{n!}{\Oldprod{n_i!}}\prod{p_i^{n_i}}$

    \setlength{\mathindent}{-3cm}
    概率是多项式$ (\sum p_i x_i)^n$展开式中的系数.
    \begin{align*}
    P(X_1=n_1)&=\dfrac{n!}{n_1!(n-n_1!)}p_1^{n_1}\sum_{n_2+\cdots +n_r=n-n_1}{\dfrac{(n-n_1)!}{\Oldprod_{i=2}^r{n_i!}}{\Oldprod_{i=2}^r{p_i^{n_i}}}} \\
              &=\dfrac{n!}{n_1!(n-n_1!)}p_1^{n_1}(\sum_{i=2}^r{p_i})^{n-n_1}\\
              &=\dfrac{n!}{n_1!(n-n_1!)}p_1^{n_1}(1-p_1)^{n-n_1}. 边缘分布为二项分布.
    \end{align*}

  \item \textbf{多维超几何分布}

    $ N $个球中, $i$号球有$ N_i$个,任取$ n$个,其中各号球的个数作为随机变量,有
    $ P(X_1=n_1,\cdots X_r=n_r)=\dfrac{\prod{C_{N_i}^{n_i}}}{C_N^n}$

  \item \textbf{多维均匀分布} $ U(D)$

    $ D $为$\mathbb{R}^n$的有界可测子集,测度为$ S_D, p(x_1\cdots x_n)=\dfrac{1}{S_D}I_D(x_1\cdots x_n)$

  \item \textbf{二维指数分布}

    $ F(x,y) = (1-e^{-x}-e^{-y}+e^{-x-y-\lambda xy})I_{\mathbb{R^+}\times \mathbb{R^+}}(x,y)$

    边际分布为$ Exp(1)$

  \item \textbf{二维正态分布} $ N(\mu_1, \mu_2,\sigma_1,\sigma_2,\rho), (|\rho| < 1)$

    $ p(x,y) = \dfrac{1}{2\pi\sigma_1\sigma_2\sqrt{1-\rho^2}}e^{-\frac{Q}{2(1-\rho^2)}} $

    其中$Q = \dfrac{(x-\mu_1)^2}{\sigma_1^2}-2\rho\dfrac{(x-\mu_1)(y-\mu_2)}{\sigma_1\sigma_2}+\dfrac{(y-\mu_2)^2}{\sigma_2^2}$

    由$ |\rho|< 1$可知$ Q$为正定二次型.
    \setlength{\mathindent}{-3cm}
    \begin{align*}
    \textbf{边际分布:} &p_X(x) =  \dfrac{1}{2\pi\sigma_1\sigma_2\sqrt{1-\rho^2}}\int_{\mathbb{R}}{e^{-\frac{Q}{2(1-\rho^2)}}\mathrm{d}y} \\
      = & \dfrac{1}{2\pi\sigma_1\sigma_2\sqrt{1-\rho^2}}e^{-\frac{(x-\mu_1)^2}{2\sigma_1^2}}\int_{\mathbb{R}}e^{-\frac{T^2}{2}}\mathrm{d}y\quad (T = \rho\dfrac{x-\mu_1}{\sigma_1\sqrt{1-\rho^2}}-\dfrac{y-\mu_2}{\sigma_2\sqrt{1-\rho^2}})\\
      = & \dfrac{1}{2\pi\sigma_1\sigma_2\sqrt{1-\rho^2}}e^{-\frac{(x-\mu_1)^2}{2\sigma_1^2}}\sigma_2\sqrt{1-\rho^2}\int_{\mathbb{R}}{e^{-\frac{t^2}{2}}\mathrm{d}t} \\
      = & \dfrac{1}{\sqrt{2\pi}\sigma_1}e^{-\frac{(x-\mu_1)^2}{2\sigma_1^2}} 为一维正态分布.
    \end{align*}
    \textbf{相关系数:}设$ (X,Y)\sim N(0,0,1,1,\rho)$

    $则p(x,y) = \dfrac{1}{2\pi\sqrt{1-\rho^2}}e^{-\frac{x^2-2\rho xy+y^2}{2(1-\rho^2)}} = \dfrac{1}{\sqrt{2\pi}}e^{-\frac{y^2}{2}}g(x,y)$

    其中$ z(x) = g(x,y)$为$Z\sim N(\rho y,1-\rho^2)$的密度函数.
    \begin{align*}
    于是 r(X,Y) =& \Cov(X,Y) = E(XY)-EXEY=\iint_{\mathbb{R}^2}xyp(x,y)\mathrm{d}x\mathrm{d}y\\
    =& \int_{\mathbb{R}}\dfrac{y}{\sqrt{2\pi}}e^{-\frac{y^2}{2}}\int_{\mathbb{R}}xg(x,y)\mathrm{d}x\mathrm{d}y\\
    =& \int_{\mathbb{R}}EZ\dfrac{y}{\sqrt{2\pi}}e^{-\frac{y^2}{2}}\mathrm{d}y\\
    =& \rho EY^2 = \rho Var(Y) = \rho
    \end{align*}

    做线性变换可知,任意二维正态分布的相关系数为$ \rho$.

    $ \rho = 0$时易证$ p(x,y) = p_X(x)p_Y(y),$于是,对于联合正态分布,不相关与独立等价.

    \textbf{椭圆域内概率:}$ D=\{(x,y): Q \le t \}, 则P((X,Y)\in D) = 1-e^{-\frac{t}{2(1-\rho^2)}}$

    \textbf{证:}先做仿射变换 $ u = \dfrac{x-\mu_1}{\sigma_1}-\rho\dfrac{y-\mu_2}{\sigma_2}, v = \dfrac{y-\mu_2}{\sigma_2}\sqrt{1-\rho^2}, u^2+v^2=Q$

    $ J= \begin{vmatrix} \dfrac{1}{\sigma_1} & 0 \\ -\dfrac{\rho}{\sigma_2} & \dfrac{\sqrt{1-\rho^2}}{\sigma_2} \end{vmatrix} = \dfrac{\sqrt{1-\rho^2}}{\sigma_1\sigma_2}$

    于是,LHS$ =\dfrac{1}{2\pi\sigma_1\sigma_2\sqrt{1-\rho^2}}\iint\limits_{D}e^{-\frac{Q}{2(1-\rho^2)}}\mathrm{d}x\mathrm{d}y = \dfrac{1}{2\pi(1-\rho^2)}\iint\limits_{u^2+v^2\le t}e^{-\frac{u^2+v^2}{2(1-\rho^2)}}\mathrm{d}u\mathrm{d}v$

    再做变换$ u = r\sin\theta, v=r\cos\theta, |J^{-1}| = r,有:$

    LHS = $\dfrac{1}{2\pi(1-\rho^2)}\int_{0}^{2\pi}{\mathrm{d}\theta}\int_{0}^{\sqrt{t}}{re^{-\frac{r^2}{2(1-\rho^2)}}\mathrm{d}r}  = 1-e^{-\frac{t}{2(1-\rho^2)}}$

\end{enumerate}

\subsection{Convolution}
\textbf{和的分布(卷积):}

$P(X+Y=k) = \sum_{i\in \mathbb{Z}}{P(X=i)P(y=k-i)}$

$(X,Y)$的联合密度函数为$ p(x,y)$,有:

$F_{X+Y}(z) = \iint\limits_{x+y\le z}p(x,y)\mathrm{d}x\mathrm{d}y \Rightarrow p_{X+Y}(z)=\int_{\mathbb{R}}p(z-t,t)\mathrm{d}t$

$X,Y$独立时,有: $F_{X+Y}(z) = \iint\limits_{x+y\le z}p_X(x)p_Y(y)\mathrm{d}x\mathrm{d}y $

$\Rightarrow p_{X+Y}(z)= \int_{\mathbb{R}}p_X(z-y)p_Y(y)\mathrm{d}y=\int_{\mathbb{R}}{p_X(x)p_Y(z-x)\mathrm{d}x}$

\begin{enumerate}
  \setlength{\mathindent}{-3cm}
  \item \textbf{Poisson分布} $ P(\lambda_1)*P(\lambda_2) = P(\lambda_1+\lambda_2)$

    $\Leftarrow  \sum_{i=0}^k{\dfrac{\lambda_1^ie^{-\lambda_1}}{i!}\dfrac{\lambda_2^{k-i}e^{-\lambda_2}}{(k-i)!}}= $
      $\dfrac{e^{-\lambda_1-\lambda_2}}{k!}\sum_{i=0}^k{C_k^i\lambda_1^i\lambda_2^{k-i}}=\dfrac{e^{-\lambda_1-\lambda_2}}{k!}(\lambda_1+\lambda_2)^k$

      (Raikov)独立变量的和服从Poisson分布,则每个都服从Poisson分布.

  \item \textbf{二项分布} $ b(n,p)*b(m,p) = b(m+n,p)$
    \begin{align*}
      \Leftarrow \sum_{i=0}^k{P(X=i)P(Y=k-i)}&=\sum_{i=\max\{ 0,k-m\}}^{\min\{ n,k\}}{C_n^ip^i(1-p)^{n-i}C_m^{k-i}p^{k-i}(1-p)^{m-(k-i)}}\\
        &= p^k(1-p)^{m+n-k}\sum{C_n^iC_m^{k-i}}\\
        &= C_{m+n}^kp^k(1-p)^{m+n-k}
    \end{align*}
  \item \textbf{Gamma分布} $ \Gamma(\alpha_1,\lambda)*\Gamma(\alpha_2,\lambda) = \Gamma(\alpha_1+\alpha_2, \lambda)$
    \begin{align*}
      \Leftarrow p_{X+Y}(z)&=  \dfrac{\lambda^{\alpha_1+\alpha_2}e^{-\lambda(z-y)}e^{-\lambda y}}{\Gamma(\alpha_1)\Gamma(\alpha_2)}\int_{0}^{z}{(z-y)^{\alpha_1-1}y^{\alpha_2-1}\mathrm{d}y} \\
      &\xlongequal{y=zt}       \dfrac{\lambda^{\alpha_1+\alpha_2}e^{-\lambda z}}{\Gamma(\alpha_1)\Gamma(\alpha_2)}z^{\alpha_1+\alpha_2-1}\int_{0}^{1}{(1-t)^{\alpha_1-1}t^{\alpha_2-1}\mathrm{d}t}\\
      &=\dfrac{\lambda^{\alpha_1+\alpha_2}}{\Gamma(\alpha_1+\alpha_2)}z^{\alpha_2+\alpha_2-1}e^{\lambda z}
    \end{align*}

    $ \Rightarrow $ \textbf{卡方分布} $ \chi^2(m)*\chi^2(n) = \chi^2(m+n)$
\item \textbf{正态分布} $ N(\mu_1,\sigma_1^2)*N(\mu_2,\sigma_2^2) = N(\mu_1+\mu_2,\sigma_1^2+\sigma_2^2)$
  \begin{align*}
    \Leftarrow p_{X+Y}(z) &= \dfrac{1}{2\pi\sigma_1\sigma_2}\int_{\mathbb{R}}{e^{-\frac{Q}{2}}\mathrm{d}y}\quad(Q = \dfrac{(z-y-\mu_1)^2}{\sigma_1^2} + \dfrac{(y-\mu_2)^2}{\sigma_2^2} )\\
      & \xlongequal{对Q配方} \dfrac{1}{2\pi\sigma_1\sigma_2}e^{-\frac{(z-\mu_1-\mu_2)^2}{2(\sigma_1^2+\sigma_2^2)}}\int_{\mathbb{R}}{e^{-\frac{A}{2}(y-T)^2}\mathrm{d}y}, (A = \dfrac{1}{\sigma_1^2} + \dfrac{1}{\sigma_2^2})\\
      & =\dfrac{1}{2\pi\sigma_1\sigma_2}e^{-\frac{(z-\mu_1-\mu_2)^2}{2(\sigma_1^2+\sigma_2^2)}}\sqrt{\dfrac{2\pi}{A}}\\
      &= \dfrac{1}{\sqrt{2\pi(\sigma_1^2+\sigma_2^2)}}e^{-\frac{(z-\mu_1-\mu_2)^2}{2(\sigma_1^2+\sigma_2^2)}}
  \end{align*}
  (Cram\'er)多个独立随机变量的和服从正态分布,则每个都服从正态分布.

\end{enumerate}

\subsection{Other Functions on Random Variables}
\begin{description}
  \item[期望:]
    $ E(g(X,Y))=\begin{cases}\sum_{i,j}g(x_i,y_j)P(X=x_i,Y=y_j)\\\iint_{\mathbb{R}^2}g(x,y)p(x,y)\mathrm{d}x\mathrm{d}y \end{cases}$

    期望的\textbf{可加性}对任意随机变量成立:

    $ E(X+Y)=\int_{\mathbb{R}}x(\int_{\mathbb{R}}p(x,y)dy)dx+\int_{\mathbb{R}}y(\int_{\mathbb{R}}p(x,y)dx)dy = E(X)+E(Y)$

  \item[最值:]
    $ Y=\max\{ X_1,\cdots, X_n\}, F_Y(x) = \prod{F_i(x)}$

    $ Z = \min\{ X_1,\cdots ,X_n\}, F_Z(x) = 1-\prod{(1-F_i(x))}$

    当$ X_1,\cdots ,X_n$i.i.d.时,$ p_Y(x)=nF(x)^{n-1}p(x), p_Z(x)=n(1-F(x))^{n-1}p(x)$
  \item[第$ k$小值:]设$ X_1,\cdots ,X_n$i.i.d.,设``第$ k$小值''这个随机变量为$ M_k$

      $ p_{M_k}(x)=n!\dfrac{F(x)^{k-1}}{(k-1)!}\dfrac{[1-F(x)]^{n-k}}{(n-k)!}p(x)$

      $i<j, p_{M_i,M_j}(x,y)=n!\dfrac{F(x)^{i-1}}{(i-1)!}\dfrac{[F(y)-F(x)]^{j-i-1}}{(j-i-1)!}\dfrac{[1-F(y)]^{n-j}}{(n-j)!}p(x)p(y)$

      $ p_{M_1,\cdots ,M_n}(x_1,\cdots x_n)=\begin{cases}n!\prod{p(x_i)},if x_1<\cdots <x_n\\0,else \end{cases}$
  \item[双射:]
   $ \begin{cases}U=g(X,Y)\\V=h(X,Y)\end{cases},p_{U,V}(u,v) = p_{X,Y}(x(u,v),y(u,v))|\dfrac{\partial(x,y)}{\partial(u,v)}|$

  \item[独立积:]$ Z=XY$\hfill

   设$T = Y$,利用二维双射,有 $p_{Z,T}(z,t) = \dfrac{p_X(\frac{z}{t})p_Y(t)}{|t|}$.对$ t$积分即得$ p_Z(z)$

 \item[独立商:]$ Z=\frac{X}{Y}$\hfill

   设$ T=Y,p_{Z,T}(z,t)=p_X(zt)p_Y(t)|t|.$对$ t$积分.

 \item[线性变换:]$ \vec Y = \mathbf{A} \vec X + \vec B$

   $ p_{\vec Y}(\vec x)=p_{\vec X}(\mathbf{A}^{-1}\vec x-\mathbf{A}^{-1}\vec B)|\det \mathbf{A}^{-1}|$

   正态分布的标准化:设$(X,Y)\sim N(\mu_1,\mu_2,\sigma_1^2,\sigma_2^2,\rho) $

   做变换$\begin{pmatrix} X'\\Y'\end{pmatrix}=\begin{pmatrix}\frac{1}{\sigma_1}&0\\0&\frac{1}{\sigma_2}
   \end{pmatrix}\begin{pmatrix}X\\Y\end{pmatrix}-\begin{pmatrix}\frac{\mu_1}{\sigma_1}\\\frac{\mu_2}{\sigma_2}\end{pmatrix}$

   则$(X',Y')\sim N(0,0,1,1,\rho) $

 \item[Fisher-Snedecor分布:]独立的$X\sim \chi^2(m),Y\sim\chi^2(n).F=\dfrac{X/m}{Y/n}\sim F(m,n).$

   $ p_{F}(x) = \dfrac{1}{B(\frac{m}{2},\frac{n}{2})}(\dfrac{m}{n})^{\frac{m}{2}}x^{\frac{m}{2}-1}(1+\dfrac{mx}{n})^{-\frac{m+n}{2}}$

   $ \dfrac{mF}{n+mF} \sim B(\dfrac{m}{2},\dfrac{n}{2})$

 \item[学生t-分布]独立的$X\sim N(0,1),Y\sim\chi^2(n).t=\dfrac{X}{\sqrt{\frac{Y}{n}}}\sim t(n) $

   由对称性,$ F_t(x)=F_t(0)+\dfrac{1}{2}P(t^2\le x^2)=\dfrac{1}{2}+\dfrac{1}{2}F_{t^2}(x^2)$

   而$ t^2\sim F(1,n).$于是$ p_t(x)=\dfrac{\Gamma(\frac{n+1}{2})}{\sqrt{n\pi}\Gamma(\frac{n}{2})}(1+\dfrac{x^2}{n})^{-\frac{n+1}{2}}$

   $t(1) = Cauchy(0,1) ;t(\infty)=N(0,1)$
\end{description}

\subsection{Correlation}

对独立的$X_1,\cdots X_n$,显然有$E(\prod{X_i})=\prod{EX_i} ,Var(\sum{a_iX_i})=\sum{a_i^2Var(X_i)}$

对任意的$ X,Y,有Var(X\pm Y)=Var(X)+Var(Y)\pm2E[(X-EX)(Y-EY)]$.

记协方差$ \Cov(X,Y) = E[(X-EX)(Y-EY)] = E(XY)-EXEY(若E(XY)存在)$

记相关系数$ r(X,Y)=\dfrac{\Cov(X,Y)}{\sigma(X)\sigma(Y)}$

独立$ \Rightarrow E(XY) = EXEY\Leftrightarrow r(X,Y) = 0\Leftrightarrow 不相关$

$ r(X,Y) = \pm 1\Leftrightarrow X,Y几乎处处线性相关\Leftrightarrow P(Y=aX+b)=1$
\\

由于$ E(XY)$是对角线上非负的对称双线性函数,可作为随机变量的内积的定义.因此,由Cauchy's Inequality:

$\Cov(X,Y)^2=|\langle X-EX, Y-EY\rangle |^2 \le |X-EX||Y-EY|=Var(X)Var(Y) $

$ \Leftrightarrow |r(X,Y)|\le 1. $

$  r$可看做$X-EX, Y-EY$的夹角余弦,$EX$看做$ X$在常数子空间上的投影.

将对角线上非负的对称双线性函数$ \Cov(X,Y)$看做内积,同样可得$ |r(X,Y)|\le 1$.
\\

\textbf{线性回归}:求使$ E[X-(aY+b)]^2$最小的$(a,b)$:

考虑内积空间中垂直最小, $E[(X-(\hat{a}Y+\hat{b})(aY+b))] = 0 $

$\Rightarrow  E[X-(\hat{a}Y+\hat{b})]=0,E[Y(X-(\hat{a}Y+\hat{b}))]=0. $

解得$\hat{a}=\dfrac{\Cov(X,Y)}{Var(Y)},\hat{b}=EX-\hat{a}EY.\quad \hat{X} = \dfrac{\Cov(X,Y)}{Var(Y)}(Y-EY)+EX$
\begin{align*}
  此时误差 E(X-\hat{X})^2 &= E[(X-EX)-\dfrac{\Cov(X,Y)}{Var(Y)}(Y-EY)] \\
  &= Var(X)-\dfrac{\Cov(X,Y)^2}{Var(Y)}\\
  &=Var(X)[1-r(X,Y)^2]
\end{align*}
\label{sec:linear}

\textbf{协方差矩阵} $ \vec{X} = (X_1,\cdots ,X_n)',每个分量期望都存在,则随机向量\vec{X}有期望$.

定义协方差矩阵为$\Cov(\vec{X})= (\Cov(X_i,X_j))_{n\times n}=E[(\vec{X}-E\vec{X})(\vec{X}-E\vec{X})']$

$Var(\sum{X_i})=\sum{Var(X_i)} +\sum_{1\le<i<j\le n}{\Cov(X_i,X_j)}$为矩阵中所有元素和.

$ \Cov(\vec{X})$为\textbf{半正定对称阵}:任取向量$ \vec{\alpha}$,
\begin{align*}
\vec{\alpha'}\Cov(\vec{X})\vec{\alpha} =& \sum_{i=1}^{n}\sum_{j=1}^{n}{\alpha_i\alpha_j\Cov(X_i,X_j)}\\
=&\sum_{i=1}^n\sum_{j=1}^n{E[\alpha_i(X_i-EX_i)\alpha_j(X_j-EX_j)]}\\
=&E[\sum_{i=1}^n\sum_{j=1}^n{\alpha_i(X_i-EX_i)\alpha_j(X_j-EX_j)}]\\
=&E[\sum_{i=1}^n{\alpha_i(X_i-EX_i)}\sum_{j=1}^n{\alpha_j(X_j-EX_j)}]\\
=&E[\sum_{i=1}^n{\alpha_i(X_i-EX_i)}]^2\ge 0
\end{align*}

\subsection{Gauss Distribution}
$ \vec{Z} = (Z_1,\cdots Z_n)^{T}, Z_i\sim N(0,1)相互独立,A=(a_{ij})_{m\times n}, \vec{\mu} = (\mu_1,\cdots ,\mu_m)^{T},则\vec{X} = A\vec{Z}+\vec{\mu}
服从m$维\textbf{Gauss分布},记$\vec{X}\sim N(E\vec{X},\Cov(\vec{X})) = N(\vec{\mu}, \Sigma)$

$ E\vec{X} = \vec{\mu}, Var(X_i) = \sum_{k}a_{ik}^2$

$ \Cov(Z_k,Z_l) = \delta_{kl}\Rightarrow \Cov(X_i,X_j)=\sum_{k,l}a_{ik}a_{jl}\Cov(Z_k,Z_l) =
\sum_{k}a_{ik}a_{jk}\Rightarrow \Sigma = \Cov(\vec{X})=AA^T$

  $ \vec{X}服从m维$\textbf{正态分布}$\Leftrightarrow r(A)=r(AA^T)=r(\Cov(X))=m\Leftrightarrow \Cov(\vec{X})正定$

$ \varphi(\vec{t})=e^{i\vec{t}^T\vec{\mu}-\frac{1}{2}\vec{t}^T\Sigma\vec{t}},$且对于每个非负实对称阵$ \Sigma$,此函数都是Gauss分布.

$ \bullet $由特征函数容易证明,Gauss分布做线性变换后仍为Gauss分布.

$ p(\vec{x})=\dfrac{1}{(2\pi)^{\frac{n}{2}}\sqrt{|\Sigma|}}e^{-\dfrac{(\vec{x}-\vec{\mu})^T\Sigma^{-1}(\vec{x}-\vec{\mu})}{2}}$
\\

Matrix Tricks:

设$n维随机向量 \vec{X}\sim N(0,I), $
取\textbf{正交阵}$A,使其首行为(\dfrac{1}{\sqrt{n}},\cdots ,\dfrac{1}{\sqrt{n}}),$
则显然$ \vec{Y} = A\vec{X}\sim N(0,I)$
\begin{flalign*}
  \bar{X} = &\dfrac{\Oldsum{X_i}}{n}=\dfrac{Y_1}{\sqrt{n}}&\\
  (n-1)S^2 =& \sum(X_i-\bar{X})^2=\sum{X_i^2}-n\bar{X}^2=X^TX-Y_1^2&\\
  =& Y^TAA^{-1}Y-Y_1^2=Y_2^2+\cdots +Y_n^2\sim \chi^2(n-1) &\\
  \Rightarrow & \bar{X}与(n-1)S^2独立&\\
\end{flalign*}

\subsection{Conditional Distribution}
设$ A$为事件,$ F_{X|A}(x)=P(X\le x|A) = \int_{-\infty}^{x}p_{X|A}(u)\mathrm{d}u$

当$ Y$为随机变量时,根据新的下标重载函数$ F$:

$ F_{X|Y}(x|y) = P(X\le x|Y=y)=\int_{-\infty}^x{\dfrac{p_{X,Y}(u,y)}{p_Y(y)}\mathrm{d}u}=\int_{-\infty}^x{ p_{X|Y}(u|y)}\mathrm{d}u$
\begin{description}
  \item [条件正态分布:]\hfill\\
    \quad $ (X,Y)\sim N(0,0,1,1,\rho)\Rightarrow X|Y=y \sim N(\rho y, 1-\rho^2)$

  $\Rightarrow  E(X|Y) = \rho Y, E(X|Y=y) = \rho y$

\quad$ (X,Y) \sim N(\mu_1, \mu_2, \sigma_1^2, \sigma_2^2, \rho) $

$\Rightarrow  X|Y=y \sim N(\mu_1+\rho\dfrac{\sigma_1}{\sigma_2}(y-\mu_2), \sigma_1^2(1-\rho^2))$

  \item [逆乘法公式:]\hfill\\
    若$p_{X,Y}(x,y) = g(y)h(x,y),且\forall y, \int_{R}h(x,y)\mathrm{d}x=1$

    $则g(y) = p_Y(y), h(x,y) = p_{X|Y}(x|y)$

  \item [连续全概率公式:] $ p_X(x) = \int_{\mathbb{R}}{p_Y(y)p_{X|Y}(x|y)\mathrm{d}y}$

  \item[连续Bayes公式:]$ p_{X|Y}(x|y) = \dfrac{p_X(x)p_{Y|X}(y|x)}{\int_{\mathbb{R}}{p_X(x)p_{Y|X}(y|x)}\mathrm{d}x}$

  \item[条件期望:]\hfill\\
    $ E(X|Y=y) = \int_{\mathbb{R}}{xp_{X|Y}(x|y)\mathrm{d}x}$, 其值是关于$ Y$的随机变量,记做$E(X|Y) $

    $\quad E[g(X)h(Y)|Y=y] = E[g(X)h(y)|Y=y] = h(y) E[g(X)|Y=y]$

    $\Rightarrow E[g(X)h(Y)|Y] = h(Y)E[g(X)|Y]$

\item[重期望公式:]\hfill
  \begin{flalign*}
    E(X) &= \int_{\mathbb{R}}\int_{\mathbb{R}}{xp(x,y)\mathrm{d}x\mathrm{d}y} =\int_{\mathbb{R}}\int_{\mathbb{R}}xp(x|y)p_Y(y)\mathrm{d}x\mathrm{d}y &\\
    &=\int_{\mathbb{R}}{E(X|Y=y)p_Y(y)\mathrm{d}y} = E(E(X|Y))&\\
    E(X) &= \sum_{j}E(X|Y=y_j)P(Y=y_j) 用于很多趣题.&\\
    推论:&随机个随机变量和.设 X_1\cdots \text{i.i.d},则有 E(\sum_{i=1}^N{X_i})=E(X)E(N)&
  \end{flalign*}
\item [条件期望是最佳平方逼近:] $ EX^2<\infty 时,E[X-E(X|Y)]^2 =\min E[X-\varphi(Y)]^2$ \hfill\\
  \textbf{证:}由$ E[E(X|Y)]^2\le E[E(X^2|Y)] =EX^2$知存在性.

  考虑期望在线性空间中的点积意义,需证$ E[[X-E(X|Y)]\varphi(Y)]=0$.

  LHS = $ E[E([X-E(X|Y)]\varphi(Y)|Y)] = E(\varphi(Y)E[X-E(X|Y)|Y]) = E[\varphi(Y)(E(X|Y) - E[E(X|Y)|Y])] = 0$


\end{description}


