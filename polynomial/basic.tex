% $file: basic.tex
% $date: thu mar 01 15:43:58 2012 +0800
% author: ppwwyyxxc@gmail.com

\section{basic definitions}
\subsection{order}
为单项式建立一种序关系(monomial ordering),满足:

1)全序.

2)$ x^a>x^b\rightarrow x^{a+c}>x^{b+c}$

3)非空集有最小元(良序)

满足此定义的一元单项式序必为$ 1<x<x^2<\cdots<\cdots$

对于多元单项式,有字典序(lex),全次字典序(grlex,总次数优先),全次反字典序(grvelex)

定义了序之后就可定义多元多项式除法

$ f=a_1f_1+\cdots+a_sf_s+r,r<f_1,\cdots,f_s$.若$ f=\langle f_1,\cdots,f_s \rangle,$则记$ r=\bar{f}^{f}$

这种除法的结果:

1.与$ f_i$的排列顺序有关.

2.$ f\in f \nrightarrow \bar{f}^f=0$

例如:$ xy^2=y(xy+1)+0(y^2-1)+(-x-y)$

但事实上$ xy^2=x(y^2-1)\in \langle y^2-1,xy+1 \rangle$


\subsection{ideals}
$ \langle f_1,\cdots,f_s \rangle$的生成理想(ideals): 
$ i=\{p_1f_1+\cdots+p_sf_s,p_i\in k[x_1,\cdots,x_n]\}$

如何判断两组多项式的理想相等?朴素:每个多项式都在对方的理想中.

根理想:$ \sqrt{i}=\{g\in k[x_1,\cdots,x_n],\exists m,g^m\in i \}$
\\

商理想:$ I:J={f: \forall g \in J,fg\in I }$

性质: $ I\cap \langle h \rangle =\langle g_1,\cdots,g_t \rangle\Rightarrow I:\langle h \rangle=\langle \frac{g_1}{h},\cdots,\frac{g_t}{h} \rangle$,
其中$ \langle g_1,\cdots,g_t \rangle $为$ I$的Grobner 基
\\

求理想的交:$ I\cap J=(tI+(1-t)J)\cap k[x_1,\cdots x_n]$.

证:对$ f\in RHS,$有$ f=a_1tf_1+\cdots+a_stf_s+\cdots+a_{s+m}(1-t)g_m$

取$ t=0,1$即得$ f\in I,f\in J$.

反之,若$ f\in I,f\in J,$ 由于$ tI+(1-t)J$为线性组合,立得$ f\in RHS$

\subsection{Grobner's Basis}
dickson's lemma:

一些单项式的理想$ I= \langle x^a : a\in A \rangle $总可写为有限个基的理想
$ I= \langle x^{a_1},\cdots,x^{a_s} \rangle $

Def:$ LT(I)=\{q: \exists f \in I,LT(f)=q\}$

则可证存在$ g_1,\cdots,g_s\in I,s.t.  \langle LT(I) \rangle = \langle LT(g_1),\cdots,LT(g_t) \rangle $

进一步有Hilbert's Basis Theorem:

任一个理想可由有限个多项式生成.(多项式环是诺特环(Noetherian Domain))   (基至少要有多少个?)

以及Grobner Basis的存在性:$ G=\{g_1,\cdots,g_t\}\subset I,s.t.\forall f\in I,\exists i, LT(f)|LT(g_i)$

性质:$ \forall f,\bar{f}^F$唯一,且$ f\in F\Rightarrow \bar{f}^F=0$

Reduced Grobner basis:$ \forall p,q\in G,p\ne q,p$中每个单项式都不被$ LT(q)$整除.

再加上首项系数为1后,称作Monic Grobner basis,它是唯一的.
\\

由Hilbert,可证理想的不严格递增序列会停止.设$ I_1\subset\cdots\subset I_n\subset\cdots$

$ I=\cup_{i=1}^{\infty}I_i$也是一个理想(证$ f,g\in I \Rightarrow f+g,pf\in I$)

$ I$可被有限生成.$  \langle f_1,\cdots,f_s \rangle \subset I_n\subset\cdots \subset I$,则之后全取等号.
\\

Grobner Basis判定:(Buchberger's S-pair criterion)

\[  S(f,g)=\frac{R}{LT(f)}f+\frac{R}{(LT(g))}g,R=Lcm(LT(f),LT(g))\]
则$  \langle g_1,\cdots,g_t \rangle $是Grobner基$ \Leftrightarrow \forall i,j,\overline{S(g_i,g_j)}^G=0$
\\

Buchberger's Algorithm:

对$ G={g_1,\cdots g_s}$:

\begin{lstlisting}[mathescape]
   REPEAT:
       $ G'=G x^2 $
       for each pair $ \{ p,q\},p\ne q $ in $ G'$,
             $ S=\overline{S(p,q)}^{G'}$
             if $  S\ne 0 $ then $  G=G \cup \{ S\} $
   UNTIL $ G==G'$
\end{lstlisting}


\subsection{varieties}
方程组$ f_{1,\cdots,s}(x_1,\cdots,x_n)=0$的所有解称为$ f_1,\cdots,f_s$的仿射簇(affine variety)$ V(f_1,\cdots,f_s)$

$ U,V$为仿射簇,则并与交也为仿射簇(可构造出对应的方程组)
\\

显然$ f_1,\cdots,f_s$的理想$ I$中任一多项式Vanishes in $ V(f_1,\cdots,f_s)$

定义$ I(V)=\{f:f(A)=0 ,\forall A \in V\}$

则显然$ I(V)$是一个理想,且$  \langle f_1,\cdots f_s \rangle  \subset I(V(f_1,\cdots,f_s))$

但两者不一定相等,如$  \langle x^2,y^2 \rangle \subset I(V(x^2,y^2))=I(\{0,0\})= \langle x,y \rangle $
\\

(Strong Nullstellensatz)设$ k$为代数闭域(Algebraically closed field),
则$ I(V(I))=\sqrt{I}$

但显然有$ V(I(V))=V$

线性方程组的解空间---线性簇(linear variety)(线,平面)

对一组多项式方程组的求解可应用于Lagrange Multipliers

描述一个不可列的仿射簇,可以用参数方程.但如何由参数方程求仿射簇?


\subsection{elimination}
Definition: the $ l$th elimination ideal $ I_l=I\cap k[x_{l+1},\cdots,x_n]$

(Elimination Theorem):$ G_l=G\cap k[x_{l+1},\cdots,x_n]$是$ I_l$的Grobner基

