% File: basic.tex
% Date: Fri Sep 14 15:02:56 2012 +0800
% Author: Yuxin Wu <ppwwyyxxc@gmail.com>

一次随机试验中每一种可能的结果称为一个{\bf 基本事件}或{\bf 样本点$ \omega$},所有基本事
件的全体为该试验的样本空间$ \Omega$

同一试验的样本空间可能不唯一,因为观察结果的角度不同.对扔两次色子,$ \Omega_1 =
\{++,+-,--,-+\}, \Omega_2 = \{\texttt{两正,两负,一正一负}\}$

$ \Omega$的可测子集$ A$称为事件.对结果$ \omega \in A$,则称事件A发生了.

$ A \subset B \Rightarrow $A发生了B必发生.

运算:Morgan律.$ (\cup A_i)^c = \cap A_i^c$

古典概型: 基本事件只有有限个且概率相同.

掷硬币$ n$次,取每种排列为基本事件,即为古典概型.$ P(\texttt{首次正面出现在k次}=\dfrac{1}{2^k})$

掷硬币直到出现正面为止,基本事件$ \omega_k$为``首次正面出现在第k次'',则有无穷个基本事件,且
概率不同. 
利用可数可加性,$ \omega_{\infty} = 0$,但不是不可能事件.

一般地,对于至多可数集合$ \Omega$,每个基本事件的概率都可求出时,$ \forall A
\subset \Omega, P(A) = \sum_{\omega \in A}P(\omega)$

若无限抛掷硬币,将排列作为基本事件,则有不可数个基本事件,此时若考虑等可能分析,则
每个基本事件概率为0.无法求出某个事件的概率(因为不可数个实数的和没有意义).
\\

古典/几何概型的另一个问题:Bertrand悖论--圆内一弦长度超过正三角形边长的概率由三
种解释.

原因:当可能结果有无穷个时,难以规定``等可能''这一概念.

概率空间$ (\Omega, \mathcal{F}, P)$:

$\Omega$ 是全体可能结果组成的集合.$ \mathcal{F}$是全体可观测事件组成的事件族.$ P:\mathcal{F} \rightarrow [0,1]$是求事件的概率的运算.

当$ \mathcal{F}$满足以下条件时,称其为{\bf $ \sigma -$代数}:
\begin{enumerate}
	\item $ \Omega \in \mathcal{F}$
	\item $ A\in \mathcal{F} \Rightarrow A^c \in \mathcal{F}$
	\item 可数并:$ A_1 \cdots \in \mathcal{F} \Rightarrow  \cup_{i = 1}^{\infty}{A_i}
		\in \mathcal{F}$
\end{enumerate}
事实上,由可数并,可推出有限并,可数交,有限交$ \in \mathcal{F}$.

当$ \Omega$为至多可数集时,总可取$ \Omega$的所有子集族作为$ \mathcal{F}$.
当$ \Omega$不可数时,取这样的$ \mathcal{F}$会造成数学上的困难,因此只取感兴趣的,
可以知道概率的事件的最小$ \sigma$-代数

对每个事件$ A\in \mathcal{F},$定义实数$ P(A)$,满足以下条件:
\begin{enumerate}
	\item 非负性:$ P(A) \ge 0$
	\item 规范性: $ P(\Omega) = 1$
	\item 可数可加性:

		对两两互不相容的事件$ A_1 \cdots \in \mathcal{F}, P(\cup{A_n}) =
		\sum{P(A_n)}$
\end{enumerate}

试验的样本空间,事件$ \sigma$代数及定义在其上的概率构成的三元组$ (\Omega,
\mathcal{F}, P)$称为描述一个随机试验的{\bf 概率空间}.
\\

事件序列的极限定义:
$ \overline {\lim \limits_{n \to \infty}}A_n  = \bigcap_{n
=1}^{\infty}\bigcup_{k=n}^{\infty}A_k $(当且仅当有无穷个$ A_n$发生)

$ \mathop{\underline \lim} \limits_{n \to \infty} A_n = \bigcup_{n
=1}^{\infty}\bigcap_{k=n}^{\infty}A_k$(当且仅当至多有有限个$ A_n$不发生)

当上下极限相等时(如对于单调事件序列),称为序列$ A_n$的极限.
\\

利用可数可加,可得到如下结论:
\begin{enumerate}
	\item $ P(\empty) = 0:
		P(\Omega) = P(\Omega \cup \empty \cup \empty \cdots) = P(\Omega) +
		P(\empty) + \cdots$
	\item 有限可加
	\item 求逆:$ P(A) + P(A^c) = 1$
	\item Jordan公式(容斥),归纳证明
	\item $ P(A - B) = P(A) - P(A\cap B)$,特别地,$ B \subset A \Rightarrow
		P(B) \le P(A)$
	\item 下连续性:设$ A_i$单调增($ A_1 \subset A_2 \subset \cdots$),则$
		P(\lim \limits_{n \to \infty }{A_n}) = \lim \limits_{n \to
		\infty}{P(A_n)}$.

		$ P(\cup A_n) = P(A_1) + \sum_{i = 1}^{\infty}{P(A_{i+1} - A_i)} =
		P(A_1) = \lim \limits_{n \to \infty}\sum_{i = 1}^{n-1}{[P(A_{i+1} -
		P(A_i))]} = \lim \limits_{n \to \infty}P(A_n)$

	\item 上连续性:设$ A_i$单调减,则$ P(\lim \limits_{n \to \infty} A_n ) =
		\lim \limits_{n \to \infty} P(A_n)$

		$ 1 - \lim \limits_{n \to \infty}P(A_n) = \lim \limits_{n \to
		\infty}P(A_n^c)=P(\bigcup_{n=1}^{\infty}A_n^c) =
		P((\bigcap_{n=1}^{\infty}A_n)^c) = 1-P(\bigcap_{n=1}^{\infty}A_n)$
	
		概率的上下连续性等价,统称为连续性.

	\item	有限可加+下连续$ \Leftrightarrow $可数可加.

		由下连续性,\[  P(\bigcup_{n=1}^{\infty}A_n) =
		P(\bigcup_{n=1}^{\infty}F_n) \mathop{=}\limits_{\texttt{下连续}}\lim\limits_{n \to \infty}P(F_n)=
		\lim\limits_{n\to\infty}P(\bigcup_{i=1}^nA_i)
		\mathop{=}\limits_{\texttt{有限可加}}
		\lim\limits_{n\to\infty}\sum_{i=1}^n{P(A_i)}\mathop{=} \limits_{\texttt{收敛}}
	\sum_{i=1}^{\infty}P(A_i)\]
\end{enumerate}

