% $File: summary.tex
% $Date: Sat Jun 02 17:47:54 2012 +0800
% Author: Yuxin Wu <ppwwyyxxc@gmail.com>
\section{总结与感悟}
	本篇报告介绍了多项式方程组求解的结式方法,
	利用线性代数中的一些简单手段,就可以通过构造矩阵实现两个方程间的联合消元,
	进一步可以实现方程组的消元.
	随后,我们简要提及了更高效的抽象代数方法中的几个有价值的概念,
	将其与线性代数中的一些基本概念进行类比,加深了我们对这些代数概念的理解.

	20世纪以来现代科学技术突飞猛进,尤其是计算机科学的兴起为数学插上了腾飞的翅膀,
	数学理论同时也为计算机科学技术提供了广阔的舞台.
	现代的计算机技术为大型的符号计算提供了可能性,
	关键的问题就在于如何把抽象的代数理论算法化, 
	使计算机高效地处理形形色色的代数问题. 
	如今强大的计算机代数系统不仅是各类工程技术的助手,
	对纯粹科学研究也起着不可忽略的推动作用.
	在多项式代数理论上发展的计算机图形学,机器人运动学,密码学
	等一系列计算机专业相关领域和理论应运而生,
	也大规模的应用在我们的日常生活中.
	数学的规模之大,影响之深远,已经超越了所有的时代.
	虚拟现实,三维真实感动画,智能机械手臂等华丽实用的科技让人们切实感受到了数学博大精深,
	以及那种浑然天成的和谐美与对称美.
	
	这次接触计算机与数学领域重叠部分的前沿应用,
	既为未来的我们在计算机系的符号计算领域的学习打下基础,
	也有益于我们对已有的线性代数知识的理解和巩固.
	同时,在尝试用宏观和类比的眼光透视数学问题的过程中,我们也更加体会到了数学的美感,
	以及从变化中抓住不变,从紊乱中归纳条理,在偶然中发现必然,从混沌中整理秩序的科研精神.
	在完成这篇作品的过程中,
	从确定方向,查阅文献,类比研讨,抽象方法,再到研究实践应用的一系列过程里,我们感受到:
	在处理数学问题时要善于类比联想;由想激疑,在释疑中启悟;由疑反思,在思辨中省悟;由思导验,在体验中领悟;由验致用,在应用中彻悟。
