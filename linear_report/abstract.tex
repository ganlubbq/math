% $File: abstract.tex
% $Date: Tue May 15 11:16:58 2012 +0800
% Author: Yuxin Wu <ppwwyyxxc@gmail.com>
\section{抽象代数方法简述}
	本节主要将抽象代数方法里的一个基本概念与线性代数进行类比.
	为了求解线性方程组,我们首先学习了秩,线性相关等概念.

	对于方程组$ \begin{cases}
		f_1(x_1,\cdots,x_n)=0\\
		f_2(x_1,\cdots,x_n)=0\\
		f_3(x_1,\cdots,x_n)=0\\
		\vdots\\
	\end{cases}$
	
	它是一个线性方程组时,若$ f_3=p_1f_1+p_2f_2,p_1,p_2\in \mathbb{R}$,那么在原方程组中,$ f_3=0$是一个"没用"的方程.
	反映在对应的系数矩阵中,
	设$ \overrightarrow b_1,\overrightarrow b_2,\cdots $为行向量,则
	$ \overrightarrow b_3$可由$ \overrightarrow b_1,\overrightarrow b_2$线性表出,或称
	$ \overrightarrow b_3$在$ \overrightarrow b_1,\overrightarrow b_2$生成的线性空间中.

	线性相关的概念描述了在线性方程组中哪些方程是"没用"的,秩的概念进一步给出了线性方程组中"有用"的方程个数.
	多项式方程组求解首先也需解决这一问题.

	将上述方程组视作一个多项式方程组,设$ K[x_1,\cdots, x_n]$为关于$ (x_1,\cdots , x_n)$的多项式集合.
	则若有$ f_3 = p_1f_1 + p_2f_2,p_1,p_2\in K[x_1,\cdots, x_n]$,
	那么$ f_3=0$是一个"没用"的方程.

	于是,类比线性表出的概念,我们很自然的给出了多项式"理想(Ideals)"的定义.
	$ f_1,\cdots,f_s $的生成理想定义为: 
	$ I=\langle f_1,\cdots,f_s \rangle = \{p_1f_1+\cdots+p_sf_s,p_i\in K[x_1,\cdots,x_n]\}$.
	于是,$ f=0$是一个"没用"的方程当且仅当它在其余多项式的生成理想中.
	因此,只需将方程组中全部多项式的生成理想用简单的形式表出,或者说,用一组性质较好的多项式生成,就容易将问题简化.

	显然,理想具有一些基本的性质,例如其中的元素对于加法,数乘封闭.事实上,一个理想本身可作为一个线性空间.
	在线性空间中我们通过找一组基来描述整个空间的性质,在理想中,可以通过Grobner基,来描述整个理想的性质.

	Grobner基是对理想的一种很好的表示,具有很多极好的性质,因此可以被用于多项式方程组求解中.
	当前的许多符号计算软件,如Maple,Mathematica,mathm$ \mu$ 都使用Grobner基方法实现多项式方程组求解.


