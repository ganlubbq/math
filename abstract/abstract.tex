% $File: abstract.tex
% $Date: Mon Mar 05 16:32:34 2012 +0800
% $Author: wyx <ppwwyyxxc@gmail.com>

\documentclass[a4paper]{article}
\usepackage{fontspec,amsmath,amssymb,zhspacing,verbatim,minted}
\usepackage[hyperfootnotes=false,colorlinks,linkcolor=blue,anchorcolor=blue,citecolor=blue]{hyperref}
\usepackage[sorting=none]{biblatex}
\usepackage[dvips]{graphicx}
\usepackage{subfigure}
\zhspacing


\renewcommand{\abstractname}{摘要}
\renewcommand{\contentsname}{目录}
\renewcommand{\figurename}{图}
\defbibheading{bibliography}{\section{参考文献}}
\bibliography{refs.bib}

% \figref{label}: reference to a figure
\newcommand{\figref}[1]{\hyperref[fig:#1]{图\ref*{fig:#1}}}
% \secref{label}: reference to a section
\newcommand{\secref}[1]{\hyperref[sec:#1]{\ref*{sec:#1}节}}
\DeclareMathOperator{\LM}{\tiny{LM}}
\DeclareMathOperator{\LT}{\tiny{LT}}

\let\Oldsum\sum
\renewcommand{\sum}{\displaystyle\Oldsum}
\let\Oldprod\prod
\renewcommand{\prod}{\displaystyle\Oldprod}

% $File: mint-defs.tex
% $Date: Sat Feb 16 22:59:11 2013 +0800
% $Author: wyx <ppwwyyxxc@gmail.com>

\usepackage{xparse}


% \inputmintedConfigured[additional minted options]{lang}{file path}{
\newcommand{\inputmintedConfigured}[3][]{\inputminted[fontsize=\footnotesize,
	label=#3,linenos,frame=lines,framesep=0.8em,tabsize=4,#1]{#2}{#3}}

% \phpsrc[additional minted options]{file path}: show highlighted php source
\newcommand{\phpsrc}[2][]{\inputmintedConfigured[#1]{php}{#2}}
% \phpsrcpart[additional minted options]{file path}{first line}{last line}: show part of highlighted php source
\newcommand{\phpsrcpart}[4][]{\phpsrc[firstline=#3,firstnumber=#3,lastline=#4,#1]{#2}}
% \phpsrceg{example id}
\newcommand{\phpeg}[1]{\inputminted[startinline,
	firstline=2,lastline=2]{php}{res/php-src-eg/#1.php}}

\newcommand{\txtsrc}[2][]{\inputmintedConfigured[#1]{text}{#2}}
\newcommand{\txtsrcpart}[4][]{\txtsrc[firstline=#3,firstnumber=#3,lastline=#4,#1]{#2}}

\newcommand{\pysrc}[2][]{\inputmintedConfigured[#1]{py}{#2}}
\newcommand{\pysrcpart}[4][]{\pysrc[firstline=#3,firstnumber=#3,lastline=#4,#1]{#2}}

\newcommand{\confsrc}[2][]{\inputmintedConfigured[#1]{squidconf}{#2}}
\newcommand{\confsrcpart}[4][]{\confsrc[firstline=#3,firstnumber=#3,lastline=#4,#1]{#2}}

\newcommand{\cppsrc}[2][]{\inputmintedConfigured[#1]{cpp}{#2}}
\newcommand{\cppsrcpart}[4][]{\cppsrc[firstline=#3,firstnumber=#3,lastline=#4,#1]{#2}}

\renewcommand{\P}[1]{\text{P}\left(#1\right)}
\renewcommand{\Pr}[1]{\text{Pr}\left\{#1\right\}}
\newcommand{\Px}[2]{\text{P}_{#1}\left(#2\right)}
\newcommand{\E}[1]{\text{E}\left[#1\right]}
\newcommand{\Ex}[1]{\text{E}#1}
\newcommand{\Var}[1]{\text{Var}\left[#1\right]}
%\newcommand{\Cov}[2]{\text{Cov}\left[#1,#2\right]}
%\newcommand{\Cov}[1]{\text{Cov}\left[#1 \right]}
\renewcommand{\T}[1]{\Theta\left(#1\right)}
\newcommand{\real}{\mathbb{R}}
\newcommand{\card}[1]{\left\|#1\right\|}
\newtheorem{lemma}{Lemma}

\NewDocumentCommand\Cov{mg}{
    \text{Cov}\left[ #1 \IfNoValueTF{#2}{}{,#2}\right]
 }

\newcommand{\qed}{\hfill \ensuremath{\Box}}



\title{Abstract}
\author{\\(ppwwyyxxc@gmail.com)}

\begin{document}
\maketitle

\tableofcontents
% File: basic.tex
% Date: Thu Dec 27 23:09:06 2012 +0800
% Author: Yuxin Wu <ppwwyyxxc@gmail.com>

\section{Basic Concepts}
\subsection{Events}
一次随机试验中每一种可能的结果称为一个{\bf 基本事件}或{\bf 样本点$ \omega$},所有基本事
件的全体为该试验的样本空间$ \Omega$

同一试验的样本空间可能不唯一,因为观察结果的角度不同.对扔两次色子,$ \Omega_1 =
\{++,+-,--,-+\}, \Omega_2 = \{\texttt{两正,两负,一正一负}\}$

至多可数的样本空间称为离散样本空间,不可数称为连续样本空间.

$ \Omega$的可测子集$ A$称为事件.对结果$ \omega \in A$,则称事件A发生了.

$ A \subset B \Rightarrow $A发生了B必发生.

Morgan律:$ (\cup A_i)^c = \cap A_i^c$

\subsection{Probability Space}

概率空间$ (\Omega, \mathcal{F}, P)$:

$\Omega$ 是全体可能结果组成的集合.$ \mathcal{F}$是全体可观测事件组成的事件族.$ P:\mathcal{F} \rightarrow [0,1]$是求事件的概率的运算.

当$ \mathcal{F}$满足以下条件时,称其为{\bf $ \sigma -$代数}:
\begin{enumerate}
	\item $ \Omega \in \mathcal{F}$
	\item $ A\in \mathcal{F} \Rightarrow A^c \in \mathcal{F}$
	\item 可数并:$ A_1 \cdots \in \mathcal{F} \Rightarrow  \cup_{i = 1}^{\infty}{A_i}
		\in \mathcal{F}$
\end{enumerate}
事实上,由可数并,可推出有限并,可数交,有限交$ \in \mathcal{F}$.

当$ \Omega$为至多可数集时,总可取$ \Omega$的所有子集族作为$ \mathcal{F}$.
当$ \Omega$不可数时,取这样的$ \mathcal{F}$会造成数学上的困难,因此只取感兴趣的,
可以知道概率的事件的最小$ \sigma$-代数.
\\

{\bf 概率的定义:}对每个事件$ A\in \mathcal{F},$定义实数$ P(A)$,满足以下条件:
\begin{enumerate}
	\item 非负性:$ P(A) \ge 0$
	\item 规范性: $ P(\Omega) = 1$
	\item 可数可加性:

		对两两互不相容的事件$ A_1 \cdots \in \mathcal{F}, P(\cup{A_n}) =
		\sum{P(A_n)}$
\end{enumerate}

试验的样本空间,事件域($ \sigma$代数)及定义在其上的概率构成的三元组$ (\Omega,
\mathcal{F}, P)$称为描述一个随机试验的{\bf 概率空间}.
\subsection{Properties of Probability}

事件序列的极限定义:
$ \overline {\lim \limits_{n \to \infty}}A_n  = \bigcap_{n
=1}^{\infty}\bigcup_{k=n}^{\infty}A_k $(当且仅当有无穷个$ A_n$发生)

$ \mathop{\underline \lim} \limits_{n \to \infty} A_n = \bigcup_{n
=1}^{\infty}\bigcap_{k=n}^{\infty}A_k$(当且仅当至多有有限个$ A_n$不发生)

当上下极限相等时(如对于单调事件序列),称为序列$ A_n$的极限.
\\

利用可数可加,可得到如下结论:
\begin{enumerate}
	\item $ P(\empty) = 0:
		P(\Omega) = P(\Omega \cup \empty \cup \empty \cdots) = P(\Omega) +
		P(\empty) + \cdots$
	\item 有限可加
	\item 求逆:$ P(A) + P(A^c) = 1$
	\item Jordan公式(容斥),归纳证明
	\item $ P(A - B) = P(A) - P(A\cap B)$,特别地,$ B \subset A \Rightarrow
		P(B) \le P(A)$
	\item 下连续性:设$ A_i$单调增($ A_1 \subset A_2 \subset \cdots$),则$
		P(\lim \limits_{n \to \infty }{A_n}) = \lim \limits_{n \to
		\infty}{P(A_n)}$.

		$ P(\cup A_n) = P(A_1) + \sum_{i = 1}^{\infty}{P(A_{i+1} - A_i)} =
		P(A_1) = \lim \limits_{n \to \infty}\sum_{i = 1}^{n-1}{[P(A_{i+1} -
		P(A_i))]} = \lim \limits_{n \to \infty}P(A_n)$

	\item 上连续性:设$ A_i$单调减,则$ P(\lim \limits_{n \to \infty} A_n ) =
		\lim \limits_{n \to \infty} P(A_n)$

		$ 1 - \lim \limits_{n \to \infty}P(A_n) = \lim \limits_{n \to
		\infty}P(A_n^c)=P(\bigcup_{n=1}^{\infty}A_n^c) =
		P((\bigcap_{n=1}^{\infty}A_n)^c) = 1-P(\bigcap_{n=1}^{\infty}A_n)$

		概率的上下连续性等价,统称为连续性.

	\item	有限可加+下连续$ \Leftrightarrow $可数可加.

		由下连续性,\[  P(\bigcup_{n=1}^{\infty}A_n) =
		P(\bigcup_{n=1}^{\infty}F_n) \mathop{=}\limits_{\texttt{下连续}}\lim\limits_{n \to \infty}P(F_n)=
		\lim\limits_{n\to\infty}P(\bigcup_{i=1}^nA_i)
		\mathop{=}\limits_{\texttt{有限可加}}
		\lim\limits_{n\to\infty}\sum_{i=1}^n{P(A_i)}\mathop{=} \limits_{\texttt{收敛}}
	\sum_{i=1}^{\infty}P(A_i)\]

\item 推广可数可加:
	$ A_1,A_2\cdots $满足$ P(A_iA_j)=0$(弱于互斥),则$ P(A) = \sum{P(A_n)}$
\end{enumerate}

\subsection{Classical Definitions of Probability}

{\bf 古典概型}: 基本事件只有有限个且概率相同.

掷硬币$ n$次,取每种排列为基本事件,即为古典概型:
\[  P(\texttt{首次正面出现在k次}=\dfrac{1}{2^k})\]

掷硬币直到出现正面为止,基本事件$ \omega_k$为``首次正面出现在第k次'',则有无穷个基本事件,且
概率不同.
利用可数可加性,$ \omega_{\infty} = 0$,但不是不可能事件.

一般地,对于至多可数集合$ \Omega$,每个基本事件的概率都可求出时,$ \forall A
\subset \Omega, P(A) = \sum_{\omega \in A}P(\omega)$

若无限抛掷硬币,将排列作为基本事件,则有不可数个基本事件,此时若考虑等可能分析,则
每个基本事件概率为0.无法求出某个事件的概率(因为不可数个实数的和没有意义).

{\bf 几何概型}:随机现象的 样本空间充满某个可测区域,且任一点落在度量相同的子区域
内是等可能的.

{\bf Buffon投针的分析做法}:设针中点与最近平行线距$ x\in [0, \dfrac{d}{2}]$,与
直线成角$ \varphi\in[0,\pi]$,在上区域中求$ x \le \dfrac{l}{2}\sin{\varphi}$
部分的概率. $ P(A) =
\dfrac{\int_{0}^{\pi}{\dfrac{l}{2}\sin{\varphi}\mathrm{d}{\varphi}}}{\dfrac{d}{2}\pi}
= \dfrac{2l}{d\pi}$

古典/几何概型的另一个问题:Bertrand悖论--圆内一弦长度超过正三角形边长的概率由三
种解释.

原因:当可能结果有无穷个时,难以规定``等可能''这一概念,因此概率空间被模糊定义了.

\subsection{Conditional Probability}
概率空间$( \Omega, \mathcal{F}, P)$中$ P(B) > 0$.定义$ P_B(A) = P(A | B) = \dfrac{P(A\cap B)}{P(B)}$,
则可证$ (\Omega, \mathcal{F}, P_B)$也是概率空间.

  {\bf 乘法公式:}$ P(\bigcap_{i=1}^{n-1}A_i)>0$ (使得条件概率有意义)时,由定义归纳可得

\[ \Rightarrow P(\bigcap_{i=1}^nA_i) = \prod_{i=1}^n{P(A_i | \bigcap_{j=1}^{i-1}A_j)}\]

{\bf 可靠性函数与风险率:}设前$ t$时刻正常,$[t, t+\Delta t]$时段失效的概率为$ \lambda(t)\Delta t + o(\Delta t)$,求
设备在$ (0,t)$上无故障的概率.

设$ A_t$表示设备在$ (0,t)$内正常,$ P(\overline{A_{t+\Delta t}} | A_t) = \lambda(t)\Delta t + o(\Delta t)$.
\[ p(t + \Delta t) = P(A_t)P(A_{t+\Delta t}|A_t) = p(t)[1 - \lambda(t)\Delta t + o(\Delta t)]\Rightarrow \dfrac{dp(t)}{dt}=-\lambda(t)p(t)\]

注意到$ p(0)=1,$有$ p(t) = e^{-\Oldint_0^t{\lambda(s)ds}}$
\\

{\bf 全概率公式:}设$ B_1,B_2\cdots$为样本空间$ \Omega$的一个正划分,则
\[ P(A) =\sum_{i=1}^{\infty}{P(AB_i)}= \sum_{i=1}^{\infty}{P(B_i)P(A|B_i)}\]

{\bf 赌徒输光:}两人各有赌资$ i, n-i$,每次赌博胜者拿走对方1元,胜率分别为$ p,1-p$.

设$ A_i$表示甲有$ i$元,最终破产.$ B$表示某次甲胜,$ P(B) = p$,则有$ P(A_i|B)=P(A_{i+1}),P(A_i|\overline{B})=P(A_{i-1})$

于是$ P(A_i)=pP(A_{i+1}) + (1-p)P(A_{i-1})$,边界$ P(A_0)=1,P(A_n)=0$.

\[  P(A_i)=\left\{ \begin{array}{lc} 1-\dfrac{1-r^i}{1-r^n}  & p \ne \dfrac{1}{2} \\ 1-\dfrac{i}{n}& p = \dfrac{1}{2} \end{array}  \right.r = \dfrac{1-p}{p} \]

对赌场$ (n \to \infty)$,甲最终会输光的概率为$ P(A_i) = \min\{1,r^i\}$
\\

{\bf Polya 模型}:从黑球,红球中任取若干次.取出的红球与黑球个数确定的情形下,概率是否与次序相关.

若放回抽样,结果不影响下次,故概率相等.

若不放回抽样,前次结果影响后次,但概率仍与次序无关.

若放回若干同色球(传染病模型),每次取出会增加下次取出同色球的概率. 但结果与次序无关.

若放回若干异色球(安全模型),结果才与次序有关.
\\

{\bf 敏感问题问卷调查:}在问卷上要求每个人准备一枚硬币,对于指定的隐私题目,请填写人投掷一次硬币:如果正
面朝上,则如实填写个人的真实情况;如果反面朝上,那么就再投掷一次硬币,正面就填"是",反面就填"否".当然,若
第一次投掷硬币为正的话,填写人完全可以假装再投一次硬币来掩人耳目.

假设回收后有效问卷有$M$份,其中该问题答"是"的有$N$个人.如实填写了该问题的人平均有$\dfrac{M}{2}$个;在另
外$ \dfrac{M}{2}$人中,平均有$ \dfrac{M}{4}$人答的"是".因此,我们所需要的最终结果应该为$ \dfrac{(N-M)/4}{M/2} $
\\

{\bf Bayes:}$ P(B_i|A) = \dfrac{P(B_i)P(A|B_i)}{P(A)} = \dfrac{P(B_i)P(A|B_i)}{\sum_{t=1}^n{P(B_t)P(A|B_t)}}$

\subsection{Independence}
{\bf 定义:}$ P(AB)=P(A)P(B)$. 实际中以经验判断.

$ A,B$独立$ \Rightarrow A与\mathcal{F}_B$中任一事件独立.

多个事件相互独立:直观想法--$ A与\mathcal{F}_{B,C\cdots}$中任一事件独立.
\\

{\bf 定义}:其中任意$ k$个事件的交的概率等于概率的乘积.

无穷个事件相互独立:任意有限个事件相互独立.

$ A_1\cdots A_n$ 相互独立,则任意对其分组,各组事件分别产生的事件域相互独立.

相关系数:$ r(A,B)=\dfrac{P(AB)- P(A)P(B)}{\sqrt{P(A)[1-P(A)]P(B)[1-P(B)]}}$

$ \left \{\begin{matrix}-P(A)P(B)\\  -[1-P(A)][1-P(B)]\end{matrix} \right . \le P(AB)-P(A)P(B) \le \left \{ \begin{matrix} P(A)[1-P(B)]\\ P(B)[1-P(A)]\end{matrix}\right .
			\Rightarrow  |r(A,B)| \le 1$

$ r(A,B)=1\Leftrightarrow P(A)=P(AB)=P(B)$

$ r(A,B)>0\Leftrightarrow P(A|B)>P(A)\Leftrightarrow P(B|A)>P(B)$

% $File: fundamental.tex
% $Date: Mon Mar 05 20:14:01 2012 +0800
% Author: WuYuxin <ppwwyyxxc@gmail.com>
\section{代数基本定理}
实系数多项式在实数域内可唯一分解为一次因式与判别式小于零的二次因式乘积.

$ x^{2m}-1=(x-1) \prod_{k=1}^m{( x^2-2x\cos\dfrac{2k\pi}{2m+1}+1 )}$

$ x^{2m+1}-1 =(x-1)(x+1)\prod_{k=1}^m{(x^2-2x\cos\dfrac{k\pi}{m}+1)}$ 

$ x^{2m+1}+1=(x+1)\prod_{k=1}^m{( x^2-2x\cos\dfrac{(2k-1)\pi}{2m+1}+1 )}$

$ x^{2m}+1=\prod_{k=1}^m{(x^2-2x\cos\dfrac{(2k-1)\pi}{2m}+1)}$

取一些特殊值,可得等式:
$ \prod_{k=1}^m{\cos\dfrac{k\pi}{2m+1}}=\dfrac{1}{2^m}$,
$ \prod_{k=1}^{m-1}{\sin\dfrac{k\pi}{2m}}=\dfrac{\sqrt{m}}{2^{m-1}}$


$ f(x)$是$ n\ge2$次多项式,$ f(x)\ge 0,\forall x\in \mathbb{R}\Rightarrow \exists g(x),h(x),f(x)=g^2(x)+h^2(x)$
证:$ f(x)$的所有一次因式均有偶幂指数,且任意恒正的首一二次式可写为平方和.

%% $File: order.tex
% $Date: Mon Mar 05 15:53:40 2012 +0800
% $Author: wyx <ppwwyyxxc@gmail.com> 

\section{Order}
Definition: $  \delta_m(a)=\min \{x|a^x \equiv 1 \pmod m\} $ 

推广: $ a^d\equiv b^d \pmod p$,取倒数$ bb'\equiv1\pmod p$,则$ d=\delta_p(ab')$.性质类似
\\
若$ a^n \equiv 1 \pmod m$ ,则 $ \delta_m(a)\mid n $ .
否则设$ n=\delta_m(a)q+r,a^r \equiv a^n \equiv 1 $且$ r<\delta_m(a)$.矛盾

特别地,若 $ a^p\equiv 1 \pmod m$ , 则 $ \delta_m(a)=1 $ 或$ p$ 

Mersenne's Prime的因子特征:$ q\mid 2^p-1\Rightarrow p=\delta_q(2)\mid (q-1)\Rightarrow q\equiv 1 \pmod{2p} $ 
\\

$ (a,p)=1$,则在$ p^0,p^1,\ldots p^{a-1} \pmod a$中抽屉得 $ \exists d \le a-1: a|p^d-1\Rightarrow \delta_p(a)\le a-1$
\\

证明$ n \nmid 2^n-1$:

设$ n$最小素因子$ p$,则$ \delta_p(2) \mid (p-1,n)=(p-1,\dfrac{n}{p^{\alpha}})=1$.

或者利用递降:$ n\rightarrow \delta_n(2) ; (a,b)\rightarrow (b,(a,b))$
\\

$ n\mid 2^n+1\Rightarrow \delta_p(2)\mid (2n,p-1)=(2,p-1)\Rightarrow p=3$.

事实上有$ 3^k \mid 2^{3^k}+1$,以及$ n\mid 2^n+1 \Rightarrow m\mid 2^m+1,m=2^n+1$
\\

反证$ n \nmid m^{n-1}+1$:

设$ n-1=2^kt\Rightarrow m^{2^kt}\equiv -1 \pmod p\Rightarrow \delta_p(m)\nmid 2^kt,\delta_p(m)\mid 2^{k+1}t\Rightarrow 2^{k+1}\mid\delta_p(m)$

又$ \delta_p(m)\mid p-1,\therefore p \equiv 1 \pmod{2^{k+1}}.$
考虑到$ p$为$ n$任意素因子$ \Rightarrow n\equiv 1 \pmod{2^{k+1}}$,与$ n-1=2^kt$矛盾
\\
\\

关于$ r_k=\delta_{p^k}(a)$的求解($ p$为奇数).设$ p^{k_0}\parallel a^{r_1}-1$

i)当$ 1\le k \le k_0$时,$ a^{r_k}\equiv1 \pmod{p^k \rightarrow p}\Rightarrow r_1\mid r_k$

$ a^r\equiv1 \pmod{p^{k_0}\rightarrow p^k}\Rightarrow r_k\mid r_1. \therefore r_k=r_1$

ii)当$ k\ge k_0$时,对$k $归纳证明$ r_k=r_1 p^{k-k_0}$

引理:$ p^{k_0+i}\parallel a^{r_1p^i}-1\Leftrightarrow a^{r_1p^i}=1+p^{k_0+i}u,(u,p)=1$.

证明:归纳.
$ a^{r_1p^{i+1}}=(a^{r_1p^i})^p=(1+p^{k_0+i}u)^p=1+p^{k_0+i+1}(1+C_p^2u^2p^{k_0+i-1})$

引理中取$ i=k-k_0,$则$ a^{r_1p^{k-k_0}}\equiv 1 \pmod{p^k}\Rightarrow r_k \mid r_1p^{k-k_0}$

$ a^{r_k}\equiv 1\pmod{p^k \rightarrow p^{k-1}}\Rightarrow r_{k-1}\mid r_k \therefore r_1p^{k-k_0-1}\mid r_k \mid r_1p^{k-k_0}$

再取$ i=k-k_0-1$,由$ p^{k-1}\parallel a^{r_1p^{k-k_0-1}}-1$知$ a^{r_1p^{k-k_0-1}}\not \equiv 1 \pmod{p^k}.$


$  \therefore r_k=
\begin{cases}
r_1, & 1\le k \le k_0 \\
r_1p^{k-k_0}, & k\ge k_0 
\end{cases} 
$
\\

$ r_k=\delta_{2^k}(a)$的求解:

i)$ a=4k+1,2^{k_0}\parallel a-1,r_k=
\begin{cases}
1,& 1\le k \le k_0 \\ 
2^{k-k_0},& k\ge k_0 
\end{cases}
$ 

ii)$ a=4k+3,2^{k_0}\parallel a+1,r_k=
\begin{cases} 
1,& k=1 \\ 
2,& 2\le k\le k_0+1 \\ 
2^{k-k_0},& k\ge k_0+1 
\end{cases}$
\\

引理的推广:$ a^{mrp^i}=1+p^{k_0+i}u,(u,p)=1.$

设$ n=mrp^i$可得一命题:$ r=\delta_p(a),r\mid n,p^{\alpha}\parallel n\Rightarrow p^{\alpha}\parallel \dfrac{a^n-1}{a^r-1} $
\\

反证:对给定$ n,a$,不存在无穷个$ k,s.t.n^k\mid a^k-1$

i)$ n$含奇因子$ p,a^k\equiv 1 \pmod {p^k}\Rightarrow r_k=r_1p^{k-k_0}\mid k\Rightarrow k>r_1p^{k-k_0}\ge 3^{k-k_0}$不可能无穷个

ii)若$ k$为奇,则$ 2^k\mid a^k-1\Rightarrow 2^k\mid a-1,$只有有限个$ k$.

若$ k$为偶,$ a^{2l}\equiv 1 \pmod{2^l}.$当$ l>k_0$时,$ 2^{l-k_0}\mid l$不可能无穷个.
\\
\\
$ r_k=\delta_m(a^k)=\dfrac{r_1}{(r_1,k)}$.

证:设$ r'=\dfrac{r_1}{(r_1,k)}$.显然$ (r',\dfrac{k}{(r_1,k)})=1$

由定义,$ a^{kr_k}\equiv 1 \pmod m, a^{kr'}\equiv 1 \pmod m. \Rightarrow  r_1 \mid kr_k,r_k\mid r'$

$ \therefore r' =\dfrac{r_1}{(r_1,k)}\mid \dfrac{k}{(r_1,k)}r_k\Rightarrow r' \mid r_k. \therefore r'=r_k$

推论:有$ \varphi(r_1)$个$ k,s.t.(r_1,k)=1.$又$ a^0,a^1,\cdots,a^{r_1-1}$对模$ m$不同余

所以其中至少有$ \varphi(r_1)$个$ k,s.t.\delta_m(a^k)=r_1$.

即在模$ m$的一个缩系中至少有$ \varphi(r_1)$个$ k,s.t.r_k=r_1$
\\

若$ (m_1,m_2)=1,$则$ \delta_{m_1m_2}(a)=[\delta_{m_1}(a),\delta_{m_2}(a)]=[r_1,r_2]$

证:i)显然对$ \forall n \mid m,\delta_n(a)\mid \delta_m(a). \therefore [r_1,r_2] \mid \delta_{m_1m_2}(a)$ 

ii)$ a^{[r_1,r_2]}\equiv 1 \pmod{m_1,m_2 \rightarrow m_1m_2}\Rightarrow \delta_{m_1m_2}\mid [r_1,r_2]$

推论:$ (m_1,m_2)=1$,则对$ \forall a_1,a_2,\exists a,s.t.\delta_{m_1m_2}(a)=[\delta_{m_1}(a_1),\delta_{m_2}(a_2)]$

证:取$ a\equiv a_i \pmod{m_i},i=1,2$.则$ \delta_{m_i}(a)=\delta_{m_i}(a_i)$.由原命题即证.
\\

$ \min\{ n|2^n\equiv -1 \pmod p\}<\delta_p(2),$否则 $,2^{n-\delta_p(2)}\equiv 2^n \equiv -1$,与最小性矛盾.
\\

$ p=3k+2$时,$ x$取$ \mod p$完系,则$ x^3$亦遍历.否则$ x^3\equiv y^3\Rightarrow \delta_p(xy^{-1})\mid(3,p-1)=1$.矛盾
\\

无穷数列$  \dfrac{1}{9}(10^{k\delta_{9a}(10)}-1)(k\ge1) $ 中,每项均由1组成且均为$ a$的倍数
\\

奇素$ p,p^n|a^p-1\Rightarrow p^{n-1}|a-1$
\\

$ \exists n,s.t.p\parallel2^n-1\Rightarrow p\parallel 2^{p-1}-1$

证:假设$ p^2 \mid 2^{p-1}-1\Rightarrow \delta_{p^2}(2)\mid p-1.$

又$ 2^{pn}-1=(2^n-1)(2^{n(p-1)}+2^{n(p-2)}+\cdots+2^n+1)\equiv(2^n-1)p\equiv 0\pmod{p^2}$

$ \therefore \delta_{p^2}(2)\mid (pn,p-1)=(n,p-1)\mid n\Rightarrow 2^n\equiv 1 \pmod {p^2}$.矛盾
\\

奇素数$ p,{pn+1}$中含无穷多素数:

证:取$ x^p-1$的因子$ q,s.t.q \nmid x-1$ (why can?).则$ \delta_q(x)=p$.

设$ (q-1,p)=d$ ,则$ \exists u,v,s.t.u(q-1)+vp=d\Rightarrow x^d\equiv(x^{q-1})^n(x^p)^v\equiv 1 \pmod q\Rightarrow d=p$ 

$ \therefore p\mid q-1\Leftrightarrow q=pn+1$.又$ \dfrac{x^p-1}{x-1}$含无穷个素因子$ q$,可知$ {pn+1}$中有无穷多素数

%\input{theory.tex}
%% $File: function.tex
% $Date: Sat Mar 03 13:24:52 2012 +0800
% Author:  ppwwyyxxc@gmail.com

\section{Arithmetic Function}
$ d(n)$约数个数,$ \sigma (n)$约数和,$ \varphi(n)$缩系大小,均有积性

$ n=\prod{p_i^{\alpha_i}}$ 则$d(n)=\prod{(\alpha_i+1)},\sigma(n)=\prod{\frac{p_i^{\alpha_i}-1}{p_i-1}},\varphi(n)=n\prod({1-\frac{1}{p_i}})$
\\

$ d(n)$为奇$ \Leftrightarrow n=k^2$ ;$ \sigma(n)$为奇$ \Leftrightarrow n=k^2,2k^2$

$ \varphi(n)=\varphi(2n)\Leftrightarrow n$为奇.

$ \varphi(n) \mid n\Leftrightarrow n=1,2^{\alpha}3^{\beta}(\alpha \ge 1,\beta \ge 0)$

$ n$的最小正缩系元素和为$ \frac{1}{2}n\varphi(n)$.配对
\\

估界:

$ n$在$ [1,\sqrt{n}]$中约数至多$ \sqrt{n}$个,$ \therefore d(n)\le 2\sqrt{n}$

$ \sigma(n)=\frac{1}{2}\sum_{d \mid n}{d+\frac{n}{d}}\ge \frac{1}{2}d(n)2\sqrt{n}=\sqrt{n}d(n)$

$ \sigma(n)^2 \mathop \le \limits_{cauchy} d(n)\sum_{d\mid n}{d^2}=d(n)\sum_{d\mid n}{(\frac{n}{d})^2}\le n^2d(n)\sum{\frac{1}{k^2}}<2n^2d(n)$

$ \varphi(p^a)=p^a-p^{a-1}>p^{\frac{a}{2}},\varphi(2^a)>\frac{2^{\frac{a}{2}}}{2}\Rightarrow \varphi(n)>\frac{\sqrt{n}}{2}.n$为奇时有$ \varphi(n)>\sqrt{n}$

$ \varphi(n)\le n-1,d(n)+\varphi(n)\le n+1$当$ n$为合数时,$ \varphi(n)\le n-\sqrt{n}.$

\[  \sum_{d \mid n}{\varphi(d)}=\sum_{e_1=0}^{\alpha_1}{\varphi(p_1^{e_1})}\sum_{e_2=0}^{\alpha_2}{\varphi(p_2^{e_2})}\cdots =\prod_{i=1}^{r}{\sum_{j=0}^{\alpha_i}{\varphi(p_i^j)}}=\prod_{i=1}^{r}{p_i^{\alpha_i}}=n\]

\[ \frac{\varphi(mn)}{mn}=\prod_{p \mid mn}{(1-\frac{1}{p})}=\frac{\prod_{p \mid m}{(1-\frac{1}{p})}\prod_{p \mid n}{(1-\frac{1}{p})}}{\prod_{p \mid (m,n)}{(1-\frac{1}{p})}}=\frac{\frac{\varphi(m)}{m}\frac{\varphi(n)}{n}}{\frac{\varphi((m,n))}{(m,n)}} \]
$ \Rightarrow \varphi(mn)\varphi((m,n))=(m,n)\varphi(m)\varphi(n) $ 
\\

$ d(n)=\prod{(\alpha+1)}\ge 2^r,\varphi(n)\ge n \prod{(1-\frac{1}{2})}=\frac{n}{2^r}\Rightarrow d(n)\varphi(n)\ge n$

对$ \pi(n)=$小于$ n$的素数个数估界: 

设$ n=k^2l,k$有$ \sqrt{n}$种取法,$ l$为不同素数积,有$ 2^{\pi(n)}$种取法.

$ n\le \sqrt{n}2^{\pi(n)}\Rightarrow \pi(n)\ge \frac{1}{2}\log_2n$
\\

Fermat-Euler Theorem:
$ (a,m)=1\Rightarrow a^{\varphi(m)}\equiv 1 \pmod m$

证:取$ m$一组缩系$ x_1\cdots x_{\varphi(m)},$则$ ax_i$也构成一组缩系.$ \prod{ax_i}\equiv \prod{x_i}$

推广:$ a^m \equiv a^{m-\varphi(m)} \pmod m$

证:设$ m=m_1m_2:m_1$的素因子均被$ a$整除,而$ (m_2,a)=1,$则$ (m_1,m_2)=1$.

首先有$ a^{\varphi(m_2)}\equiv 1 \pmod{m_2}\Rightarrow a^{\varphi(m)\equiv 1 \pmod{m_2}}\Rightarrow a^m\equiv a^{m-\varphi(m)\pmod{m_2}}$.

于是只需$ a^m\equiv a^{m-\varphi(m)}\pmod{m_1}\Leftrightarrow m_1\mid a^{m-\varphi(m)}$ 

$\Leftrightarrow V_p(m_1)\le (m-\varphi(m))V_p(a)$

又$ V_p(m_1)=V_p(m)\le2^{V_p(m)-1}\le p^{V_p(m)-1}\le p^{V_p(m)-1}\varphi(\frac{m}{p^{V_p(m)}})$ 

$=p^{V_p(m)}\varphi(\frac{m}{p^{V_p(m)}})-\varphi(p^{V_p(m)})\varphi(\frac{m}{p^{V_p(m)}})=p^{V_p(m)}\varphi(\frac{m}{p^{V_p(m)}})-\varphi(m)$ 

$\le m-\varphi(m)\le (m-\varphi(m))V_p(a)$得证.
\\

一些等式:

$ (m,n)=1\Rightarrow m^{\varphi(n)}+n^{\varphi(m)}\equiv 1 \pmod{mn}$

$ a\varphi(a^kb^{k+1})=b\varphi(b^ka^{k+1})$

$n=4k+3 \Rightarrow \forall d \mid n ,d+\frac{n}{d}\equiv 0 \pmod 4\Rightarrow 4 \mid \sigma(n)$
\\

$ (m,n)=1,\{ a_i\}_{1}^{\varphi(m)},\{ b_i\}_1^{\varphi(n)}$为缩系,则

$ S=\{ mb_i+na_j | 1\le j\le \varphi(m),1\le i \le \varphi(n)\}$为mod $ mn$缩系.

$ 1.(S_k,mn)=1; $ 

$ 2.S_i\equiv S_j \pmod n\Rightarrow b_i\equiv b_j \pmod n\Rightarrow i=j;$ 

$ 3.|S|=\varphi(m)\varphi(n)=\varphi(mn)$;

%% $File: summary.tex
% $Date: Sat Jun 02 17:47:54 2012 +0800
% Author: Yuxin Wu <ppwwyyxxc@gmail.com>
\section{总结与感悟}
	本篇报告介绍了多项式方程组求解的结式方法,
	利用线性代数中的一些简单手段,就可以通过构造矩阵实现两个方程间的联合消元,
	进一步可以实现方程组的消元.
	随后,我们简要提及了更高效的抽象代数方法中的几个有价值的概念,
	将其与线性代数中的一些基本概念进行类比,加深了我们对这些代数概念的理解.

	20世纪以来现代科学技术突飞猛进,尤其是计算机科学的兴起为数学插上了腾飞的翅膀,
	数学理论同时也为计算机科学技术提供了广阔的舞台.
	现代的计算机技术为大型的符号计算提供了可能性,
	关键的问题就在于如何把抽象的代数理论算法化, 
	使计算机高效地处理形形色色的代数问题. 
	如今强大的计算机代数系统不仅是各类工程技术的助手,
	对纯粹科学研究也起着不可忽略的推动作用.
	在多项式代数理论上发展的计算机图形学,机器人运动学,密码学
	等一系列计算机专业相关领域和理论应运而生,
	也大规模的应用在我们的日常生活中.
	数学的规模之大,影响之深远,已经超越了所有的时代.
	虚拟现实,三维真实感动画,智能机械手臂等华丽实用的科技让人们切实感受到了数学博大精深,
	以及那种浑然天成的和谐美与对称美.
	
	这次接触计算机与数学领域重叠部分的前沿应用,
	既为未来的我们在计算机系的符号计算领域的学习打下基础,
	也有益于我们对已有的线性代数知识的理解和巩固.
	同时,在尝试用宏观和类比的眼光透视数学问题的过程中,我们也更加体会到了数学的美感,
	以及从变化中抓住不变,从紊乱中归纳条理,在偶然中发现必然,从混沌中整理秩序的科研精神.
	在完成这篇作品的过程中,
	从确定方向,查阅文献,类比研讨,抽象方法,再到研究实践应用的一系列过程里,我们感受到:
	在处理数学问题时要善于类比联想;由想激疑,在释疑中启悟;由疑反思,在思辨中省悟;由思导验,在体验中领悟;由验致用,在应用中彻悟。

\printbibliography
%\input{appendix.tex}

\end{document}

