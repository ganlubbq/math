% $File: abstract.tex
% $Date: Wed Mar 14 08:47:43 2012 +0800
% $Author: wyx <ppwwyyxxc@gmail.com>

\documentclass[a4paper]{article}
\usepackage{fontspec,amsmath,amssymb,zhspacing,verbatim,minted}
\usepackage[hyperfootnotes=false,colorlinks,linkcolor=blue,anchorcolor=blue,citecolor=blue]{hyperref}
\usepackage[sorting=none]{biblatex}
\usepackage[dvips]{graphicx}
\usepackage{subfigure}
\usepackage{indentfirst}
\zhspacing


\renewcommand{\abstractname}{摘要}
\renewcommand{\contentsname}{目录}
\renewcommand{\figurename}{图}
\defbibheading{bibliography}{\section{参考文献}}
\bibliography{refs.bib}

% \figref{label}: reference to a figure
\newcommand{\figref}[1]{\hyperref[fig:#1]{图\ref*{fig:#1}}}
% \secref{label}: reference to a section
\newcommand{\secref}[1]{\hyperref[sec:#1]{\ref*{sec:#1}节}}
\DeclareMathOperator{\LM}{\tiny{LM}}
\DeclareMathOperator{\LT}{\tiny{LT}}
\DeclareMathOperator{\rank}{\tiny{rank}}

\let\Oldsum\sum
\renewcommand{\sum}{\displaystyle\Oldsum}
\let\Oldprod\prod
\renewcommand{\prod}{\displaystyle\Oldprod}

% $File: mint-defs.tex
% $Date: Sat Feb 16 22:59:11 2013 +0800
% $Author: wyx <ppwwyyxxc@gmail.com>

\usepackage{xparse}


% \inputmintedConfigured[additional minted options]{lang}{file path}{
\newcommand{\inputmintedConfigured}[3][]{\inputminted[fontsize=\footnotesize,
	label=#3,linenos,frame=lines,framesep=0.8em,tabsize=4,#1]{#2}{#3}}

% \phpsrc[additional minted options]{file path}: show highlighted php source
\newcommand{\phpsrc}[2][]{\inputmintedConfigured[#1]{php}{#2}}
% \phpsrcpart[additional minted options]{file path}{first line}{last line}: show part of highlighted php source
\newcommand{\phpsrcpart}[4][]{\phpsrc[firstline=#3,firstnumber=#3,lastline=#4,#1]{#2}}
% \phpsrceg{example id}
\newcommand{\phpeg}[1]{\inputminted[startinline,
	firstline=2,lastline=2]{php}{res/php-src-eg/#1.php}}

\newcommand{\txtsrc}[2][]{\inputmintedConfigured[#1]{text}{#2}}
\newcommand{\txtsrcpart}[4][]{\txtsrc[firstline=#3,firstnumber=#3,lastline=#4,#1]{#2}}

\newcommand{\pysrc}[2][]{\inputmintedConfigured[#1]{py}{#2}}
\newcommand{\pysrcpart}[4][]{\pysrc[firstline=#3,firstnumber=#3,lastline=#4,#1]{#2}}

\newcommand{\confsrc}[2][]{\inputmintedConfigured[#1]{squidconf}{#2}}
\newcommand{\confsrcpart}[4][]{\confsrc[firstline=#3,firstnumber=#3,lastline=#4,#1]{#2}}

\newcommand{\cppsrc}[2][]{\inputmintedConfigured[#1]{cpp}{#2}}
\newcommand{\cppsrcpart}[4][]{\cppsrc[firstline=#3,firstnumber=#3,lastline=#4,#1]{#2}}

\renewcommand{\P}[1]{\text{P}\left(#1\right)}
\renewcommand{\Pr}[1]{\text{Pr}\left\{#1\right\}}
\newcommand{\Px}[2]{\text{P}_{#1}\left(#2\right)}
\newcommand{\E}[1]{\text{E}\left[#1\right]}
\newcommand{\Ex}[1]{\text{E}#1}
\newcommand{\Var}[1]{\text{Var}\left[#1\right]}
%\newcommand{\Cov}[2]{\text{Cov}\left[#1,#2\right]}
%\newcommand{\Cov}[1]{\text{Cov}\left[#1 \right]}
\renewcommand{\T}[1]{\Theta\left(#1\right)}
\newcommand{\real}{\mathbb{R}}
\newcommand{\card}[1]{\left\|#1\right\|}
\newtheorem{lemma}{Lemma}

\NewDocumentCommand\Cov{mg}{
    \text{Cov}\left[ #1 \IfNoValueTF{#2}{}{,#2}\right]
 }

\newcommand{\qed}{\hfill \ensuremath{\Box}}



\title{Abstract}
\author{\\(ppwwyyxxc@gmail.com)}

\begin{document}
\maketitle

\tableofcontents
% $File: structure.tex
% $Date: Wed Mar 07 14:36:27 2012 +0800
% Author: ppwwyyxxc@gmail.com
\section{Structure}
群:一个集合与一个二元运算,满足封闭,结合,单位元,逆元

环:加法构成交换群,乘法构成半群(封闭,结合),乘法对加法左右分配.记加法单位元为0

若环中乘法交换,称为交换环.

若乘法存在单位元,称为有单位元环,记其为1

环$ R$中$\exists b\ne 0,s.t.ab=0$,称$ a$为左零因子.若没有非平凡零因子,称为
无零因子环.

有单位元的无零因子交换环称为整环.

$ R_1$成为$ R$的充要条件是$ R_1$对$ R$的减法与乘法封闭.(减法才能得出加法逆元存在)

在有单位元环中,存在逆元的元素称为可逆元

域:加法与乘法均构成交换群
\\


%% $File: order.tex
% $Date: Mon Mar 05 15:53:40 2012 +0800
% $Author: wyx <ppwwyyxxc@gmail.com> 

\section{Order}
Definition: $  \delta_m(a)=\min \{x|a^x \equiv 1 \pmod m\} $ 

推广: $ a^d\equiv b^d \pmod p$,取倒数$ bb'\equiv1\pmod p$,则$ d=\delta_p(ab')$.性质类似
\\
若$ a^n \equiv 1 \pmod m$ ,则 $ \delta_m(a)\mid n $ .
否则设$ n=\delta_m(a)q+r,a^r \equiv a^n \equiv 1 $且$ r<\delta_m(a)$.矛盾

特别地,若 $ a^p\equiv 1 \pmod m$ , 则 $ \delta_m(a)=1 $ 或$ p$ 

Mersenne's Prime的因子特征:$ q\mid 2^p-1\Rightarrow p=\delta_q(2)\mid (q-1)\Rightarrow q\equiv 1 \pmod{2p} $ 
\\

$ (a,p)=1$,则在$ p^0,p^1,\ldots p^{a-1} \pmod a$中抽屉得 $ \exists d \le a-1: a|p^d-1\Rightarrow \delta_p(a)\le a-1$
\\

证明$ n \nmid 2^n-1$:

设$ n$最小素因子$ p$,则$ \delta_p(2) \mid (p-1,n)=(p-1,\dfrac{n}{p^{\alpha}})=1$.

或者利用递降:$ n\rightarrow \delta_n(2) ; (a,b)\rightarrow (b,(a,b))$
\\

$ n\mid 2^n+1\Rightarrow \delta_p(2)\mid (2n,p-1)=(2,p-1)\Rightarrow p=3$.

事实上有$ 3^k \mid 2^{3^k}+1$,以及$ n\mid 2^n+1 \Rightarrow m\mid 2^m+1,m=2^n+1$
\\

反证$ n \nmid m^{n-1}+1$:

设$ n-1=2^kt\Rightarrow m^{2^kt}\equiv -1 \pmod p\Rightarrow \delta_p(m)\nmid 2^kt,\delta_p(m)\mid 2^{k+1}t\Rightarrow 2^{k+1}\mid\delta_p(m)$

又$ \delta_p(m)\mid p-1,\therefore p \equiv 1 \pmod{2^{k+1}}.$
考虑到$ p$为$ n$任意素因子$ \Rightarrow n\equiv 1 \pmod{2^{k+1}}$,与$ n-1=2^kt$矛盾
\\
\\

关于$ r_k=\delta_{p^k}(a)$的求解($ p$为奇数).设$ p^{k_0}\parallel a^{r_1}-1$

i)当$ 1\le k \le k_0$时,$ a^{r_k}\equiv1 \pmod{p^k \rightarrow p}\Rightarrow r_1\mid r_k$

$ a^r\equiv1 \pmod{p^{k_0}\rightarrow p^k}\Rightarrow r_k\mid r_1. \therefore r_k=r_1$

ii)当$ k\ge k_0$时,对$k $归纳证明$ r_k=r_1 p^{k-k_0}$

引理:$ p^{k_0+i}\parallel a^{r_1p^i}-1\Leftrightarrow a^{r_1p^i}=1+p^{k_0+i}u,(u,p)=1$.

证明:归纳.
$ a^{r_1p^{i+1}}=(a^{r_1p^i})^p=(1+p^{k_0+i}u)^p=1+p^{k_0+i+1}(1+C_p^2u^2p^{k_0+i-1})$

引理中取$ i=k-k_0,$则$ a^{r_1p^{k-k_0}}\equiv 1 \pmod{p^k}\Rightarrow r_k \mid r_1p^{k-k_0}$

$ a^{r_k}\equiv 1\pmod{p^k \rightarrow p^{k-1}}\Rightarrow r_{k-1}\mid r_k \therefore r_1p^{k-k_0-1}\mid r_k \mid r_1p^{k-k_0}$

再取$ i=k-k_0-1$,由$ p^{k-1}\parallel a^{r_1p^{k-k_0-1}}-1$知$ a^{r_1p^{k-k_0-1}}\not \equiv 1 \pmod{p^k}.$


$  \therefore r_k=
\begin{cases}
r_1, & 1\le k \le k_0 \\
r_1p^{k-k_0}, & k\ge k_0 
\end{cases} 
$
\\

$ r_k=\delta_{2^k}(a)$的求解:

i)$ a=4k+1,2^{k_0}\parallel a-1,r_k=
\begin{cases}
1,& 1\le k \le k_0 \\ 
2^{k-k_0},& k\ge k_0 
\end{cases}
$ 

ii)$ a=4k+3,2^{k_0}\parallel a+1,r_k=
\begin{cases} 
1,& k=1 \\ 
2,& 2\le k\le k_0+1 \\ 
2^{k-k_0},& k\ge k_0+1 
\end{cases}$
\\

引理的推广:$ a^{mrp^i}=1+p^{k_0+i}u,(u,p)=1.$

设$ n=mrp^i$可得一命题:$ r=\delta_p(a),r\mid n,p^{\alpha}\parallel n\Rightarrow p^{\alpha}\parallel \dfrac{a^n-1}{a^r-1} $
\\

反证:对给定$ n,a$,不存在无穷个$ k,s.t.n^k\mid a^k-1$

i)$ n$含奇因子$ p,a^k\equiv 1 \pmod {p^k}\Rightarrow r_k=r_1p^{k-k_0}\mid k\Rightarrow k>r_1p^{k-k_0}\ge 3^{k-k_0}$不可能无穷个

ii)若$ k$为奇,则$ 2^k\mid a^k-1\Rightarrow 2^k\mid a-1,$只有有限个$ k$.

若$ k$为偶,$ a^{2l}\equiv 1 \pmod{2^l}.$当$ l>k_0$时,$ 2^{l-k_0}\mid l$不可能无穷个.
\\
\\
$ r_k=\delta_m(a^k)=\dfrac{r_1}{(r_1,k)}$.

证:设$ r'=\dfrac{r_1}{(r_1,k)}$.显然$ (r',\dfrac{k}{(r_1,k)})=1$

由定义,$ a^{kr_k}\equiv 1 \pmod m, a^{kr'}\equiv 1 \pmod m. \Rightarrow  r_1 \mid kr_k,r_k\mid r'$

$ \therefore r' =\dfrac{r_1}{(r_1,k)}\mid \dfrac{k}{(r_1,k)}r_k\Rightarrow r' \mid r_k. \therefore r'=r_k$

推论:有$ \varphi(r_1)$个$ k,s.t.(r_1,k)=1.$又$ a^0,a^1,\cdots,a^{r_1-1}$对模$ m$不同余

所以其中至少有$ \varphi(r_1)$个$ k,s.t.\delta_m(a^k)=r_1$.

即在模$ m$的一个缩系中至少有$ \varphi(r_1)$个$ k,s.t.r_k=r_1$
\\

若$ (m_1,m_2)=1,$则$ \delta_{m_1m_2}(a)=[\delta_{m_1}(a),\delta_{m_2}(a)]=[r_1,r_2]$

证:i)显然对$ \forall n \mid m,\delta_n(a)\mid \delta_m(a). \therefore [r_1,r_2] \mid \delta_{m_1m_2}(a)$ 

ii)$ a^{[r_1,r_2]}\equiv 1 \pmod{m_1,m_2 \rightarrow m_1m_2}\Rightarrow \delta_{m_1m_2}\mid [r_1,r_2]$

推论:$ (m_1,m_2)=1$,则对$ \forall a_1,a_2,\exists a,s.t.\delta_{m_1m_2}(a)=[\delta_{m_1}(a_1),\delta_{m_2}(a_2)]$

证:取$ a\equiv a_i \pmod{m_i},i=1,2$.则$ \delta_{m_i}(a)=\delta_{m_i}(a_i)$.由原命题即证.
\\

$ \min\{ n|2^n\equiv -1 \pmod p\}<\delta_p(2),$否则 $,2^{n-\delta_p(2)}\equiv 2^n \equiv -1$,与最小性矛盾.
\\

$ p=3k+2$时,$ x$取$ \mod p$完系,则$ x^3$亦遍历.否则$ x^3\equiv y^3\Rightarrow \delta_p(xy^{-1})\mid(3,p-1)=1$.矛盾
\\

无穷数列$  \dfrac{1}{9}(10^{k\delta_{9a}(10)}-1)(k\ge1) $ 中,每项均由1组成且均为$ a$的倍数
\\

奇素$ p,p^n|a^p-1\Rightarrow p^{n-1}|a-1$
\\

$ \exists n,s.t.p\parallel2^n-1\Rightarrow p\parallel 2^{p-1}-1$

证:假设$ p^2 \mid 2^{p-1}-1\Rightarrow \delta_{p^2}(2)\mid p-1.$

又$ 2^{pn}-1=(2^n-1)(2^{n(p-1)}+2^{n(p-2)}+\cdots+2^n+1)\equiv(2^n-1)p\equiv 0\pmod{p^2}$

$ \therefore \delta_{p^2}(2)\mid (pn,p-1)=(n,p-1)\mid n\Rightarrow 2^n\equiv 1 \pmod {p^2}$.矛盾
\\

奇素数$ p,{pn+1}$中含无穷多素数:

证:取$ x^p-1$的因子$ q,s.t.q \nmid x-1$ (why can?).则$ \delta_q(x)=p$.

设$ (q-1,p)=d$ ,则$ \exists u,v,s.t.u(q-1)+vp=d\Rightarrow x^d\equiv(x^{q-1})^n(x^p)^v\equiv 1 \pmod q\Rightarrow d=p$ 

$ \therefore p\mid q-1\Leftrightarrow q=pn+1$.又$ \dfrac{x^p-1}{x-1}$含无穷个素因子$ q$,可知$ {pn+1}$中有无穷多素数

%% $File: polymial_ring.tex
% $Date: Mon Mar 05 13:13:40 2012 +0800
% Author: WuYuxin <ppwwyyxxc@gmail.com>
\section{一元多项式环}
K为一数域,$ K[x]$为主理想环,但$ Z[x]$不是

一个$ n$次多项式能被其导数整除$ \Leftrightarrow f(x)=a(cx+b)^n$

必要性:由$ \dfrac{f(x)}{(f(x),f'(x))}=\dfrac{f(x)}{\LM(f'(x))}$
为一次且无重因式即得.

% $File: root.tex
% $Date: Tue Mar 06 13:24:08 2012 +0800
% Author: WuYuxin <ppwwyyxxc@gmail.com>
\section{根}
实系数多项式在实数域内可唯一分解为一次因式与判别式小于零的二次因式乘积.

$ x^{2m}-1=(x-1) \prod_{k=1}^m{( x^2-2x\cos\dfrac{2k\pi}{2m+1}+1 )}$

$ x^{2m+1}-1 =(x-1)(x+1)\prod_{k=1}^m{(x^2-2x\cos\dfrac{k\pi}{m}+1)}$ 

$ x^{2m+1}+1=(x+1)\prod_{k=1}^m{( x^2-2x\cos\dfrac{(2k-1)\pi}{2m+1}+1 )}$

$ x^{2m}+1=\prod_{k=1}^m{(x^2-2x\cos\dfrac{(2k-1)\pi}{2m}+1)}$

取一些特殊值,可得等式:
$ \prod_{k=1}^m{\cos\dfrac{k\pi}{2m+1}}=\dfrac{1}{2^m}$,
$ \prod_{k=1}^{m-1}{\sin\dfrac{k\pi}{2m}}=\dfrac{\sqrt{m}}{2^{m-1}}$


$ f(x)$是$ n\ge2$次多项式,$ f(x)\ge 0,\forall x\in \mathbb{R}\Rightarrow \exists g(x),h(x),f(x)=g^2(x)+h^2(x)$

证:$ f(x)$的所有一次因式均有偶幂指数,且任意恒正的首一二次式可写为平方和.

首一多项式$ f(x)=x^n+a_{n-1}x^{n-1}+\cdots+a_1x+a_0$的根为$ c_1\cdots c_n$,定义其判别式为
\[ \begin{aligned}
	D(f) & =\prod_{1\le j< i\le n}{(c_i-c_j)^2}=
		\begin{vmatrix} 
		1 & 1 & \cdots & 1\\
		c_1 & c_2 & \cdots & c_n \\
		\vdots & \vdots & & \vdots \\
		c_1^{n-1} & c_2^{n-1} & \cdots & c_n^{n-1}\\
		\end{vmatrix}
		\begin{vmatrix}
		1 & c_1 & \cdots & c_1^{n-1}\\
		1 & c_2 & \cdots & c_2^{n-1}\\
		\vdots & \vdots & & \vdots \\
		1 & c_n & \cdots & c_n^{n-1}\\
		\end{vmatrix} \\
		& = \begin{vmatrix}
			s_0 & s_1 & \cdots & s_{n-1}\\
			s_1 & s_2 & \cdots & s_{n}\\
			\vdots & \vdots & & \vdots \\
			s_{n-1} & s_{n} & \cdots & s_{2n-2}\\
			\end{vmatrix}
\end{aligned} \]

幂和式可与初等对称多项式互化:

$ s_k=\sum_{i=1}^{n}{x_i^k}=\begin{vmatrix}
\sigma_1  & 1        & 0		& 0 &\cdots  & 0 & 0 \\
2\sigma_2 & \sigma_1 & 1		& 0 & \cdots & 0 & 0\\
3\sigma_3 & \sigma_2 & \sigma_1 & 1 & \cdots & 0 & 0\\
\vdots	  & \vdots   & \vdots   &\vdots & \ddots & \vdots & \vdots \\
\vdots	  & \vdots   & \vdots   &\vdots & \ddots & \vdots & \vdots \\
(k-1)\sigma_{k-1} &\sigma_{k-2} &\sigma_{k-3} & \sigma_{k-4}& \cdots&\sigma_1&1\\
k\sigma_k & \sigma_{k-1} &\sigma_{k-2} &\sigma_{k-3} &\cdots &\sigma_2 &\sigma_1 
\end{vmatrix}$

$ \sigma_k=\sum_{1\le j_1<\cdots <j_k\le n}{\prod_{t=1}^k{x_{j_t}}} = \dfrac{1}{k!}\begin{vmatrix} 
s_1		& 1		& 0		& \cdots & 0 & 0\\
s_2 	& s_1	& 2 	& \cdots & 0 & 0\\
s_3 	& s_2 	& s_1	& \cdots & 0 & 0\\
\vdots  & \vdots& \vdots& \ddots & \vdots & \vdots\\
s_{k-1} &s_{k-2}&s_{k-3}& \cdots &s_1 &k-1\\
s_{k}	&s_{k-1}&s_{k-2}& \cdots &s_2 &s_1
\end{vmatrix}$

% $File: polymial_field.tex
% $Date: Wed Mar 07 14:46:48 2012 +0800
% Author: WuYuxin <ppwwyyxxc@gmail.com>
\section{polynomial in field}
(Zero Function) 若$ k$ 为无限域,则$ f:k^n\rightarrow k$是零函数(值域为0)
$ \Leftrightarrow f=0$(零多项式)

于是,两个多项式诱导同一个函数当且仅当它们相等

有限域上未必,如$ \mathbb{Z}_3[x]$中$ x^3+2x^2+2$与$ 2x^2+x+2$是同一函数.
\\

判断$ f(x)=3x^5+11x^2+7$不可约:将其转化到$ \mathbb{Z}_2$域中,得

$ \tilde{f}(x)=x^2(x^3+\bar{1})+\bar{1}=x^2(x+\bar{1})(x^2+x+\bar{1})+\bar{1}$

但$ \mathbb{Z}_2$上的一次多项式只有$ x,x+\bar{1}$,不可约二次多项式只有$ x^2+x+\bar{1},$
都不是$ \tilde{f}(x)$的因式
\\

$ \mathbb{Z}_p$上的函数都是$ \mathbb{Z}_p$上的一元多项式函数:

{\bf 证}:考虑所有次数小于$ p$的一元多项式集合$ \mathbb{W}$,其中任两个不同的多项式诱导不同的函数,
否则由Lagrange定理可得矛盾.

考虑每个系数的取法,$ |\mathbb{W}|=p^p$,与$ \mathbb{Z}_p$上函数的总个数相等.得证.
\\

$ \eta^n =1,\eta^l\ne 1(1\le l<n),$称$ \eta$为本原$ n$次单位根

且有$ \eta^k$为本原$ \dfrac{n}{(n,k)}$次单位根

设$ \eta_1\cdots \eta_{\varphi(n)}$是全部本原$ n$次单位根

定义$ n$阶分圆多项式$ f_n(x)=\prod_{i=1}^{\varphi(n)}{( x-\eta_i )}$

有:$ f_n(x)$是集合$ \{ f(x)\in \mathbb{Q}[x]\mid f(\eta)=0\}$中次数最低的首一多项式(极小多项式)

$ f_n(x)$在$ \mathbb{Q}$上不可约.

$ x^n-1=\prod_{d \mid n}{f_d(x)}$

% $File: resultant.tex
% $Date: Sat Jun 02 17:24:14 2012 +0800
% Author: WuYuxin <ppwwyyxxc@gmail.com>
\section{结式方法}
\subsection{原理}
	消元是解方程组的基本原理.
	对于解线性方程组,由于方程只有一次,可以直接将每个方程中的一变元用其余变元表出,再代入消元.
	但对于多项式方程组,由于很多类型的高次方程无根式解,即使有也大多极其复杂,难以使用传统的代入消元手段进行消元,
	而基于换元或方程相加减等的消元方法太过于依赖结构,不具有计算机处理所需的通用性.
	因此需要引入其他的消元方法,结式(Resultant)就是一种.

	首先考虑复数域上关于$ x$的两个多项式方程构成的方程组
	\[  \begin{cases} 
			f(x)=\sum_{i=0}^{n}{a_ix^{n-i}}=0 \\
			g(x)=\sum_{i=0}^{m}{b_ix^{m-i}}=0
	\end{cases} , \texttt{其中} a_0b_0\ne0\]

	这个方程组的解,实际上就是指两多项式的公共复根.

	$ f(x)=0,g(x)=0$有公共复根

	$ \Leftrightarrow \exists f_1(x), g_1(x), \deg f_1(x) < n, \deg g_1(x) < m, \dfrac{f(x)}{f_1(x)} = \dfrac{g(x)}{g_1(x)}$

	$ \Leftrightarrow \exists f_1(x),g_1(x),\deg f_1(x)<n,\deg g_1(x)<m,f(x)g_1(x)=g(x)f_1(x)$

	于是设$ f_1(x)=\sum_{i=0}^{n-1}{u_ix^{n-1-i}},g_1(x)=\sum_{i=0}^{m-1}{v_ix^{m-1-i}}$,
	代入上式,比较左右两边所有$ m+n$个项的系数,可得$ m+n$个线性方程.

	\[ \left \{  \begin{array}{lrclr}
	a_0v_0		&			&= &b_0u_0		&		\\
	a_1v_0+a_0v_1&			&= &b_1u_0+b_0u_1&		\\
	\cdots \cdots &\cdots	&= & \cdots\cdots &\cdots		\\
			&{a_nv_{m-2}+a_{n-1}v_{m-1}}&=&		&b_mu_{n-2}+b_{m-1}u_{n-1}\\
			& a_nv_{m-1}&= &				&b_mu_{n-1}\\
	\end{array}\right . \]

	这个关于$(v_0,\cdots,v_{m-1},u_0,\cdots,u_{n-1})$ 的$ m+n$元齐次线性方程组必须有非零解
	,于是其系数行列式
	
	\[ A=\begin{vmatrix} 
		a_0	&	&	&	&	 &		& b_0 &		&	 &	\\
		a_1	&a_0&	&	&			&	& b_1 & b_0	&	&	\\
  \cdots &\cdots &\cdots &\cdots	&  \cdots&	& \cdots &\cdots	&	\cdots	&	\\
		 &	&	&	&a_n &a_{n-1}   &		&	& b_m  & b_{m-1} \\
		 &	&	&	&	 &a_n		&		&	&	  & b_m\\
	\end{vmatrix} = 0\]

	也即其转置
	\[A^{T}= \begin{vmatrix}
	a_0 & a_1 & \cdots & \cdots & \cdots & \cdots & a_n		&		&	&   \\
		& a_0 & a_1	   & \cdots & \cdots & \cdots & \cdots  &a_n   &	&	\\
		&	  & \cdots & \cdots & \cdots & \cdots & \cdots &		&	&	\\
		&	  &		   & a_0	& a_1	 & \cdots & \cdots & \cdots &	& a_n\\
	b_0 & b_1 & \cdots & b_m	&		 &		  &		   &		&	&   \\
		& b_0 & b_1	   & \cdots  &b_m   &	    	&		&		&	&	\\
		&	  & \cdots & \cdots & \cdots & \cdots & \cdots &		&	&	\\
		&	  &		   &		&		&		& b_0		& b_1	& \cdots & b_m\\
	\end{vmatrix} =0\]
	注意到$ A^{T}$仅由多项式$ f,g$的系数决定,可记$ Res_x(f,g)=|A^{T}|,$
	称其为多项式$ f,g$的{\bf Sylvester结式(Sylvester's Resultant) },简称结式.

	根据上述讨论,我们有结论:$ f(x),g(x)$有重根$ \Leftrightarrow Res_x(f,g) = 0$
	
	\vspace{13pt}
	再考虑二元多项式方程组:

	$ \begin{cases}
		f(x,y)=0 \\g(x,y)=0 
	\end{cases}$,以$ x$为主元,视$ y$为常数,可求得结式$ P(y) = R_x(f,g)$,是$ y$的多项式.

	由之前对一元方程组的讨论可知,对$ P(y)$的每个零点$ y_0,f(x,y_0),g(x,y_0)$有公共复根.
	反之,对于原方程组的每个解$ (x_0,y_0), $必有$ P(y_0) = 0$

	于是,对原方程组的求解转化为对一元方程$ P(y) = 0$的求解,成功实现了消元,降低了问题的难度.
	
	\vspace{13pt}
	对两个方程构成的含有多个变元的方程组,可以使用相同手段消去其中某个变元.
	而对于超过两个方程构成的方程组,可以每次取其中两个方程计算结式,例如解如下方程组:
	\[  \begin{cases} f_1 = xy+z-5 = 0\\ f_2 = x+y+z-6 = 0 \\ f_3 = x^2-2xy+y^2-2z = 0
	\end{cases}\]

	$ Res_x(f_1,f_2) = y^2+(z-6)y-z+5 $ 

	$ Res_x(f_2,f_3)=4y^2+(4z-24)y + z^2-14z+36$

	$ Res_y(Res_x(f_1,f_2),Res_x(f_2,f_3)) = (z-2)^2(z-8)^2$

	于是消去了$ y,z$,得到$ z=2,z=8,$代回原方程可求得完整解.

	对于多个方程构成的方程组,也有多元结式理论可以用于解决消元问题.\cite{ideals}

\subsection{应用}
	对于一元方程组,结式方法的一个重要应用是判断多项式是否有重根.
	考虑多项式$ f(x) = \prod_{i=1}^{k}{(x-x_i)^{r_i}}$,
	显然对于每个$ i$,有 $ r_i >1 \Leftrightarrow  f'(x_i) = 0 \Leftrightarrow x_i $ 是$f(x)$与$ f'(x)$的公共复根.
	所以, $ f(x)=0$有重根$ \Leftrightarrow Res(f,f')=0$.
	
	\vspace{5pt}
	另外,上述的二元方程组的结式求解方法可应用于物理解题中常常能遇到的一类代数曲面/曲线隐式化问题.
	例如,用参数方程表示的曲线
	$ \begin{cases}x=\dfrac{-t^2+2t}{t^2+1} \\ y=\dfrac{2t^2+2t}{t^2+1} \end{cases}$

	将两式都写为关于$ t$的多项式: 
	$\begin{cases}f(t)=(x+1)t^2 -2t +x\\ g(t)=(y-2)t^2-2t+y \end{cases}$

	排除掉$ x+1=y-2=0$的情形,计算结式$ Res_t(f,g)= 8x^2-4xy+5y^2+12x-12y$ 
	
	于是$8x^2-4xy+5y^2+12x-12y=0,(x,y) \ne (-1,2)$为曲线隐式化后的结果.

\printbibliography
%\input{appendix.tex}

\end{document}

