% $File: order.tex
% $Date: Wed Mar 07 17:04:08 2012 +0800
% Author: ppwwyyxxc@gmail.com
\section{order}
偏序:
反对称,传递,自反(eg. 整除关系).
也称半序.

拟序:
非自反,传递

全序:
反对称,传递,完全(任两者可比).
也称线序/简单序.

完全性蕴含自反性$ \Rightarrow $全序蕴涵偏序

良序:
任意非空子集都有最小元的偏序.

良序集一定是全序集,有限全序集一定是良序集.

定义一个非良序集合上的全序关系使之成为良序集,成为良序化.

(良序定理)任意集合可以良序化.是选择公理的等价形式之一.

\\


偏序集$ \langle A,\le \rangle$上的子集$ B$中任两元素可比,称$ B$是$ A$的一条链;
任两元素不可比,称为反链.

(Zorn)若一个偏序集的每条链都有上界,则此集中有极大元.

Theorem:
设$ A$中最长链长度为$ n$,则将$ A$中元素分为不交的反链,反链个数至少为$ n$.

{\bf 证 }:归纳.设$ A$中最长链长为$ k+1$,
令$ M$为$ A$的极大元的集合,则$ M$为反链,且$ A-M$中最长链长为$ k$.归纳即得.
\\

Erdos-Szekeres:
$ mn+1$个元素的$ A$中或存在长为$ n+1$的链,或存在长为$ m+1$的反链.

