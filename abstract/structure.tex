% $File: structure.tex
% $Date: Mon Jun 11 21:42:22 2012 +0800
% Author: ppwwyyxxc@gmail.com
\section{Structure}
一个代数系统(封闭)中有左单位元(左乘任何元素都仍等于那个元素)与右单位元,则必相等且等于单位元,且单位元唯一.

一个有单位元的代数系统,若某$ x,\exists x', x'x=e$,则称$ x$左可逆.
若此代数系统适合结合律,则左右均可逆的元素的左右逆元必相等,称为逆元,且逆元唯一.

含单位元的半群(结合)称为幺群

若幺群$ M$中存在$ g,s.t.\forall a\in M, a=g^m$,则称$ M$为循环幺群,$ g$为$ M$的一个生成元.
显然循环幺群必可交换.

$ (A, \cdot), (B, \star),f:A \rightarrow B. \forall a,b \in A, f(a \cdot b)=f(a)\star f(b).$
称$ f$是$ A$到$ B$的同态.$ f$为单射,满射,双射时,分别称作单同态,满同态,同构.

群:一个集合与一个二元运算,满足封闭,结合,单位元,逆元,记做$ (G, \cdot, e)$
事实上可证,只需有左单位元和左逆元的半群就是群.

欲证$ H$是$ G$的子群,需证$ H$仍满足封闭,单位元,逆元.或合为一个条件:$ \forall a,b\in H, ab^{-1} \in H$.

循环群$ G$由$ a$生成,记做$ G=\langle a \rangle,$阶数$ O\langle a \rangle$.
若$ O \langle a \rangle = \infty, G$中生成元只有$ a,a^{-1}.$
若$ O \langle a \rangle = n ,G$中有$ \varphi(n)$个生成元.

循环群的子群都是循环群,无限循环群的非平凡子群是无限群.
有限循环群$ G=\langle a \rangle,|G|=n,a^k$是$ H$中$ a$的最小正幂,则$ |H| = \dfrac{n}{k}$,
且对$ n$的每个正因子$ d,G$有且只有一个$ d$阶子群.

循环群要么和$ (Z, +)$同构,要么和$ (Z_n, +)(mod)$同构.

环:加法构成交换群,乘法构成半群,乘法对加法左右分配.记加法单位元为0
若环中乘法交换,称为交换环.

若乘法存在单位元,称为有单位元环,记其为1

环$ R$中$\exists b\ne 0,s.t.ab=0$,称$ a$为左零因子.若没有非平凡零因子,称为
无零因子环.

有单位元的无零因子交换环称为整环.

$ R_1$成为$ R$的充要条件是$ R_1$对$ R$的减法与乘法封闭.(减法才能得出加法逆元存在)

在有单位元环中,存在逆元的元素称为可逆元

域:加法与乘法均构成交换群
\\

